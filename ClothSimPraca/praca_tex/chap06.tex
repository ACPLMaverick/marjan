\chapter{Wyniki testów symulatora}
\label{t:wyniki}

	\section{Czas wykonania}
	\label{t:wyniki:czas_wykonania}

		Czas wykonania rozumiemy jako czas potrzebny na przetworzenie jednego pełnego kroku symulacji tkaniny. Wyrażony został w milisekundach. Jest on najważniejszym kryterium porównawczym, gdyż mówi nam, jak bardzo nasze obliczenia obciążają sprzęt, jak duży procent całości pracy silnika stanowią i w efekcie -- czy działanie symulatora jest płynne. 
		
		Wpływ na czas wykonania ma ilość przetwarzanych danych, czyli gęstość siatki tkaniny, oraz oczywiście wybrana implementacja. Pierwszą zależność przedstawiono w formie tabel oraz wykresów, osobno dla każdej metody i implementacji. Liczbę wierzchołków można w aplikacji łatwo modyfikować, zmieniając liczbę krawędzi poziomych i pionowych. Przyjęto zakres od siatki posiadającej \(10 \times 10 \) wszystkich krawędzi (100 wierzchołków) do \( 120 \times 120 \) (14400 wierzchołków), z krokiem co 10 krawędzi poziomych i pionowych.
		
		Warto wspomnieć, że do zachowania pełnej płynności obrazu na ekranie należy rysować jedną jego klatkę przynajmniej 30 razy na sekundę. Oznacza to, iż czas wykonania symulacji nie może być większy niż ok. 33 ms. Najbardziej satysfakcjonującym wynikiem byłoby osiągnięcie go niższego niż ok. 16 ms, co równe jest 60 klatkom na sekundę -- to maksymalna szybkość renderingu przy włączonej synchronizacji pionowej obrazu. Założono oczywiście, że pozostałe obliczenia związane z pracą silnika symulacji są pomijalnie krótkie.
		
		Przyjęto oznaczenia:
		
		\begin{enumerate}
			\item C -- ilość wszystkich wierzchołków,
			\item MS-GPU-A -- Model masy na sprężynie, implementacja GPU, platforma Android,
			\item PB-GPU-A -- Model oparty na pozycji, implementacja GPU, platforma Android,
			\item MS-GPU-W -- Model masy na sprężynie, implementacja GPU, platforma Windows,
			\item PB-GPU-W -- Model oparty na pozycji, implementacja GPU, platforma Windows,
			\item MS-CPU-A -- Model masy na sprężynie, implementacja CPU, platforma Android,
			\item PB-CPU-A -- Model oparty na pozycji, implementacja CPU, platforma Android,
			\item MS-CPUx4-A -- Model masy na sprężynie, implementacja CPU (4 wątki robocze), platforma Android,
			\item PB-CPUx4-A -- Model oparty na pozycji, implementacja CPU (4 wątki robocze), platforma Android.
			\newline
		\end{enumerate}
		
		\pgfplotstabletypeset{chart_6_1_a.dat}
		
		\pgfplotstabletypeset{chart_6_1_b.dat}
		
		\begin{tikzpicture}
			\begin{axis}[
			xlabel=C,
			ylabel=$t_{x}$,
			y SI prefix=milli,
			y unit=s,
			width=15cm,
			grid=major,
			legend style={at={(0.025, 0.8)}, anchor=west}
			]
			\addplot[orange, very thick] table [y=$t_{MS-GPU-A}$, x=C]{chart_6_1_a.dat};
			\addlegendentry{$t_{MS-GPU-A}$}
			\addplot[red, very thick] table [y=$t_{PB-GPU-A}$, x=C]{chart_6_1_a.dat};
			\addlegendentry{$t_{PB-GPU-A}$}
			\addplot[yellow, very thick] table [y=$t_{MS-GPU-W}$, x=C]{chart_6_1_a.dat};
			\addlegendentry{$t_{MS-GPU-W}$}
			\addplot[purple, very thick] table [y=$t_{PB-GPU-W}$, x=C]{chart_6_1_a.dat};
			\addlegendentry{$t_{PB-GPU-W}$}
			\addplot[blue, very thick] table [y=$t_{MS-CPU-A}$, x=C]{chart_6_1_b.dat};
			\addlegendentry{$t_{MS-CPU-A}$}
			\addplot[cyan, very thick] table [y=$t_{PB-CPU-A}$, x=C]{chart_6_1_b.dat};
			\addlegendentry{$t_{PB-CPU-A}$}
			\addplot[green, very thick] table [y=$t_{MS-CPUx4-A}$, x=C]{chart_6_1_b.dat};
			\addlegendentry{$t_{MS-CPUx4-A}$}
			\addplot[olive, very thick] table [y=$t_{PB-CPUx4-A}$, x=C]{chart_6_1_b.dat};
			\addlegendentry{$t_{PB-CPUx4-A}$}
			\end{axis}
		\end{tikzpicture}
		
		Wykres pokazuje dużą przewagę wydajnościową metod implementowanych na GPU. W przypadku Androida, czas obliczeń jest niemalże stały, niezależnie od ilości wierzchołków tkaniny. Drobne wahania wynikają głównie z błędu pomiaru (rzędu kilku ms). Niewielki wzrost czasu przetwarzania w końcowej fazie testów może wynikać nie tyle z samego narzutu obliczeniowego, ile z rosnącej temperatury urządzenia i związanego z tym stopniowego obniżania wydajności przez system operacyjny. 
		
		Niemożność uzyskania czasu obliczeń niższego niż ok. 12--15 ms wynika prawdopodobnie z faktu wymuszenia synchronizacji pionowej przez implementację transformacyjnego sprzężenia zwrotnego na karcie graficznej Adreno. Jak można się było spodziewać, wersję GPU na platformie PC cechuje dużo większa wydajność. W omawianym przypadku jest ona niemal 300-krotnie większa. Co ciekawe, problem z synchronizacją pionową tu nie występuje, choć czas przetwarzania także utrzymuje się na stałym poziomie. 
		
		Osobną kwestią są implementacje na CPU. Można zauważyć, iż czas przetwarzania rośnie liniowo wraz z kwadratem ilości wierzchołków i bardzo szybko osiąga wartości, które uniemożliwiają generowanie płynnego obrazu. Jedynie dla niskiej gęstości siatki uzyskano przewagę nad GPU, z racji wspomnianego wcześniej problemu. Widać także, że spadek wydajności dla implementacji z użyciem 4 wątków roboczych jest ok. dwukrotnie mniejszy niż w przypadku podejścia sekwencyjnego.
		
		W przypadku GPU nie zarejestrowano znaczących różnic czasu wykonania pomiędzy metodami symulacji, aczkolwiek na CPU model oparty na pozycji osiągał dla dużych ilości wierzchołków minimalnie lepsze wyniki niż jego rywal.
		
		%\subsection{Model masy na sprężynie -- GPU -- Android}
		%\label{t:wyniki:czas_wykonania:ms_gpu_andro}
		
		
		%\subsection{Model oparty na pozycji -- GPU -- Android}
		%\label{t:wyniki:czas_wykonania:pb_gpu_andro}
		
		
	%	\subsection{Model masy na sprężynie -- GPU -- Windows}
	%	\label{t:wyniki:czas_wykonania:ms_gpu_pc}
		
		
	%	\subsection{Model oparty na pozycji -- GPU -- Windows}
	%	\label{t:wyniki:czas_wykonania:pb_gpu_pc}
		
		
	%	\subsection{Model masy na sprężynie -- CPU -- Android}
	%	\label{t:wyniki:czas_wykonania:ms_cpu_andro}
		
		
	%	\subsection{Model oparty na pozycji -- CPU -- Android}
	%	\label{t:wyniki:czas_wykonania:pb_cpu_andro}
		
		
	%	\subsection{Model masy na sprężynie -- CPU (4 wątki) -- Android}
	%	\label{t:wyniki:czas_wykonania:ms_cpux4_andro}
		
		
	%	\subsection{Model oparty na pozycji -- CPU (4 wątki) -- Android}
	%	\label{t:wyniki:czas_wykonania:pb_cpux4_andro}
		
	
	\section{Stabilność}
	\label{t:wyniki:stabilnosc}
	
		Drugą najważniejszą cechą symulacji jest jej stabilność, rozumiana jako skłonność do wpadania siatki tkaniny w niekontrolowane drgania, co w efekcie może prowadzić do ``eksplozji'', czyli dążenia pozycji wierzchołków do nieskończoności. Nawet jeśli się tak nie stanie, ciągłe ruchy układu skutkują nierealistycznym efektem wizualnym. Zjawisko to jest więc bardzo niepożądane i często zmusza do uruchomienia symulatora od początku.
		
		Trudno określić, które dokładnie parametry mają wpływ na stabilność tkaniny. Z pewnością najważniejszym z nich jest sztywność -- większe siły sprężystości bądź większy udział ograniczników mogą prowadzić do powstawania anomalii w procesie symulacji. Dla modelu masy na sprężynie znaczenie w redukcji drgań ma także współczynnik ich tłumienia. Nie bez wpływu pozostają też takie zmienne, jak gęstość siatki, masa czy siła grawitacji.
		
		Na potrzeby testów wybrano jeden z położonych w środku tkaniny wierzchołków oraz zbadano jego drgania w stanie spoczynku, tj. średnią różnicę pomiędzy położeniem obecnym a poprzednim, w każdym kroku symulacji. Pomiarów dokonano dla różnych współczynników sztywności, a następnie przedstawiono tę zależność w postaci tabel i wykresów. Przy każdej metodzie zostały zbadane dwa przypadki, uwzględniające inne masy, siły grawitacji, współczynniki tłumienia oraz gęstości siatki. Stan spoczynku określono jako stan, w którym tkanina opadnie swobodnie z pozycji poziomej do pionowej, zawieszonej w dwóch punktach i przestanie się poruszać. Warto przypomnieć, że dla modelu opartego na pozycji parametr sztywności (\(s\)) został odpowiednio przeskalowany tak, by mieścił się w zakresie [0, 1] i niósł ze sobą podobny efekt, co jego odpowiednik w modelu masy na sprężynie. Platformą testową jest mobilna wersja aplikacji, z implementacją na GPU.
		\newline
		
		Pomiar pierwszy -- sztywność: [50, 600], krok 50; masa: 0.2 \(kg\); grawitacja: 1 \(\frac{m}{s^2}\); współczynnik tłumienia: -0.5; gęstość siatki: 625 wierzchołków.
		
		Pomiar drugi -- sztywność: [50, 600], krok 50; masa: 0.7 \(kg\); grawitacja: 2 \(\frac{m}{s^2}\); współczynnik tłumienia: -10; gęstość siatki: 6400 wierzchołków.
		\newline
		
		Drgania (\(d_{x}\)) podano w mikrometrach. Ponadto przyjęto oznaczenia: \(A-n\), gdzie \(A\) -- rodzaj zastosowanego modelu symulacji (MS -- masy na sprężynie, PB -- oparty na pozycji), \(n\) -- numer pomiaru.
		\newline
		
		\pgfplotstabletypeset{chart_6_2.dat}
		
		\begin{tikzpicture}
			\begin{axis}[
			xlabel=s,
			ylabel=$d_{x}$,
			y SI prefix=micro,
			y unit=s,
			width=14cm,
			grid=major,
			legend style={at={(0.025, 0.9)}, anchor=west},
			every axis y label/.style={
				at={(-0.12, 0.5)}, rotate=90,
				anchor=east}
			]
			\addplot[orange, very thick] table [y=$d_{MS-01}$, x=s]{chart_6_2.dat};
			\addlegendentry{$d_{MS-01}$}
			\addplot[green, very thick] table [y=$d_{PB-01}$, x=s]{chart_6_2.dat};
			\addlegendentry{$d_{PB-01}$}
			\addplot[red, very thick] table [y=$d_{MS-02}$, x=s]{chart_6_2.dat};
			\addlegendentry{$d_{MS-02}$}
			\addplot[olive, very thick] table [y=$d_{PB-02}$, x=s]{chart_6_2.dat};
			\addlegendentry{$d_{PB-02}$}
			\end{axis}
		\end{tikzpicture}
		\pagebreak
		
		% screeny (2)
		
		Główna różnica pomiędzy modelami masy na sprężynie i opartym na pozycji ukazuje się właśnie tutaj. Widać, że w pierwszym przypadku, dla pierwszej próby, rejestrujemy najniższą ze wszystkich oscylację drgań, jednak rośnie ona szybko wraz ze wzrostem parametru sztywności, dla najwyższej jego wartości doprowadzając nawet do ``wybuchu'' symulacji. Jeśli chodzi o drugie podejście, można zaobserwować duże oscylacje praktycznie niezależnie od elastyczności tkaniny, co pozwala wnioskować, iż zagęszczanie siatki także ma niebagatelny wpływ na drgania. Były one widoczne praktycznie przez cały czas symulacji, możemy zaobserwować je na Rysunku X (po lewej). Ciągle poruszające się drobne zniekształcenia bardzo negatywnie wpływają na odbiór wizualny i w jakichkolwiek zastosowaniach praktycznych byłyby nie do zaakceptowania.
		
		Testy udowodniły, iż model oparty na pozycji cechuje wyjątkowa stabilność -- oscylacje były często nieznacznie większe niż u rywala, jednakże w obu próbach utrzymywały się na stałym poziomie, niezależnie od zwiększania parametru sztywności. Drugi test ukazał jednak, że dla małej elastyczności i gęstej siatki, tkanina zaczyna wchodzić w niekontrolowane kolizje z samą sobą. Jest ona na tyle sztywna, by przy odpowiednim ułożeniu cząstek masy doprowadzić do ``zawiśnięcia samej na sobie'' i unieruchomieniu się w powietrzu, de facto ignorując siłę grawitacji. Efekt ten można zaobserwować w prawej części Rysunku X. Takie zachowanie tkaniny także jest nie do zaakceptowania w warunkach praktycznych, jednak należy zaznaczyć, iż nie dochodzi tu do ``wybuchu''. Spadek oscylacji drgań w przypadku współczynnika sztywności większego niż 450, w drugiej próbie, da się prawdopodobnie wytłumaczyć sytuacją widoczną na Rysunku X (po prawej). Duże drgania miały miejsce głównie w części siatki oznaczonej czerwonym prostokątem, a obszar środkowy, z którego pobieramy próbkę, pozostawał we względnym spoczynku.
		
	%	\subsection{Model masy na sprężynie}
	%	\label{t:wyniki:stabilnosc:ms}
		
		
	%	\subsection{Model oparty na pozycji}
	%	\label{t:wyniki:stabilnosc:pb}
		
		
	\section{Efekt wizualny}
	\label{t:wyniki:efektwiz}
		
		Ostatnim kryterium oceny to tzw. ``efekt wizualny''. Przyjęta nazwa oznacza po prostu stopień, w jakim zachowanie i wygląd symulowanej tkaniny odzwierciedla rzeczywistość. Wyznacznik ten jest całkowicie subiektywny, jednak na pewno można zauważyć wprost proporcjonalną zależność pomiędzy jakością a gęstością siatki. Mała liczba wierzchołków fizycznie nie pozwala na wygenerowanie realistycznych zmarszczek ani zagięć, tak charakterystycznych elementów animacji tkanin. Dla każdego modelu symulacji zaprezentowane zostaną zrzuty ekranu, prezentujące ``efekt wizualny'' z różnymi liczbami krawędzi pionowych oraz poziomych. Zakres przyjęto od \(10 \times 10\) do \(120 \times 120\) wszystkich krawędzi siatki. Platformą testową jest mobilna wersja aplikacji, z implementacją na GPU.
		
		% kolizje z boksem!!!!!!!
			
	%	\subsection{Model masy na sprężynie}
	%	\label{t:wyniki:efektwiz:ms}
			
			
	%	\subsection{Model oparty na pozycji}
	%	\label{t:wyniki:efektwiz:pb}