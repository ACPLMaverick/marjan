\chapter{Wykorzystane technologie}
\label{t:technologie}


	\section{Analiza możliwości urządzeń mobilnych}
	\label{t:technologie:mobilne}
	
		\subsection{Sensowność wykorzystania urządzeń mobilnych w symulacji tkanin}
		\label{t:technologie:mobilne:dlaczego}
	
		\subsection{Konfiguracja sprzętowa urządzeń mobilnych}
		\label{t:technologie:mobilne:konfiguracja}
		
		\subsection{Porównanie wydajności i możliwości przykładowego komputera PC i smartfona}
		\label{t:technologie:mobilne:porownanie}
		
		\subsection{Unikalne możliwości platform mobilnych w interakcji użytkownika z tkaniną}
		\label{t:technologie:mobilne:interakcja}

	
	\section{Przedstawienie wybranych technologii i narzędzi}
	\label{t:technologie:narzedzia}
	
		\subsection{OpenGL 4.0 i OpenGL ES 3.0}
		\label{t:technologie:narzedzia:ogl}
		
		\subsection{GLSL}
		\label{t:technologie:narzedzia:glsl}
		
		\subsection{CUDA}
		\label{t:technologie:narzedzia:cuda}
		
		\subsection{Android NDK}
		\label{t:technologie:narzedzia:ndk}
		
		\subsection{Visual Studio 2015 Community + Cross-platform Development Kit}
		\label{t:technologie:narzedzia:vs}