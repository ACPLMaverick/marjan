\chapter{Wykorzystane technologie}
\label{t:technologie}


	\section{Analiza możliwości urządzeń mobilnych}
	\label{t:technologie:mobilne}
	
		\subsection{Sensowność wykorzystania urządzeń mobilnych w symulacji tkanin}
		\label{t:technologie:mobilne:dlaczego}
		
		% niższa wydajność ALE unikalne możliwości interakcji i wszechobecna dostępność, wykorzystanie w grach 3D (rozwój tychże) i aplikacjach branży tekstylnej (włókienniczej) oraz odzieżowej, niższa wydajność wiążąca się z niższą jakością, wykorzystanie GPU by ją podnieść
		
		W związku z faktem, iż niniejsza praca zajmuje się symulacją tkaniny na urządzeniach mobilnych, naturalnie nasuwa się pytanie o~sens realizacji tego typu obliczeń z~użyciem smartfonu bądź tabletu. Prawdą jest, że mamy do czynienia ze sporym zapotrzebowaniem na moc obliczeniową, zwiększającym się proporcjonalnie do oczekiwanej dokładności rozwiązania, a~konkretnie - do gęstości siatki tkaniny. Z~drugiej strony, urządzenia mobilne, w~przeciwieństwie do platformy PC, szczytowymi osiągnięciami technologii pod względem wydajności nigdy nie były -- raczej starano się tu osiągnąć kompromis między sensowną mocą obliczeniową a~niskim zużyciem energii. Jednakże smartfony i~inne tego typu urządzenia cechują się dostępnością dla użytkownika praktycznie zawsze, co nie do końca może być powiedziane o~PC, oraz unikalnymi metodami interakcji, takimi jak ekran dotykowy, bądź akcelerometr.
		
		Mimo oczywistych wad, także i~na platformach mobilnych symulacja tkanin znajduje zastosowanie. Jako pierwszy i~najważniejszy przykład należy wskazać rynek gier i~wizualizacji 3D. Dawno minęły już czasy, gdy najpopularniejszą grą na telefonach komórkowych był kultowy, dwuwymiarowy ``\emph{Snake}''. Obecnie spora część rynku koncentruje się na złożonych grach trójwymiarowych, z~coraz ładniejszą grafiką, na potrzeby których tworzy się coraz bardziej zaawansowane silniki graficzne i~korzysta z najnowszych technologii. Podobnie, jak na platformie PC, także i~tutaj możliwe, a~nawet pożądane jest użycie symulacji tkanin do m.in. realistycznej animacji elementów stroju bohaterów, flag powiewających na wietrze oraz innych przedmiotów tekstylnych.
		
		Obecnie wielu producentów i~sprzedawców z różnych branż decyduje się na stworzenie i~wypuszczenie na rynek własnej, wyspecjalizowanej aplikacji dla urządzeń mobilnych, pozwalającej w prosty, przyjazny sposób przeglądać oferty, oglądać towary i~dokonywać zakupów. Zdecydowanie zwiększa to przychody danej firmy. Branżą, która mogłaby skorzystać na zastosowaniu symulacji tkanin w~swoich aplikacjach jest oczywiście branża włókiennicza i~odzieżowa. Przykładem może być chociażby stworzenie ``wirtualnej przymierzalni'' \cite{tryon}, przy pomocy której klient byłby w stanie ``ubrać się'' w każdy wybrany element odzieży. Aplikacja pozwoliłaby mu chociażby na obejrzenie go ze wszystkich stron, sprawdzenie elastyczności i~zachowania się go w różnych pozach. A to wszystko na ekranie tabletu, dostępne w~każdym możliwym miejscu. 
		
		Oczywistym jest, że niższa wydajność urządzeń mobilnych wiąże się z~niższą jakością symulacji. Warto jednak pamiętać, iż wyświetlacze urządzeń mobilnych z~reguły są mniejsze od ``pecetowych'' monitorów. Co za tym idzie -- możemy zastosować siatkę tkaniny o~mniejszej gęstości i~dokonywać mniej dokładnych obliczeń np. detekcji kolizji bez dużego spadku jakości wizualnej. Ten fakt, oraz omówiona wcześniej mnogość zastosowań sprawiają, że symulacja tkanin na urządzeniach mobilnych zdaje się jak najbardziej mieć sens. W rozdziałach \ref{t:wyniki} i~\ref{t:wnioski} przekonamy się, w~jakim stopniu.
	
		\subsection{Konfiguracja sprzętowa urządzeń mobilnych i porównanie z konfiguracją PC}
		\label{t:technologie:mobilne:konfiguracja}
		
		% architektura CPU, szybkości CPU, ilość core'ów, architektura GPU, ilosć SM, API GPU i GPGPU, ilość pamięci op., 
		
		Jak już wyżej wspomniano, wydajność konfiguracji sprzętowa urządzeń mobilnych to drobny ułamek wydajności komputerów klasy PC. Warto dokładniej zwrócić uwagę, jaka jest między tymi platformami różnica i~z~jakimi ograniczeniami się spotkamy, tworząc symulację tkanin na tej pierwszej. Porównania dokonamy na przykładzie urządzenia testowego -- smartfona LG Nexus 4 E960. Dane techniczne zaczerpniemy z~\cite{specs}, \cite{specs_adreno}, \cite{specs_gtx750} i \cite{specs_gtxtitan}. 
		
		Urządzenie oparto o mikrokontroler APQ8064 Snapdragon S4 Pro. Serce układu to czterordzeniowy procesor Krait o~taktowaniu 1.5 GHz i~architekturze ARMv7-A. Szybkość zegara jest przeszło dwa razy mniejsza niż w~przeciętnym odpowiedniku PC. Można stąd wnioskować, że wydajność mamy dwukrotnie mniejszą, jednak diabeł tkwi w szczegółach. Dzisiejsze procesory architektury x86 dysponują szerokim wachlarzem specjalnych instrukcji, takich jak SSE czy AVX, bardzo przyspieszających operacje wektorowe, typu SIMD. Jedynym ich odpowiednikiem w omawianym układzie są instrukcje NEON, dużo mniej wydajne. A~zatem, zgodnie z~\cite{versus}, procesor Krait cechuje ponad dziesięciokrotnie mniejsza wydajność niż jego przykładowego odpowiednika z~komputera klasy PC - Intel Core i7-4770.
		
		Układ Snapdragon jest także wyposażony w dedykowane GPU, specjalnie na potrzeby renderingu grafiki 2D i~3D. To Adreno 320, cechujące się taktowaniem zegara 400 MHz i~w sumie 64 procesorami strumieniowymi. Karta graficzna osiąga wydajność ok. 60 GFLOPS\footnote{\emph{Floating-point Operations Per Second} -- ilość operacji na liczbach zmiennoprzecinkowych w~czasie 1 sekundy.}. Dla porównania weźmy średniej klasy GPU komputerów stacjonarnych sprzed paru lat, GeForce GTX 750. Jego zegar to 1020 MHz, ma ono 512 SP, a~wydajnościowo plasuje się trochę ponad 1 TFLOPS. Jedno z~najpotężniejszych GPU obecnie, GeForce GTX Titan, cechuje z kolei 3072 procesorów strumieniowych i~ok. 6 TFLOPS. Pod uwagę bierzemy oczywiście obliczenia na liczbach zmiennoprzecinkowych pojedynczej precyzji. Widzimy więc, że mobilne GPU wydajnościowo stanowią zaledwie ułamek ich ``pełnowymiarowych'' odpowiedników.
		
		Ważną kwestią jest też dostępność i~obsługa odpowiednich technologii, a~w~szczególności API graficznych i~GPGPU. Tutaj na szczęście sytuacja ma się dużo lepiej. Adreno 320 wspiera zaawansowane już całkiem OpenGL ES 3.0 oraz Direct3D 11.1. Patrząc na sprawę pod kątem symulacji tkanin, brakuje tylko obsługi OpenGL ES 3.2, który wprowadził bardzo przydatne w omawianym problemie bufory teksturowe. Mimo to jednak API graficzne spełniają wymagania niniejszej pracy. Oczywiście z~ogólnego punktu widzenia, brakuje wielu nowych rozwiązań, wprowadzonych w najnowszych ``dużych'' wersjach OpenGL. Do obliczeń GPGPU wspierane są technologie OpenCL 1.1, będący open-source'owym odpowiednikiem CUDA, oraz RenderScript. W~tej pracy jednak okazało się niemożliwe użycie żadnego z nich. Obsługa OpenCL została wycofana na urządzeniach Google, w~tym na Nexusie 4, z~powodów marketingowych, a~konkretnie, z~powodu promowania drugiej wymienionej technologii. Tej z~kolei nie można było użyć z~racji tego, iż API nie pozwala jawnie wybrać i~ustalić, czy obliczenia będą dokonywane na GPU, czy na CPU.
		
		\subsection{Unikalne możliwości platform mobilnych w interakcji użytkownika z tkaniną}
		\label{t:technologie:mobilne:interakcja}
		
		% ekran dotykowy, akcelerometr
		
		Urządzenia mobilne cechuje jedna ważna przewaga nad komputerami PC -- charakterystyczne tylko dla nich interfejs. Sztandarowym przykładem jest oczywiście ekran dotykowy. W~rozważanej kwestii, z~jego pomocą użytkownik może ruchami palca po ekranie przemieszczać i~rozciągać tkaninę. Pozwala to na nadanie jej ruchu w~odpowiednią stronę, sprawdzanie elastyczności oraz dokładne, precyzyjne ustawienie w~wirtualnym świecie poprzez sprowokowanie kolizji z~obiektami otoczenia. Przydatne może być to szczególnie w omawianym wcześniej przypadku wykorzystania symulacji do aplikacji typu ``przymierzalnia''. Klient jest w stanie realistycznie założyć oglądany element ubioru na swojego awatara i~dokładnie go dopasować. 
		
		Kolejnym charakterystycznym dla platformy mobilnej urządzeniem wejścia jest akcelerometr. Najważniejszy przykład jego wykorzystania to możliwość zmiany kierunku siły grawitacji działającej na symulowany układ, poprzez obracanie telefonu w~odpowiednim kierunku. Użytkownik będzie miał wrażenie obracania układem, a~siła grawitacji pozostanie stała i~skierowana w dół względem niego. Stwarza to kolejne możliwości zmiany położenia tkaniny, dokładnego jej układania i~odwzorowania realizmu świata wirtualnego.

	
	\section{Przedstawienie wybranych technologii i narzędzi}
	\label{t:technologie:narzedzia}
	
		\subsection{Android NDK}
		\label{t:technologie:narzedzia:ndk}
		
		% jedyne sensowne rozwiązanie jesli sie chce wycisnąć 100% wydajności z aplikacji, odradzany przez dokumentację, % C++ !	
		
		Wiodącym językiem programowania na platformie Android jest język Java. Wykorzystujące go Android SDK posiada bardzo szeroką dokumentację i~pomoc techniczną. Można na jego temat znaleźć wiele publikacji, a~w~przypadku, gdy mamy z~nim jakieś trudności, w~Internecie na pewno znajdziemy rozwiązanie. Istnieje jednak jeden duży problem -- Java nie jest językiem natywnym i~uruchamia się na maszynie wirtualnej. Przez to nie nadaje się do aplikacji, w~których kluczowa jest wydajność, takich jak na przykład symulacja fizyczna tkanin.
		
		Jest to powód, dla którego cały projekt został napisany w~języku C++. Umożliwiło to wykorzystanie Android NDK\footnote{Native Development Kit.}, czyli zbioru większości funkcjonalności Androida, dostępnych z~poziomu C bądź C++. Kod ten zostaje połączony z~inicjalizacyjnym kodem Javy przy pomocy technologii JNI\footnote{Java Native Interface}, będącej swoistym pomostem pomiędzy dwoma platformami. Cykl życia aplikacji kontrolowany z~poziomu C++ jest niemalże identyczny, jak w przypadku języka Java, opiera się na zdarzeniach wywoływanych przez system operacyjny. W~ten sposób zarządzamy m.in. inicjalizacją i~zwalnianiem pamięci oraz sygnałami przychodzącymi z~urządzeń wejściowych, takich jak ekran dotykowy czy akcelerometr. Odwrotnie jak w~przypadku PC, gdzie sami ręcznie ``wyciągamy'' informację o~tym, czy dany klawisz na klawiaturze jest wciśnięty. Za pomocą plików konfiguracyjnych i~tzw. \emph{Android Manifest} mamy kontrolę nad wszelkimi opcjami konfiguracyjnymi dotyczącymi aplikacji. Możemy tam ustalić np. jej nazwę, widoczność elementów interfejsu systemu oraz to, czy aplikacja sama potrafi obsłużyć zmianę orientacji ekranu, czy należy ją wtedy zresetować.
		
		Jedna z~implementacji symulacji zakłada przetwarzanie jej wielowątkowo na CPU. Jako, że Android wywodzi się z~rodziny UNIX'ów, korzystając z~C++ mamy do dyspozycji bibliotekę \emph{pthread}. Umożliwia ona łatwe i~przejrzyste uruchamianie nowych wątków, zarządzanie nimi oraz ich synchronizację.
	
		\subsection{OpenGL}
		\label{t:technologie:narzedzia:ogl}
		
		% różnice między nimi i brak paru ważnych rzeczy!
		
		Stworzenie samej symulacji tkanin nie ma sensu, jeśli nie jesteśmy w stanie pokazać jej działania. Konieczne więc było stworzenie silnika wizualizującego wygląd i~zachowanie naszej tkaniny. OpenGL to darmowe API graficzne, czyli zestaw bibliotek służących do komunikacji z GPU i~w efekcie -- rysowania grafiki 2D i~3D. Najważniejszą cechą tej technologii jest jej dostępność na praktycznie wszystkich platformach. Właśnie z~tego powodu została ona użyta w niniejszej pracy, jako że jest obsługiwana zarówno na smartfonach z systemem Android, jak i~na komputerach PC z Windows. W pierwszym przypadku mamy do czynienia z wersją OpenGL ES 3.0, będącą okrojoną i~przystosowaną do użytku na platformach mobilnych oraz innych systemach wbudowanych. Na PC użyliśmy standardowego OpenGL 3.3. Nie są to najnowsze edycje OpenGL, jednak oferują już programowalny potok renderingu i~systemy wspierające GPGPU, takie jak transformacyjne sprzężenie zwrotne. Uznaliśmy je więc za wystarczające do zrealizowania niniejszej pracy.
		
		Jak możemy przeczytać w~\cite{oglspec}, OpenGL wymaga także API pomocniczego, zarządzającego tworzeniem okien, do których można rysować grafikę, przydzielaniem kontekstów graficznych, wymaganych by cokolwiek dało się w~tej materii zadziałać, oraz innymi zasobami. W przypadku programu na platformie Android, użyte zostało tu EGL -- specjalne API do systemów mobilnych i~wbudowanych, dzieła twórców OpenGL. W wersji ``pecetowej'' wykorzystaliśmy darmowe API GLFW oraz GLEW.
		
		Ważną kwestią w tworzeniu zarówno symulacji fizycznych, jak i~samego silnika graficznego jest odpowiednia baza matematyczna. W niniejszej pracy użyta została darmowa biblioteka GLM, zalecana jako nieodłączny element programowania w OpenGL. Zapewnia ona dostęp do wielu wygodnych i~przydatnych funkcji oraz struktur matematycznych, ułatwiających szczególnie obliczenia na wektorach oraz macierzach, kluczowe w grafice 3D. GLM w swoim założeniu ma jak najbardziej przypominać składnię i~semantykę związanego z OpenGL języka programów cieniujących -- GLSL, o~którym więcej w Podrozdziale \ref{t:technologie:narzedzia:ogl:glsl}.
		
		Do wczytywania i przetwarzania tekstur wykorzystano także darmową bibliotekę SOIL2. Tekstury są w naszym programie niezbędne, z racji tego, iż chcemy wyświetlać elementy interfejsu użytkownika -- przyciski i tekst. SOIL2 pozwala na łatwe i szybkie załadowanie tekstury w jednym z wielu obsługiwanych formatów, a następnie utworzenie odpowiedniego obiektu OpenGL'owego.
		
			\subsubsection{OpenGL ES 3.0 kontra OpenGL 3.3}
			\label{t:technologie:narzedzia:ogl:vs}
			
			Według \cite{oglspec}, każda wersja OpenGL ES jest podzbiorem funkcji pewnej ``dużej'' wersji OpenGL. W przypadku użytego tutaj Nexusa 4, obsługiwana jest edycja ES 3.0, a~jej odpowiednikiem na większych platformach -- wersja 3.3. Właśnie dlatego ta ostatnia posłużyła do stworzenia ``pecetowego'' modelu symulacji tkanin. Dzięki temu można było zapewnić jak największą kompatybilność i~łatwość przeniesienia z~jednej platformy na drugą.
			
			Filozofią przyświecającą deweloperom przy tworzeniu edycji OpenGL ES jest przede wszystkim to, by każda czynność, którą możemy zrobić przy pomocy API, była osiągalna tylko w jeden, konkretny sposób. Idąc tym tropem, ujednolicono wiele funkcji i~usunięto te, które się dublowały, bądź w dalszej perspektywie dawały identyczne efekty. Dzięki temu API stało się bardziej przejrzyste oraz łatwiej się z~niego korzysta. Warto zaznaczyć, że użyteczność nie zmniejszyła się względem wersji API 3.3, a po prostu ``zrobiono porządek''.
			
			\subsubsection{GLSL}
			\label{t:technologie:narzedzia:ogl:glsl}
			
			% zalety - zintegrowany z API graficznym, ogólnodostępny, szybki; wady - konieczność dostosowania GPGPU do API graficznego, mało funkcjonalności wspierających GPGPU na OpenGL ES
			
			GLSL jest językiem specjalistycznym dla API OpenGL, przy pomocy którego piszemy programy cieniujące, czyli tzw. \emph{shadery}. W niniejszej pracy wykorzystany został do stworzenia podstawowego, prostego modelu cieniowania opartego o~wzory Phonga-Blinna. Z racji opisanego w~Podrozdziale \ref{t:technologie:mobilne:konfiguracja} braku innych sensownych technologii na wybranej platformie sprzętowej, użyliśmy go także do napisania obliczeń symulacji tkanin na GPU, tj. wyliczenia przesunięć wierzchołków, rozwiązania kolizji zewnętrznych i~wewnętrznych oraz przeliczenia wektorów normalnych.
			
			Niewątpliwą zaletą GLSL jest jego integracja z API graficznym i~brak konieczności integracji dodatkowych bibliotek. Chcąc prowadzić specjalistyczne obliczenia dla tkaniny, możemy korzystać z tych samych buforów wierzchołków, których używamy do jej rysowania. Bardzo łatwo uniknąć jest niepotrzebnego kopiowania danych. Co za tym idzie -- GLSL jest bardzo optymalnym pod względem wydajności rozwiązaniem. Wart wspomnienia jest także fakt jego dostępności, możemy, właściwie musimy, z~niego skorzystać wszędzie tam, gdzie możliwe jest uruchomienie biblioteki OpenGL. 
			
			GLSL ma jednak kilka wad, które dają o~sobie znać w momencie, gdy chcemy wykorzystać go do obliczeń GPGPU. Obojętnie, jaki problem trzeba rozwiązać, musimy dostosować dane oraz algorytmy do struktur danych potoku renderingu i~do funkcji API graficznego. W~rozważanym przypadku nie jest to jednak duży problem. Gorszą bolączką jest tak naprawdę niewielkie wsparcie dla GPGPU w OpenGL ES 3.0, omówione zostanie ono w~następnym podrozdziale. Warto wspomnieć, że najnowsze wersje API mają już pełną obsługę obliczeń GPGPU, czego przejawem jest chociażby obecność \emph{Compute Shaderów}, czyli programów ogólnego przeznaczenia uruchamianych na GPU.
			
			\subsubsection{Bufory teksturowe i bufory jednorodne}
			\label{t:technologie:narzedzia:bufory}
			
			Dużym minusem urządzenia testowego jest brak obsługi OpenGL ES 3.2, w którym dodana została funkcjonalność zwana buforami teksturowymi. Zostały one wykorzystane do zaimplementowania prostej symulacji tkanin w~\cite{receptury}, co dowodzi ich kluczowej przydatności. Wykonując obliczenia na każdym wierzchołku, musimy mieć dostęp do pozycji wierzchołków sąsiednich, aby obliczyć odległości między nimi a~tym aktualnym. Prowadzi to do konieczności posiadania dostępu do bufora pozycji obecnych oraz~poprzednich \uline{wszystkich} wierzchołków przez każdy uruchomiony na GPU kernel, czy też program cieniujący. 
			
			Można tego dokonać idealnie wykorzystując bufory teksturowe, będące \emph{de facto} jednowymiarową teksturą, pod którą da się ``podpiąć'' jakikolwiek istniejący na karcie graficznej bufor danych. Jeden bufor teksturowy może pomieścić co najmniej 64 KB liczb, zależnie od pamięci konkretnej GPU \cite{opengl_wiki}. Zapewnia też bardzo szybki losowy odczyt danych. 
			
			Zamiast tego zmuszeni byliśmy użyć innej techniki, a mianowicie tzw. \emph{Uniform Buffer Objects}, co można przetłumaczyć jako bufory jednorodne. Wykorzystywane są one do łączenia danych trafiających do programów cieniujących w jeden ciągły blok i~dają możliwość szybkiego wykorzystania tego bloku w~wielu programach \cite{opengl_wiki}. Ich zastosowanie jest, jak widać, trochę inne niż to, o~które nam chodzi, jednak w~efekcie spełniają podobną rolę, jak bufory teksturowe. Musimy się jednak liczyć z~faktem, że maksymalna ich wielkość to tylko 16 KB danych, co ogranicza nam maksymalną możliwą gęstość siatki tkaniny. Odczyt jest tu także dużo wolniejszy, niż w~pierwszym przypadku.
			
			\subsubsection{Transformacyjne sprzężenie zwrotne}
			\label{t:technologie:narzedzia:transformfeedback}
			
			O mechanizmie OpenGL zwanym \emph{Transform Feedback}, czyli transformacyjnym sprzężeniem zwrotnym, wspomnieliśmy już w~Podrozdziale \ref{t:teoria:gpu:architektura}. Znajdujący się tam Rysunek \ref{pic_2_9} przedstawia m.in. ogólną zasadę działania tej technologii. Jak wiemy, dane w potoku renderingu koniec końców trafiają do bufora ramki, bądź tylnego bufora i~są po drodze przetwarzane przez kilka różnych etapów. Aby w~ogóle móc dokonywać obliczeń GPGPU korzystając z~API graficznego, musimy mieć możliwość odczytu danych wyjściowych. Następnie albo uznamy, że są to nasze wyniki obliczeń, albo podamy je znowu na wejście potoku do ponownego przetworzenia. 
			
			Taką możliwość daje nam właśnie transformacyjne sprzężenie zwrotne. Dane trafiają do shadera wierzchołków, gdzie program cieniujący wykonuje na nich ciąg operacji. Następnie zamiast zostać podane do rasteryzera, zapisywane są do ustalonego wcześniej, specjalnego bufora, oznaczonego jako bufor sprzężenia zwrotnego. Po zakończeniu działania \emph{Transform Feedbacku} mamy dostęp do niego, tak samo, jak do każdego innego bufora w~OpenGL. Opisywana symulacja tkanin jest idealnym przykładem zastosowania tego rozwiązania, jako, że takie same obliczenia wykonujemy w każdym kroku, a~następnie po prostu zamieniamy miejscami bufory wejściowe i~wyjściowe. 
		
		%\subsection{CUDA}		% nie, bo projekt PC-towy trzeba przerobić idealnie tak jak jest androidowy zrobiony
		%\label{t:technologie:narzedzia:cuda}
		
		\subsection{Visual Studio 2015 Community + Cross-platform Development Kit}
		\label{t:technologie:narzedzia:vs}
		
		% nowość, pełna intergracja z VS, zero Javy (też i wada), alternatywa - płatny Xamarin
		
		Jeśli chodzi o~środowisko programistyczne, wykorzystano najnowszą edycję Microsoft Visual Studio w~darmowej wersji Community. Wybór został dokonany ze względu na jego wszechstronność i~możliwości, a~przede wszystkim dobrą znajomość. Nowością w wydaniu 2015 aplikacji jest pakiet Cross-platform Development Kit. Umożliwia on budowanie programów na platformy mobilne, takie jak Android lub iOS. Dotychczas było to wspierane tylko w~Eclipse, bądź Android Studio, będącym jego specjalistyczną edycją dla programowania na Androidzie. Nowy pakiet Microsoftu pozwala na tworzenie aplikacji w natywnym języku C++, z użyciem opisanych w~Podrozdziale \ref{t:technologie:narzedzia:ndk} bibliotek. Zapewnia odgórnie także integrację z systemem operacyjnym, co jest dużym ułatwieniem dla początkujących programistów, którzy nie muszą się zajmować zawiłościami inicjalizacji natywnego kodu. 
		
		Wiąże się z~tym jednak poważne ograniczenie -- oznacza to bowiem, że nie mamy dostępu do części aplikacji w Javie, tworzonej tu automatycznie. Język ten jest w ogóle nieobsługiwany przez Visual Studio, ale teoretycznie wszystkie potrzebne zasoby, np. referencję do menedżera assetów\footnote{Zasobów dodatkowych, potrzebnych do działania aplikacji, np. tekstur, dźwięków, siatek geometrycznych, itp.}, dostajemy bezpośrednio przy starcie programu. Jeżeli jakieś rozwiązanie wymaga bardziej zaawansowanej komunikacji między Javą a C++ -- nie możemy z~niego skorzystać.
		
		Microsoft pomyślał też i~o~tym, wprowadzając pakiet Xamarin. Daje on możliwość pisania ``wysokopoziomowej'' części kodu przeznaczonego na platformy mobilne w~C\#. Jednakże jest to już opcja płatna, a~edycja darmowa wprowadza duże ograniczenia odnośnie budowanych aplikacji, ograniczając chociażby rozmiar pliku wykonywalnego do 128 kB i~uniemożliwiając łączenie go z bibliotekami tworzonymi w~innych językach, jak C/C++ lub Java.
		