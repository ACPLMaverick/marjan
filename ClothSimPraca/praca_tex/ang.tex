\chapter*{Abstract}

% what exactly is cloth simulation
% why was it created and where is it used
% usage of cloth simulation on mobile devices
% main purposes of this paper
% what is to be done
% every chapter's summary

The realistic simulation of cloths is nowadays a~key to produce good-quality, authentic graphical visualizations of various fabrics, such as characters' garment elements, flags or curtains. This can be computationally expensive, more and more as number of particles, which fabric is divided into, increases. The solution to this matter was to use GPU -- \emph{Graphic Processing Unit} and perform all calculations on this device. On PC platform, this technique proved to be much faster than the standard CPU approach.

The main purpose of this work is to check whether this solution could also be introduced on the mobile devices. Most of them nowadays also have their own specialized GPU chips, but will they prove to be computationally faster than mobile CPUs? Is it possible and worth one's while to create visually appealing and efficient cloth simulation here? And how big is the difference between PC and mobile platform in GPGPU performance? This paper answers these questions.

A~test application was created, one of its main purposes being the visualisation of two selected simulation methods -- mass-spring and position-based model. It was equally important to show cloth's collisions between other objects in scene and itself. The user is allowed to set various parameters that influence the simulation, such as the aforementioned method type, mesh density and dimensions or elasticity coefficient. He can also impact the movement of the cloth, swiping his finger along the device's touch screen, which is something unique to the mobile platform. To fully measure every important factor of the simulation, its three implementations were created -- one using GPU for computing and the other two using GPU, in sequential and multi-threaded approach. To have a~comparison between mobile and PC platform, a~PC version of the application was created, both similar and sharing as much code with each other as possible.

In chapter 2 of this paper there are described theoretical basis of incorporated simulation methods and collision resolving. General Purpose GPU Computing is also mentioned, along with GPU framework and a~comparison between it and a~CPU is made, in the matter of architecture and performance. Chapter 3 analyses abilities of mobile devices, also mentioning their unique UI capabilities after comparing test device -- LG Nexus 4 -- to an~example PC. All used APIs, libraries and most important functions are also described. Among them are OpenGL ES 3.0, OpenGL 3.3 and Android NDK, as test application is completely written in C++. Chapter 4 shows architecture of~its engine, magnifying characteristics of~the most important components. Algorithms of~the engine's operation are also given, including such actions as~updating scene's entities, communication with Android OS, rendering or UI. In Chapter 5 attention is turned to the cloth simulator only, describing its general work flow, divided into three stages -- particle movement computation, collision solving and recalculation of normal vectors. Every stage is then thoroughly described, with their most important fragments's code given. Finally, in chapters 6 and 7, results of test application's work are shown, in the matter of both simulation models' computation time, stability and visual appeal, considering all three implementations and two tested platforms. In the end it is proved that the cloth simulation can be implemented on mobile devices and the average one's GPU can perform very well, producing smooth animation of fabric's dense mesh, but not without a few important limitations. These include less useful API functions and shorter work time on battery as a result of intensive computations and tendency to overheating.