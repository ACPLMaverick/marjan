\documentclass[12pt, oneside, a4paper]{mwbk}
\usepackage[polish]{babel}
\usepackage[utf8]{inputenc}
\usepackage[OT4]{fontenc}

\usepackage{graphicx}
\usepackage{verbatim}

\usepackage[hidelinks]{hyperref}

\usepackage{titletoc}
\usepackage{etoolbox}

\usepackage{enumitem}
\usepackage[noend]{algorithmic}
\usepackage[polish,ruled]{algorithm2e}

\usepackage{color}
\usepackage[normalem]{ulem}

\usepackage{rotating}
\usepackage{float}
\usepackage{textpos}

\usepackage{longtable}
\usepackage{colortbl}
\usepackage{booktabs}
\usepackage{pgfplotstable}
\usepgfplotslibrary{units}

\definecolor{dkgreen}{rgb}{0, 0.4, 0}


\linespread{1,3}
\oddsidemargin = 10pt
\textwidth = 470pt

\hyphenpenalty=1000
\tolerance=500

\setcounter{secnumdepth}{4}
\usepackage{listings}

\newcommand{\myfigure}[4]{
	\begin{figure}[H]
		\centering
		\scalebox{#3}
		{
			\includegraphics{#2}
		}
		\caption[#1]{#1}
		\label{#4}
	\end{figure}
}
\newcommand{\myownfigure}[4]{
	\begin{figure}[H]
		\centering
		\scalebox{#3}
		{
			\includegraphics{#2}
		}
		\caption[#1]{#1 Źródło: opracowanie własne.}
		\label{#4}
	\end{figure}
}
\newcommand{\myownextfigure}[4]{
	\begin{figure}[H]
		\centering
		\scalebox{#3}
		{
			\includegraphics{#2}
		}
		\caption[#1]{#1 Źródło: opracowanie własne w oparciu o źródło zewnętrzne.\footnotemark}
		\label{#4}
	\end{figure}
}
\newcommand{\myextfigure}[4]{
	\begin{figure}[H]
		\centering
		\scalebox{#3}
		{
			\includegraphics{#2}
		}
		\caption[#1]{#1\footnotemark}
		\label{#4}
	\end{figure}
}
\newcommand{\myextcitefigure}[5]{
	\begin{figure}[H]
		\centering
		\scalebox{#4}
		{
			\includegraphics{#3}
		}
		\caption[#1]{#1 #2}
		\label{#5}
	\end{figure}
}


\pgfplotstableset{
	begin table=\begin{longtable}, % -------- CF
		end table=\end{longtable},
	every head row/.style={
		%before row=\caption{X-Powered-By header}\\\toprule, after row=\bottomrule \endhead,% --------- CF
		% as in the previous example, this patches the first row:
		before row={\hline},
		after row=\hline,
	},
	every last row/.style={% ------------ CF
		after row=\hline,
	},
	every even row/.style={
		before row={\rowcolor[gray]{0.92}}},
}
	
\lstset{frame=tb,
	aboveskip=3mm,
	belowskip=3mm,
	showstringspaces=false,
	columns=flexible,
	basicstyle={\small\ttfamily},
	numbers=none,
	numberstyle=\tiny\color{gray},
	keywordstyle=\color{blue},
	commentstyle=\color{dkgreen},
	stringstyle=\color{mauve},
	breaklines=true,
	breakatwhitespace=true,
	tabsize=3}
	
\newcommand{\beginnumbered}{\begin{enumerate}[label*=\arabic*.]}

\lstdefinelanguage{GLSL}
{
	sensitive=true,
	morekeywords=[1]{
		attribute, const, uniform, varying,
		layout, centroid, flat, smooth,
		noperspective, break, continue, do,
		for, while, switch, case, default, if,
		else, in, out, inout, float, int, void,
		bool, true, false, invariant, discard,
		return, mat2, mat3, mat4, mat2x2, mat2x3,
		mat2x4, mat3x2, mat3x3, mat3x4, mat4x2,
		mat4x3, mat4x4, vec2, vec3, vec4, ivec2,
		ivec3, ivec4, bvec2, bvec3, bvec4, uint,
		uvec2, uvec3, uvec4, lowp, mediump, highp,
		precision, sampler1D, sampler2D, sampler3D,
		samplerCube, sampler1DShadow,
		sampler2DShadow, samplerCubeShadow,
		sampler1DArray, sampler2DArray,
		sampler1DArrayShadow, sampler2DArrayShadow,
		isampler1D, isampler2D, isampler3D,
		isamplerCube, isampler1DArray,
		isampler2DArray, usampler1D, usampler2D,
		usampler3D, usamplerCube, usampler1DArray,
		usampler2DArray, sampler2DRect,
		sampler2DRectShadow, isampler2DRect,
		usampler2DRect, samplerBuffer,
		isamplerBuffer, usamplerBuffer, sampler2DMS,
		isampler2DMS, usampler2DMS,
		sampler2DMSArray, isampler2DMSArray,
		usampler2DMSArray, struct},
	morekeywords=[2]{
		radians,degrees,sin,cos,tan,asin,acos,atan,
		atan,sinh,cosh,tanh,asinh,acosh,atanh,pow,
		exp,log,exp2,log2,sqrt,inversesqrt,abs,sign,
		floor,trunc,round,roundEven,ceil,fract,mod,modf,
		min,max,clamp,mix,step,smoothstep,isnan,isinf,
		floatBitsToInt,floatBitsToUint,intBitsToFloat,
		uintBitsToFloat,length,distance,dot,cross,
		normalize,faceforward,reflect,refract,
		matrixCompMult,outerProduct,transpose,
		determinant,inverse,lessThan,lessThanEqual,
		greaterThan,greaterThanEqual,equal,notEqual,
		any,all,not,textureSize,texture,textureProj,
		textureLod,textureOffset,texelFetch,
		texelFetchOffset,textureProjOffset,
		textureLodOffset,textureProjLod,
		textureProjLodOffset,textureGrad,
		textureGradOffset,textureProjGrad,
		textureProjGradOffset,texture1D,texture1DProj,
		texture1DProjLod,texture2D,texture2DProj,
		texture2DLod,texture2DProjLod,texture3D,
		texture3DProj,texture3DLod,texture3DProjLod,
		textureCube,textureCubeLod,shadow1D,shadow2D,
		shadow1DProj,shadow2DProj,shadow1DLod,
		shadow2DLod,shadow1DProjLod,shadow2DProjLod,
		dFdx,dFdy,fwidth,noise1,noise2,noise3,noise4,
		EmitVertex,EndPrimitive},
	morekeywords=[3]{
		gl_VertexID,gl_InstanceID,gl_Position,
		gl_PointSize,gl_ClipDistance,gl_PerVertex,
		gl_Layer,gl_ClipVertex,gl_FragCoord,
		gl_FrontFacing,gl_ClipDistance,gl_FragColor,
		gl_FragData,gl_MaxDrawBuffers,gl_FragDepth,
		gl_PointCoord,gl_PrimitiveID,
		gl_MaxVertexAttribs,gl_MaxVertexUniformComponents,
		gl_MaxVaryingFloats,gl_MaxVaryingComponents,
		gl_MaxVertexOutputComponents,
		gl_MaxGeometryInputComponents,
		gl_MaxGeometryOutputComponents,
		gl_MaxFragmentInputComponents,
		gl_MaxVertexTextureImageUnits,
		gl_MaxCombinedTextureImageUnits,
		gl_MaxTextureImageUnits,
		gl_MaxFragmentUniformComponents,
		gl_MaxDrawBuffers,gl_MaxClipDistances,
		gl_MaxGeometryTextureImageUnits,
		gl_MaxGeometryOutputVertices,
		gl_MaxGeometryOutputVertices,
		gl_MaxGeometryTotalOutputComponents,
		gl_MaxGeometryUniformComponents,
		gl_MaxGeometryVaryingComponents,gl_DepthRange},
	morecomment=[l]{//},
	morecomment=[s]{/*}{*/},
	morecomment=[l][keywordstyle4]{\#},
}


\begin{document}
\author{Marcin Wawrzonowski}
\title{Wykorzystanie GPU urządzeń mobilnych w~symulacji dynamiki tkanin}
\begin{titlepage}
\thispagestyle{empty}
\begin{textblock}{1}(-2.65,-1.65)
\includegraphics{figures/tytulowa_pusta_mgrinz.pdf}
\end{textblock}
\vspace{7.3cm}
\begin{center}
\fontfamily{ptm}
\selectfont
\Huge
Wykorzystanie GPU urządzeń mobilnych w~symulacji dynamiki tkanin
\end{center}
\begin{center}
\fontfamily{ptm}
\selectfont
Praca dyplomowa inżynierska
\end{center}
\vspace{7.9cm}
\begin{center}
\fontfamily{ptm}
\selectfont
\hspace{-1cm}
\begin{tabular}{l}
Wydział Fizyki Technicznej, Informatyki i Matematyki Stosowanej \\
Promotor: dr inż. Dominik Szajerman \\
Dyplomant: Marcin Wawrzonowski \\
Nr albumu: 180729
\end{tabular}
\end{center}
\vspace{-.5cm}
\begin{center}
\fontfamily{ptm}
\selectfont
\begin{textblock}{13}(0,0.4)
Łódź, 2016
\end{textblock}
\end{center}
\end{titlepage}

\tableofcontents

\chapter{Wprowadzenie}
\label{t:int}

\section{Aktualny stan zagadnienia}
\label{t:int:stateofart}

	\subsection{\dots}

\chapter{Teoria symulacji tkanin i obliczeń na GPU}
\label{t:teoria}


	\section{Analiza istniejących modeli symulacji tkanin}
	\label{t:teoria:analiza}
	
		\subsection{Model masy na sprężynie}
		\label{t:teoria:analiza:masa}		
			
			Pierwszym z rozważanych w niniejszej pracy modeli symulacji tkanin jest model masy na sprężynie. Wiadomo, że rysowana przez API graficzne tkanina jest w postaci siatki wielokątowej. Siatka taka składa się z punktów w przestrzeni 3D -- wierzchołków. Na potrzeby symulacji przyjmujemy, że każdy z tych wierzchołków ma określoną masę i poddajemy go działaniu sił, w wyniku których następuje przemieszczenie. Aby zachować kształt i odpowiednie dla tkaniny zachowanie się siatki, wierzchołki połączone są sprężynami o określonych współczynnikach sprężystości oraz tłumienia drgań. 
			
			
			\myfigure{Rysunek 2.1}{figures/pic_2_1.png}{0.4}{Schemat modelu masy na sprężynie}
			
			
			Rysunek 2.1 przedstawia przykładowy fragment tkaniny opartej o model masy na sprężynie. Zakładamy dla uproszczenia, iż poszczególne wierzchołki tworzą kształt prostokątów, aczkolwiek w praktyce mogą przyjmować dowolne ustawienia. Ważne jednak dla zachowania poprawnej symulacji jest to, by punkty masy były równomiernie rozłożone w całej powierzchni tkaniny. W omawianym przypadku tak właśnie jest. 
			
			Możemy zaobserwować trzy rodzaje sprężyn, jakie występują w modelu na Rysunku 2.1. Kolorem czerwonym zostały oznaczone sprężyny strukturalne, które służą do utrzymania ogólnego kształtu tkaniny. Jednakże one same nie są w stanie zasymulować zachowania tkaniny w poprawny i miły dla oka sposób. Kolorem zielonym narysowane zostały sprężyny odpowiedzialne za wierne oddanie zgięć tkaniny, położone są one wzdłuż diagonalnych krawędzi siatki. Kolorem niebieskim zaznaczono sprężyny, których obecność zapewnia tkaninie odpowiednią elastyczność i chroni przed nadmiernym jej rozciąganiem. Łączą one nie sąsiednie wierzchołki, lecz następne, za sąsiadem, w tym samym kierunku. Każdy z rodzajów sprężyn może być opisany innymi współczynnikami sprężystości i tłumienia drgań, co pozwala na uzyskanie innych zachowań tkaniny w symulacji.
			
			
			\myfigure{Rysunek 2.2}{figures/pic_2_2.png}{0.3}{}
			
			
			Na Rysunku 2.2 widzimy, jakie siły oddziałują na każdy punkt masy -- wierzchołek siatki tkaniny. Wszystkie równania oprócz (2.4) zostały zaczerpnięte z \cite{cloth-dobre-wzory}. Siły możemy zaklasyfikować jako wewnętrzne i zewnętrzne. Z zewnętrznych wyróżniamy siłę grawitacji, opisaną wzorem:
			
			\begin{equation}
			F_{g} = m_{i} \cdot g \ .
			\end{equation}
			
			Kolejna siła zewnętrzna to siła oporu powietrza. Zgodnie z Prawem Stokesa jest ona proporcjonalna do prędkości punktu masy oraz pewnego współczynnika oporu \emph{k}:
			
			\begin{equation}
			F_{a} = -k_{a} \cdot v_{i} \ .
			\end{equation}
			
			Siłami wewnętrznymi działającymi na wierzchołki tkaniny są oczywiście siły sprężystości, wynikające z istnienia omówionych wyżej sprężyn. Do wyznaczenia jej wartości wykorzystujemy Prawo Hooke'a, mówiące, że siła sprężystości oraz jej kierunek i zwrot są proporcjonalne do wychylenia sprężyny, tj. różnicy odległości między jej aktualną długością, a długością w stanie spoczynku:
			
			\begin{equation}
			F_{se} = - \sum_{n = 0}^{n < 12} k_{s} (|x_{i} - x_{j}| - l_{(i, j)}) \cdot \frac{x_{i} - x_{j}}{|x_{i} - x_{j}|} \ ,
			\end{equation}
			
			gdzie \(k_{s}\) -- współczynnik sprężystości, \(x_{i}\) oraz \(x_{j}\) -- położenia wierzchołków połączonych daną sprężyną, \(l_{(i, j)}\) -- odległość między tymi punktami w stanie spoczynku.
			
			Zgodnie z \cite{receptury}, wprowadzamy także siłę tłumienia drgań sprężystych, aby zminimalizować niepotrzebne, nierealistyczne drgania oraz ryzyko wymknięcia się symulacji spod kontroli ("wybuchnięcia" - obliczane siły są takie, że pozycje wierzchołków dążą do nieskończoności). Przedstawia się ona następującym wzorem:
			
			\begin{equation}
			F_{sd} = \sum_{n = 0}^{n < 12} k_{d} (\frac{|x_{i} - x_{j}| \cdot |v_{i} - v_{j}|}{l_{(i, j)}}) \ ,
			\end{equation}
			
			gdzie oznaczenia są takie same, jak wyżej z tym, że \(k_{d}\) jest w tym przypadku współczynnikiem tłumienia drgań, \(|v_{i} - v_{j}|\) oznacza różnicę prędkości obu wierzchołków, a działanie \(|x_{i} - x_{j}| \cdot |v_{i} - v_{j}|\) to iloczyn skalarny wektora różnicy położenia i wektora różnicy prędkości. Końcowa siła sprężystości jest sumą dwóch powyższych wzorów i możemy zapisać ją w postaci:
			
			\begin{equation}
			F_{s} = F_{se} + F_{sd} \ ,
			\end{equation}
			\begin{equation}
			F_{s} = \sum_{n = 0}^{n < 12} - k_{s} (|x_{i} - x_{j}| - l_{(i, j)}) \cdot \frac{x_{i} - x_{j}}{|x_{i} - x_{j}|} + k_{d}(\frac{|x_{i} - x_{j}| \cdot |v_{i} - v_{j}|}{l_{(i, j)}}) \ .
			\end{equation}
			
			W sumie dla każdego wierzchołka rozważamy 12 sił sprężystości dla wszystkich przyłączonych do niego sprężyn, siłę grawitacji oraz oporu powietrza. Należy zauważyć, że nie uwzględniamy tutaj żadnych sił związanych z reakcją na kolizje -- będą one rozwiązane w późniejszej sekcji algorytmu. Wzór na siłę wypadkową działającą na pojedynczy wierzchołek tkaniny oraz wypadkowe przyspieszenie można zapisać w postaci:
			
			\begin{equation}
			F = F{s} + F{g} + F{d}		
			\end{equation}
			
			\begin{equation}
			a(t) = \frac{F}{m_{i}}	
			\end{equation}
			
			W celu wyznaczenia zmiany położenia punktu masy w danym kroku symulacji, posługujemy się, zgodnie z \cite{cloth-dobre-wzory} i \cite{receptury}, całkowaniem Verleta, będącym jedną z technik całkowania numerycznego nie wprost. Opisane jest ono wzorem:
			
			\begin{equation}
			x(t + \delta t) = 2x(t) - x(t - \delta t) + a(t) \delta t^{2} \ ,		
			\end{equation}
			
			gdzie \(a(t)\) jest przyspieszeniem, a \(x(t + \delta t)\), \(x(t)\) i \(x(t - \delta t)\) oznaczają położenia wierzchołka w następnym, obecnym oraz poprzednim kroku symulacji. 
			
			Całkowanie to jest równie proste obliczeniowo i implementacjyjnie, jak najzwyklejsza metoda Eulera, a zapewnia dużo stabilniejsze zachowanie się symulacji i minimalizację tendencji do "wybuchnięcia". Niejawnie obliczona zostaje tutaj aktualna prędkość wierzchołka, co sprawia, że do symulatora musimy dostarczyć nie tylko obecną pozycję każdego punktu masy, ale także położenie poprzednie. Zwiększa to koszt pamięciowy symulacji względem innych technik całkowania, lecz zapewnia bardzo szybkie obliczenia i stabilne ich wyniki.
		
		\subsection{Model oparty na pozycji}
		\label{t:teoria:analiza:poz}
		
			Model oparty na pozycji i model masy na sprężynie mają pewną część wspólną - jest to obliczanie przesunięć wynikających z sił grawitacji oraz oporu powietrza za pomocą całkowania Verleta. Podejście do symulacji ruchów wynikających ze struktury samej tkaniny jest jednak kompletnie odmienne. Oczywiście tkanina ma w tym modelu taką samą postać, jak i w poprzednim - zbiór wierzchołków, połączonych krawędziami, tworzących prostokątne kształty. Ponadto zaznaczyć należy, że w niniejszej pracy została wykorzystana tylko część aktualnie omawianego modelu, bardziej szczegółowo opisanego w \cite{posbased}. Prezentowane tu rozwiązanie nie bierze pod uwagę ograniczników zginania oraz całego systemu detekcji kolizji, zaproponowanego w rozdziale 4 \cite{posbased}.
			
			Przesunięcia wynikające z sił zewnętrznych działających na układ nazywamy tutaj przesunięciami przewidywanymi. Każdy wierzchołek tkaniny opisywany jest, oprócz, masy, pozycji i prędkości, także przez zbiór tzw. ograniczników. Każdy z nich definiowany jest poprzez pewną funkcję \(C_{j} : R^{3n_{j}} \rightarrow R\), zestaw indeksów \(\{ i_{1}, \dots, i_{n_{j}}  \}, i_{k} \in [1, \dots, N] \) i parametr sztywności \(k \in [0\dots1] \). Ogranicznik może być typu równości, co oznacza, że jego ograniczenie jest spełnione, kiedy \( C_{j}(x_{i_{1}}, \dots, x_{i_{n_{j}}} ) = 0 \). Może być także typu nierówności, i spełnia je warunek \( C_{j}(x_{i_{1}}, \dots, x_{i_{n_{j}}} ) \geq 0 \). W naszym przypadku będziemy rozważać tylko ograniczniki pierwszego typu. Kluczowym elementem jest oczywiście funkcja \(C_{j}\), która określa sposób, w jaki przewidywana pozycja zostanie poprawiona, czyli w efekcie zachowanie się tkaniny. Funkcja ta może być zupełnie inna, gdy rozważać będziemy ograniczenia przesuwania, a inna przy ograniczeniach zginania bądź kolizjach. Widać, że dzięki temu model ten charakteryzuje się pewną elastycznością, a ograniczniki są narzędziem, przy pomocy którego możemy realizować wiele rozmaitych założeń dotyczących ruchu tkaniny. Przesunięcia wynikające z nałożonych ograniczników są obliczane po kolei, następnie pozycja przewidywana jest poprawiana do takiej, która spełnia wszystkie warunki, równości bądź nierówności, dla funkcji \(C_{j}\) ograniczników. Proces ten autorzy \cite{posbased} nazywają projekcją. Pod uwagę brana jest także sztywność ogranicznika, czyli procent, w jakim się go stosuje.
			
			\myfigure{Rysunek 2.2}{figures/pic_2_3.png}{0.3}{}
			
			Podstawowym typem ogranicznika jest ogranicznik rozciągania. To on definiuje ogólny kształt i odpowiednie zachowanie tkaniny. Jego funkcja ma postać:
			
			\begin{equation}
			C(p_{1}, p_{2}) = |p_{1} - p_{2}| - d \ .		
			\end{equation}
			
			Gdzie \( p_{1} \) i \( p_{2} \) są pozycjami rozpatrywanych wierzchołków, a \(d\) - początkową odległością między nimi. Na Rysunku 2.3 widzimy efekt projekcji. W [3] wyprowadzone zostają wzory na jej obliczenie, na podstawie funkcji \( C_{j}(x_{i_{1}}, \dots, x_{i_{n_{j}}} ) \):
			
			\begin{equation}
			s = \frac{C_{j}(p_{i_{1}}, \dots, p_{i_{n_{j}}} )}{ \sum _{j} w_{j} | \nabla _{p_{j}} C_{j}(p_{i_{1}}, \dots, p_{i_{n_{j}}} ) | ^{2} } \ ,		
			\end{equation}
			
			\begin{equation}
			\delta p_{i} = -sw_{i} \nabla _{p_{i}} C_{j}(p_{i_{1}}, \dots, p_{i_{n_{j}}} ) \ .
			\end{equation}
			
			Gdzie \(w_{i}\) jest odwrotnością masy wierzchołka tkaniny. Biorąc pod uwagę, iż przemieszczenie jest do niej wprost proporcjonalne, możemy łatwo wyobrazić sobie dowód na poprawność tego podejścia - w przypadku, gdy masa cząstki jest nieskończona, przesunięcie będzie równe zeru. Kiedy na miejsce funkcji \(C_{j}(x_{i_{1}}, \dots, x_{i_{n_{j}}} ) \) wstawimy \(C(p_{1}, p_{2}) = |p_{1} - p_{2}| - d\), otrzymamy po przekształceniach: 
			
			\begin{equation}
			\delta p_{1} = - \frac{w_{1}}{w_{1} + w{2}} (|p_{1} - p_{2}| - d) \frac{p_{1} - p_{2}}{|p_{1} - p_{2}|} \ ,
			\end{equation}
			
			\begin{equation}
			\delta p_{2} = \frac{w_{2}}{w_{1} + w{2}} (|p_{1} - p_{2}| - d) \frac{p_{1} - p_{2}}{|p_{1} - p_{2}|} \ .
			\end{equation}
			
			Możemy zauważyć, że wzory (2.13) i (2.14) są bardzo podobne do wzoru (2.3). Tak samo, jak w modelu masy na sprężynie, "siła" ogranicznika zależy od różnicy aktualnej odległości pomiędzy punktami masy i odległości spoczynkowej. Rolę współczynnika sprężystości pełni tutaj parametr sztywności, przez który mnożymy na koniec przesunięcie będące wynikiem projekcji. Dla \(k\) równego 0 ogranicznik nie będzie w ogóle brany pod uwagę, a dla równego 1 -- punkt nigdy nie zmieni swojej początkowej pozycji.
			
			W rozważanym przypadku nie stosujemy ograniczników zginania, używamy także innej metody detekcji kolizji. Autorzy \cite{posbased} używają ograniczników rozciągania biorąc pod uwagę tylko wierzchołki leżące w sąsiedztwie danego punktu. Okazuje się, że podobny do użycia ograniczników zginania efekt osiągamy zwiększając zbiór rozpatrywanych wierzchołków o te leżące jedną pozycję w siatce dalej - tak jak sprężyny oznaczone kolorem niebieskim na Rysunku 2.1. Przykładowo, ograniczając elastyczność przemieszczania wierzchołka \(A\) względem \(C\) ograniczamy przecież tak naprawdę możliwość zginania się trójkątów \(ABD\) i \(BCD\) na wspólnej krawędzi \(BD\). Nie jest to tak dokładna metoda jak ograniczniki zginania, gdzie regulujemy kąt pomiędzy trójkątami, lecz mimo to daje ona poprawny wizualny efekt, jak pokazane zostanie w Rozdziale 5. Jest też szybsza do obliczenia przez procesor, jako że sam wzór jest prostszy.
		
		\subsection{Porównanie powyższych metod}
		\label{t:teoria:analiza:porownanie}
		
		\subsection{Detekcja kolizji}
		\label{t:teoria:analiza:kolizje}
		
			\subsubsection{Kolizje zewnętrzne}
			\label{t:teoria:analiza:kolizje:zewn}
			
			\subsubsection{Kolizje wewnętrzne}
			\label{t:teoria:analiza:kolizje:wewn}
			
			
	\section{Zastosowanie GPU w symulacji tkanin}
	\label{t:teoria:gpu}
	
		\subsection{Architektura GPU}
		\label{t:teoria:gpu:architektura}
		
		\subsection{Porównanie GPU i CPU pod kątem architektury i zastosowań}
		\label{t:teoria:gpu:porownanie}
		
		\subsection{Zalety zastosowania GPU w symulacji tkanin}
		\label{t:teoria:gpu:zalety}

\chapter{Wykorzystane technologie}
\label{t:technologie}


	\section{Analiza możliwości urządzeń mobilnych}
	\label{t:technologie:mobilne}
	
		\subsection{Sensowność wykorzystania urządzeń mobilnych w symulacji tkanin}
		\label{t:technologie:mobilne:dlaczego}
		
		% niższa wydajność ALE unikalne możliwości interakcji i wszechobecna dostępność, wykorzystanie w grach 3D (rozwój tychże) i aplikacjach branży tekstylnej (włókienniczej) oraz odzieżowej, niższa wydajność wiążąca się z niższą jakością, wykorzystanie GPU by ją podnieść
		
		W związku z faktem, iż niniejsza praca zajmuje się symulacją tkaniny na urządzeniach mobilnych, naturalnie nasuwa się pytanie o~sens realizacji tego typu obliczeń z~użyciem smartfonu bądź tabletu. Prawdą jest, że mamy do czynienia ze sporym zapotrzebowaniem na moc obliczeniową, zwiększającym się proporcjonalnie do oczekiwanej dokładności rozwiązania, a~konkretnie - do gęstości siatki tkaniny. Z~drugiej strony, urządzenia mobilne, w~przeciwieństwie do platformy PC, szczytowymi osiągnięciami technologii pod względem wydajności nigdy nie były -- raczej starano się tu osiągnąć kompromis między sensowną mocą obliczeniową a~niskim zużyciem energii. Jednakże smartfony i~inne tego typu urządzenia cechują się dostępnością dla użytkownika praktycznie zawsze, co nie do końca może być powiedziane o~PC, oraz unikalnymi metodami interakcji, takimi jak ekran dotykowy, bądź akcelerometr.
		
		Mimo oczywistych wad, także i~na platformach mobilnych symulacja tkanin znajduje zastosowanie. Jako pierwszy i~najważniejszy przykład należy wskazać rynek gier i~wizualizacji 3D. Dawno minęły już czasy, gdy najpopularniejszą grą na telefonach komórkowych był kultowy, dwuwymiarowy ``\emph{Snake}''. Obecnie spora część rynku koncentruje się na złożonych grach trójwymiarowych, z~coraz ładniejszą grafiką, na potrzeby których tworzy się coraz bardziej zaawansowane silniki graficzne i~korzysta z najnowszych technologii. Podobnie, jak na platformie PC, także i~tutaj możliwe, a~nawet pożądane jest użycie symulacji tkanin do m.in. realistycznej animacji elementów stroju bohaterów, flag powiewających na wietrze oraz innych przedmiotów tekstylnych.
		
		Obecnie wielu producentów i~sprzedawców z różnych branż decyduje się na stworzenie i~wypuszczenie na rynek własnej, wyspecjalizowanej aplikacji dla urządzeń mobilnych, pozwalającej w prosty, przyjazny sposób przeglądać oferty, oglądać towary i~dokonywać zakupów. Zdecydowanie zwiększa to przychody danej firmy. Branżą, która mogłaby skorzystać na zastosowaniu symulacji tkanin w~swoich aplikacjach jest oczywiście branża włókiennicza i~odzieżowa. Przykładem może być chociażby stworzenie ``wirtualnej przymierzalni'' \cite{tryon}, przy pomocy której klient byłby w stanie ``ubrać się'' w każdy wybrany element odzieży. Aplikacja pozwoliłaby mu chociażby na obejrzenie go ze wszystkich stron, sprawdzenie elastyczności i~zachowania się go w różnych pozach. A to wszystko na ekranie tabletu, dostępne w~każdym możliwym miejscu. 
		
		Oczywistym jest, że niższa wydajność urządzeń mobilnych wiąże się z~niższą jakością symulacji. Warto jednak pamiętać, iż wyświetlacze urządzeń mobilnych z~reguły są mniejsze od ``pecetowych'' monitorów. Co za tym idzie -- możemy zastosować siatkę tkaniny o~mniejszej gęstości i~dokonywać mniej dokładnych obliczeń np. detekcji kolizji bez dużego spadku jakości wizualnej. Ten fakt, oraz omówiona wcześniej mnogość zastosowań sprawiają, że symulacja tkanin na urządzeniach mobilnych zdaje się jak najbardziej mieć sens. W rozdziałach \ref{t:wyniki} i~\ref{t:wnioski} przekonamy się, w~jakim stopniu.
	
		\subsection{Konfiguracja sprzętowa urządzeń mobilnych i porównanie z konfiguracją PC}
		\label{t:technologie:mobilne:konfiguracja}
		
		% architektura CPU, szybkości CPU, ilość core'ów, architektura GPU, ilosć SM, API GPU i GPGPU, ilość pamięci op., 
		
		Jak już wyżej wspomniano, wydajność konfiguracji sprzętowa urządzeń mobilnych to drobny ułamek wydajności komputerów klasy PC. Warto dokładniej zwrócić uwagę, jaka jest między tymi platformami różnica i~z~jakimi ograniczeniami się spotkamy, tworząc symulację tkanin na tej pierwszej. Porównania dokonamy na przykładzie urządzenia testowego -- smartfona LG Nexus 4 E960. Dane techniczne zaczerpniemy z~\cite{specs}, \cite{specs_adreno}, \cite{specs_gtx750} i \cite{specs_gtxtitan}. 
		
		Urządzenie oparto o mikrokontroler APQ8064 Snapdragon S4 Pro. Serce układu to czterordzeniowy procesor Krait o~taktowaniu 1.5 GHz i~architekturze ARMv7-A. Szybkość zegara jest przeszło dwa razy mniejsza niż w~przeciętnym odpowiedniku PC. Można stąd wnioskować, że wydajność mamy dwukrotnie mniejszą, jednak diabeł tkwi w szczegółach. Dzisiejsze procesory architektury x86 dysponują szerokim wachlarzem specjalnych instrukcji, takich jak SSE czy AVX, bardzo przyspieszających operacje wektorowe, typu SIMD. Jedynym ich odpowiednikiem w omawianym układzie są instrukcje NEON, dużo mniej wydajne. A~zatem, zgodnie z~\cite{versus}, procesor Krait cechuje ponad dziesięciokrotnie mniejsza wydajność niż jego przykładowego odpowiednika z~komputera klasy PC - Intel Core i7-4770.
		
		Układ Snapdragon jest także wyposażony w dedykowane GPU, specjalnie na potrzeby renderingu grafiki 2D i~3D. To Adreno 320, cechujące się taktowaniem zegara 400 MHz i~w sumie 64 procesorami strumieniowymi. Karta graficzna osiąga wydajność ok. 60 GFLOPS\footnote{\emph{Floating-point Operations Per Second} -- ilość operacji na liczbach zmiennoprzecinkowych w~czasie 1 sekundy.}. Dla porównania weźmy średniej klasy GPU komputerów stacjonarnych sprzed paru lat, GeForce GTX 750. Jego zegar to 1020 MHz, ma ono 512 SP, a~wydajnościowo plasuje się trochę ponad 1 TFLOPS. Jedno z~najpotężniejszych GPU obecnie, GeForce GTX Titan, cechuje z kolei 3072 procesorów strumieniowych i~ok. 6 TFLOPS. Pod uwagę bierzemy oczywiście obliczenia na liczbach zmiennoprzecinkowych pojedynczej precyzji. Widzimy więc, że mobilne GPU wydajnościowo stanowią zaledwie ułamek ich ``pełnowymiarowych'' odpowiedników.
		
		Ważną kwestią jest też dostępność i~obsługa odpowiednich technologii, a~w~szczególności API graficznych i~GPGPU. Tutaj na szczęście sytuacja ma się dużo lepiej. Adreno 320 wspiera zaawansowane już całkiem OpenGL ES 3.0 oraz Direct3D 11.1. Patrząc na sprawę pod kątem symulacji tkanin, brakuje tylko obsługi OpenGL ES 3.2, który wprowadził bardzo przydatne w omawianym problemie bufory teksturowe. Mimo to jednak API graficzne spełniają wymagania niniejszej pracy. Oczywiście z~ogólnego punktu widzenia, brakuje wielu nowych rozwiązań, wprowadzonych w najnowszych ``dużych'' wersjach OpenGL. Do obliczeń GPGPU wspierane są technologie OpenCL 1.1, będący open-source'owym odpowiednikiem CUDA, oraz RenderScript. W~tej pracy jednak okazało się niemożliwe użycie żadnego z nich. Obsługa OpenCL została wycofana na urządzeniach Google, w~tym na Nexusie 4, z~powodów marketingowych, a~konkretnie, z~powodu promowania drugiej wymienionej technologii. Tej z~kolei nie można było użyć z~racji tego, iż API nie pozwala jawnie wybrać i~ustalić, czy obliczenia będą dokonywane na GPU, czy na CPU.
		
		\subsection{Unikalne możliwości platform mobilnych w interakcji użytkownika z tkaniną}
		\label{t:technologie:mobilne:interakcja}
		
		% ekran dotykowy, akcelerometr
		
		Urządzenia mobilne cechuje jedna ważna przewaga nad komputerami PC -- charakterystyczne tylko dla nich interfejs. Sztandarowym przykładem jest oczywiście ekran dotykowy. W~rozważanej kwestii, z~jego pomocą użytkownik może ruchami palca po ekranie przemieszczać i~rozciągać tkaninę. Pozwala to na nadanie jej ruchu w~odpowiednią stronę, sprawdzanie elastyczności oraz dokładne, precyzyjne ustawienie w~wirtualnym świecie poprzez sprowokowanie kolizji z~obiektami otoczenia. Przydatne może być to szczególnie w omawianym wcześniej przypadku wykorzystania symulacji do aplikacji typu ``przymierzalnia''. Klient jest w stanie realistycznie założyć oglądany element ubioru na swojego awatara i~dokładnie go dopasować. 
		
		Kolejnym charakterystycznym dla platformy mobilnej urządzeniem wejścia jest akcelerometr. Najważniejszy przykład jego wykorzystania to możliwość zmiany kierunku siły grawitacji działającej na symulowany układ, poprzez obracanie telefonu w~odpowiednim kierunku. Użytkownik będzie miał wrażenie obracania układem, a~siła grawitacji pozostanie stała i~skierowana w dół względem niego. Stwarza to kolejne możliwości zmiany położenia tkaniny, dokładnego jej układania i~odwzorowania realizmu świata wirtualnego.

	
	\section{Przedstawienie wybranych technologii i narzędzi}
	\label{t:technologie:narzedzia}
	
		\subsection{Android NDK}
		\label{t:technologie:narzedzia:ndk}
		
		% jedyne sensowne rozwiązanie jesli sie chce wycisnąć 100% wydajności z aplikacji, odradzany przez dokumentację, % C++ !	
		
		Wiodącym językiem programowania na platformie Android jest język Java. Wykorzystujące go Android SDK posiada bardzo szeroką dokumentację i~pomoc techniczną. Można na jego temat znaleźć wiele publikacji, a~w~przypadku, gdy mamy z~nim jakieś trudności, w~Internecie na pewno znajdziemy rozwiązanie. Istnieje jednak jeden duży problem -- Java nie jest językiem natywnym i~uruchamia się na maszynie wirtualnej. Przez to nie nadaje się do aplikacji, w~których kluczowa jest wydajność, takich jak na przykład symulacja fizyczna tkanin.
		
		Jest to powód, dla którego cały projekt został napisany w~języku C++. Umożliwiło to wykorzystanie Android NDK\footnote{Native Development Kit.}, czyli zbioru większości funkcjonalności Androida, dostępnych z~poziomu C bądź C++. Kod ten zostaje połączony z~inicjalizacyjnym kodem Javy przy pomocy technologii JNI\footnote{Java Native Interface}, będącej swoistym pomostem pomiędzy dwoma platformami. Cykl życia aplikacji kontrolowany z~poziomu C++ jest niemalże identyczny, jak w przypadku języka Java, opiera się na zdarzeniach wywoływanych przez system operacyjny. W~ten sposób zarządzamy m.in. inicjalizacją i~zwalnianiem pamięci oraz sygnałami przychodzącymi z~urządzeń wejściowych, takich jak ekran dotykowy czy akcelerometr. Odwrotnie jak w~przypadku PC, gdzie sami ręcznie ``wyciągamy'' informację o~tym, czy dany klawisz na klawiaturze jest wciśnięty. Za pomocą plików konfiguracyjnych i~tzw. \emph{Android Manifest} mamy kontrolę nad wszelkimi opcjami konfiguracyjnymi dotyczącymi aplikacji. Możemy tam ustalić np. jej nazwę, widoczność elementów interfejsu systemu oraz to, czy aplikacja sama potrafi obsłużyć zmianę orientacji ekranu, czy należy ją wtedy zresetować.
		
		Jedna z~implementacji symulacji zakłada przetwarzanie jej wielowątkowo na CPU. Jako, że Android wywodzi się z~rodziny UNIX'ów, korzystając z~C++ mamy do dyspozycji bibliotekę \emph{pthread}. Umożliwia ona łatwe i~przejrzyste uruchamianie nowych wątków, zarządzanie nimi oraz ich synchronizację.
	
		\subsection{OpenGL}
		\label{t:technologie:narzedzia:ogl}
		
		% różnice między nimi i brak paru ważnych rzeczy!
		
		Stworzenie samej symulacji tkanin nie ma sensu, jeśli nie jesteśmy w stanie pokazać jej działania. Konieczne więc było stworzenie silnika wizualizującego wygląd i~zachowanie naszej tkaniny. OpenGL to darmowe API graficzne, czyli zestaw bibliotek służących do komunikacji z GPU i~w efekcie -- rysowania grafiki 2D i~3D. Najważniejszą cechą tej technologii jest jej dostępność na praktycznie wszystkich platformach. Właśnie z~tego powodu została ona użyta w niniejszej pracy, jako że jest obsługiwana zarówno na smartfonach z systemem Android, jak i~na komputerach PC z Windows. W pierwszym przypadku mamy do czynienia z wersją OpenGL ES 3.0, będącą okrojoną i~przystosowaną do użytku na platformach mobilnych oraz innych systemach wbudowanych. Na PC użyliśmy standardowego OpenGL 3.3. Nie są to najnowsze edycje OpenGL, jednak oferują już programowalny potok renderingu i~systemy wspierające GPGPU, takie jak transformacyjne sprzężenie zwrotne. Uznaliśmy je więc za wystarczające do zrealizowania niniejszej pracy.
		
		Jak możemy przeczytać w~\cite{oglspec}, OpenGL wymaga także API pomocniczego, zarządzającego tworzeniem okien, do których można rysować grafikę, przydzielaniem kontekstów graficznych, wymaganych by cokolwiek dało się w~tej materii zadziałać, oraz innymi zasobami. W przypadku programu na platformie Android, użyte zostało tu EGL -- specjalne API do systemów mobilnych i~wbudowanych, dzieła twórców OpenGL. W wersji ``pecetowej'' wykorzystaliśmy darmowe API GLFW oraz GLEW.
		
		Ważną kwestią w tworzeniu zarówno symulacji fizycznych, jak i~samego silnika graficznego jest odpowiednia baza matematyczna. W niniejszej pracy użyta została darmowa biblioteka GLM, zalecana jako nieodłączny element programowania w OpenGL. Zapewnia ona dostęp do wielu wygodnych i~przydatnych funkcji oraz struktur matematycznych, ułatwiających szczególnie obliczenia na wektorach oraz macierzach, kluczowe w grafice 3D. GLM w swoim założeniu ma jak najbardziej przypominać składnię i~semantykę związanego z OpenGL języka programów cieniujących -- GLSL, o~którym więcej w Podrozdziale \ref{t:technologie:narzedzia:ogl:glsl}.
		
			\subsubsection{OpenGL ES 3.0 kontra OpenGL 3.3}
			\label{t:technologie:narzedzia:ogl:vs}
			
			Według \cite{oglspec}, każda wersja OpenGL ES jest podzbiorem funkcji pewnej ``dużej'' wersji OpenGL. W przypadku użytego tutaj Nexusa 4, obsługiwana jest edycja ES 3.0, a~jej odpowiednikiem na większych platformach -- wersja 3.3. Właśnie dlatego ta ostatnia posłużyła do stworzenia ``pecetowego'' modelu symulacji tkanin. Dzięki temu można było zapewnić jak największą kompatybilność i~łatwość przeniesienia z~jednej platformy na drugą.
			
			Filozofią przyświecającą deweloperom przy tworzeniu edycji OpenGL ES jest przede wszystkim to, by każda czynność, którą możemy zrobić przy pomocy API, była osiągalna tylko w jeden, konkretny sposób. Idąc tym tropem, ujednolicono wiele funkcji i~usunięto te, które się dublowały, bądź w dalszej perspektywie dawały identyczne efekty. Dzięki temu API stało się bardziej przejrzyste oraz łatwiej się z~niego korzysta. Warto zaznaczyć, że użyteczność nie zmniejszyła się względem wersji API 3.3, a po prostu ``zrobiono porządek''.
			
			\subsubsection{GLSL}
			\label{t:technologie:narzedzia:ogl:glsl}
			
			% zalety - zintegrowany z API graficznym, ogólnodostępny, szybki; wady - konieczność dostosowania GPGPU do API graficznego, mało funkcjonalności wspierających GPGPU na OpenGL ES
			
			GLSL jest językiem specjalistycznym dla API OpenGL, przy pomocy którego piszemy programy cieniujące, czyli tzw. \emph{shadery}. W niniejszej pracy wykorzystany został do stworzenia podstawowego, prostego modelu cieniowania opartego o~wzory Phonga-Blinna. Z racji opisanego w~Podrozdziale \ref{t:technologie:mobilne:konfiguracja} braku innych sensownych technologii na wybranej platformie sprzętowej, użyliśmy go także do napisania obliczeń symulacji tkanin na GPU, tj. wyliczenia przesunięć wierzchołków, rozwiązania kolizji zewnętrznych i~wewnętrznych oraz przeliczenia wektorów normalnych.
			
			Niewątpliwą zaletą GLSL jest jego integracja z API graficznym i~brak konieczności integracji dodatkowych bibliotek. Chcąc prowadzić specjalistyczne obliczenia dla tkaniny, możemy korzystać z tych samych buforów wierzchołków, których używamy do jej rysowania. Bardzo łatwo uniknąć jest niepotrzebnego kopiowania danych. Co za tym idzie -- GLSL jest bardzo optymalnym pod względem wydajności rozwiązaniem. Wart wspomnienia jest także fakt jego dostępności, możemy, właściwie musimy, z~niego skorzystać wszędzie tam, gdzie możliwe jest uruchomienie biblioteki OpenGL. 
			
			GLSL ma jednak kilka wad, które dają o~sobie znać w momencie, gdy chcemy wykorzystać go do obliczeń GPGPU. Obojętnie, jaki problem trzeba rozwiązać, musimy dostosować dane oraz algorytmy do struktur danych potoku renderingu i~do funkcji API graficznego. W~rozważanym przypadku nie jest to jednak duży problem. Gorszą bolączką jest tak naprawdę niewielkie wsparcie dla GPGPU w OpenGL ES 3.0, omówione zostanie ono w~następnym podrozdziale. Warto wspomnieć, że najnowsze wersje API mają już pełną obsługę obliczeń GPGPU, czego przejawem jest chociażby obecność \emph{Compute Shaderów}, czyli programów ogólnego przeznaczenia uruchamianych na GPU.
			
			\subsubsection{Bufory teksturowe i bufory jednorodne}
			\label{t:technologie:narzedzia:bufory}
			
			Dużym minusem urządzenia testowego jest brak obsługi OpenGL ES 3.2, w którym dodana została funkcjonalność zwana buforami teksturowymi. Zostały one wykorzystane do zaimplementowania prostej symulacji tkanin w~\cite{receptury}, co dowodzi ich kluczowej przydatności. Wykonując obliczenia na każdym wierzchołku, musimy mieć dostęp do pozycji wierzchołków sąsiednich, aby obliczyć odległości między nimi a~tym aktualnym. Prowadzi to do konieczności posiadania dostępu do bufora pozycji obecnych oraz~poprzednich \uline{wszystkich} wierzchołków przez każdy uruchomiony na GPU kernel, czy też program cieniujący. 
			
			Można tego dokonać idealnie wykorzystując bufory teksturowe, będące \emph{de facto} jednowymiarową teksturą, pod którą da się ``podpiąć'' jakikolwiek istniejący na karcie graficznej bufor danych. Jeden bufor teksturowy może pomieścić co najmniej 64 KB liczb, zależnie od pamięci konkretnej GPU \cite{opengl_wiki}. Zapewnia też bardzo szybki losowy odczyt danych. 
			
			Zamiast tego zmuszeni byliśmy użyć innej techniki, a mianowicie tzw. \emph{Uniform Buffer Objects}, co można przetłumaczyć jako bufory jednorodne. Wykorzystywane są one do łączenia danych trafiających do programów cieniujących w jeden ciągły blok i~dają możliwość szybkiego wykorzystania tego bloku w~wielu programach \cite{opengl_wiki}. Ich zastosowanie jest, jak widać, trochę inne niż to, o~które nam chodzi, jednak w~efekcie spełniają podobną rolę, jak bufory teksturowe. Musimy się jednak liczyć z~faktem, że maksymalna ich wielkość to tylko 16 KB danych, co ogranicza nam maksymalną możliwą gęstość siatki tkaniny. Odczyt jest tu także dużo wolniejszy, niż w~pierwszym przypadku.
			
			\subsubsection{Transformacyjne sprzężenie zwrotne}
			\label{t:technologie:narzedzia:transformfeedback}
			
			O mechanizmie OpenGL zwanym \emph{Transform Feedback}, czyli transformacyjnym sprzężeniem zwrotnym, wspomnieliśmy już w~Podrozdziale \ref{t:teoria:gpu:architektura}. Znajdujący się tam Rysunek \ref{pic_2_9} przedstawia m.in. ogólną zasadę działania tej technologii. Jak wiemy, dane w potoku renderingu koniec końców trafiają do bufora ramki, bądź tylnego bufora i~są po drodze przetwarzane przez kilka różnych etapów. Aby w~ogóle móc dokonywać obliczeń GPGPU korzystając z~API graficznego, musimy mieć możliwość odczytu danych wyjściowych. Następnie albo uznamy, że są to nasze wyniki obliczeń, albo podamy je znowu na wejście potoku do ponownego przetworzenia. 
			
			Taką możliwość daje nam właśnie transformacyjne sprzężenie zwrotne. Dane trafiają do shadera wierzchołków, gdzie program cieniujący wykonuje na nich ciąg operacji. Następnie zamiast zostać podane do rasteryzera, zapisywane są do ustalonego wcześniej, specjalnego bufora, oznaczonego jako bufor sprzężenia zwrotnego. Po zakończeniu działania \emph{Transform Feedbacku} mamy dostęp do niego, tak samo, jak do każdego innego bufora w~OpenGL. Opisywana symulacja tkanin jest idealnym przykładem zastosowania tego rozwiązania, jako, że takie same obliczenia wykonujemy w każdym kroku, a~następnie po prostu zamieniamy miejscami bufory wejściowe i~wyjściowe. 
		
		%\subsection{CUDA}		% nie, bo projekt PC-towy trzeba przerobić idealnie tak jak jest androidowy zrobiony
		%\label{t:technologie:narzedzia:cuda}
		
		\subsection{Visual Studio 2015 Community + Cross-platform Development Kit}
		\label{t:technologie:narzedzia:vs}
		
		% nowość, pełna intergracja z VS, zero Javy (też i wada), alternatywa - płatny Xamarin
		
		Jeśli chodzi o~środowisko programistyczne, wykorzystano najnowszą edycję Microsoft Visual Studio w~darmowej wersji Community. Wybór został dokonany ze względu na jego wszechstronność i~możliwości, a~przede wszystkim dobrą znajomość. Nowością w wydaniu 2015 aplikacji jest pakiet Cross-platform Development Kit. Umożliwia on budowanie programów na platformy mobilne, takie jak Android lub iOS. Dotychczas było to wspierane tylko w~Eclipse, bądź Android Studio, będącym jego specjalistyczną edycją dla programowania na Androidzie. Nowy pakiet Microsoftu pozwala na tworzenie aplikacji w natywnym języku C++, z użyciem opisanych w~Podrozdziale \ref{t:technologie:narzedzia:ndk} bibliotek. Zapewnia odgórnie także integrację z systemem operacyjnym, co jest dużym ułatwieniem dla początkujących programistów, którzy nie muszą się zajmować zawiłościami inicjalizacji natywnego kodu. 
		
		Wiąże się z~tym jednak poważne ograniczenie -- oznacza to bowiem, że nie mamy dostępu do części aplikacji w Javie, tworzonej tu automatycznie. Język ten jest w ogóle nieobsługiwany przez Visual Studio, ale teoretycznie wszystkie potrzebne zasoby, np. referencję do menedżera assetów\footnote{Zasobów dodatkowych, potrzebnych do działania aplikacji, np. tekstur, dźwięków, siatek geometrycznych, itp.}, dostajemy bezpośrednio przy starcie programu. Jeżeli jakieś rozwiązanie wymaga bardziej zaawansowanej komunikacji między Javą a C++ -- nie możemy z~niego skorzystać.
		
		Microsoft pomyślał też i~o~tym, wprowadzając pakiet Xamarin. Daje on możliwość pisania ``wysokopoziomowej'' części kodu przeznaczonego na platformy mobilne w~C\#. Jednakże jest to już opcja płatna, a~edycja darmowa wprowadza duże ograniczenia odnośnie budowanych aplikacji, ograniczając chociażby rozmiar pliku wykonywalnego do 128 kB i~uniemożliwiając łączenie go z bibliotekami tworzonymi w~innych językach, jak C/C++ lub Java.
		
\chapter{Budowa i działanie aplikacji}
\label{t:praktyka}
	
	\section{Cele oraz możliwości aplikacji}
	\label{t:praktyka:cel}
	
	% co program ma robić
	
	Na potrzeby niniejszej pracy została stworzona aplikacja prezentacyjna. Realizuje ona kilka kluczowych celów. Po pierwsze, ma za zadanie zaprezentować działanie dwóch omówionych w~rozdziale \ref{t:teoria} modeli symulacji tkanin. Musi pozwalać na porównanie ich pod względem wydajności, stabilności i~efektu wizualnego. Wydajność rozumie się jako czas potrzebny na obliczenie jednego kroku symulacji, im mniejszy, tym oczywiście lepiej. Aplikacja informuje o~nim użytkownika, wyświetlając stosowną informację w formie tekstowej. Jeśli chodzi o~dwa następne czynniki, najlepiej ocenić zachowanie tkaniny wizualnie. W~tym celu program rysuje ją w~przestrzeni 3D. 
	
	Kluczową kwestią jest tutaj interakcja z~innymi obiektami. Aplikacja tworzy wirtualną scenę i~umieszcza w~niej pewną liczbę podstawowych kształtów geometrycznych, takich jak płaszczyzna, prostopadłościan, bądź sfera. Dzięki temu można sprawdzić, jak zachowa się tkanina, wchodząc w kolizje z~tymi elementami. Istnieje także opcja przemieszczania wybranego obiektu po scenie, co pozwala na tworzenie różnych konfiguracji zderzeń pomiędzy nim a~przedmiotem symulacji. Warto wspomnieć, że inne elementy sceny także kolidują ze sobą.
	
	Celem była także możliwość porównania prędkości obliczeń symulacji modeli tkanin na CPU i~GPU oraz zbadanie różnicy wydajności GPU urządzenia mobilnego i~GPU komputera PC. Aplikacja umożliwia określenie, który z~wymienionych wyżej komponentów sprzętowych będzie przetwarzać symulację. Występuje ona także w wersjach na obie rozpatrywane platformy, pozwalając na dokonanie wszelkich niezbędnych porównań.
	
	Ważną kwestią w~niniejszych rozważaniach są możliwości interakcji z~tkaniną, jakie udostępnia smartfon. Program umożliwia ją, pozwalając przemieszczać fragment symulowanego obiektu poprzez przesuwanie palca po ekranie dotykowym.
	
	Oprócz tego aplikacja dysponuje sporą liczbą ułatwień i~udogodnień dla użytkownika. Pozwala na sterowanie kamerą, także przy pomocy ekranu dotykowego, przesuwanie jej po płaszczyźnie XZ, obrót w dowolnym kierunku, przybliżanie i~oddalanie. Kluczowe informacje wyświetlane są w~formie tekstu: wspomniany wcześniej czas trwania obliczeń jednego kroku symulacji, liczba klatek na sekundę, czy też aktualnie przetwarzany model tkaniny. Zmienić można wiele różnych parametrów, opisanych szerzej w rozdziale \ref{t:symulacja} oraz tryb rysowania obiektów, jeżeli użytkownik chciałby zwrócić baczniejszą uwagę na zachowanie siatki -- tzw. tryb \emph{wireframe}, polegający na renderowaniu tylko krawędzi.
	
	\section{Ogólna architektura aplikacji}
	\label{t:praktyka:ogolne}
	
	% z jakich elementów składa się program (Singletony!), jak działają te elementy, jak są ze sobą powiązane,
	% ogólny algorytm pracy całego programu
	
	\myownfigure{Najważniejsze elementy aplikacji i wzajemne powiązania.}{figures/pic_4_1.png}{0.4}{pic_4_1}
	
	Na rysunku \ref{pic_4_1} przedstawione zostały główne komponenty silnika oraz zależności pomiędzy nimi. Klasy, których nazwy napisano pogrubioną czcionką są singletonami. Strzałka z~linią ciągłą oznacza, że klasa A wywołuje funkcje klasy B i~działanie B zależy od A. Jeżeli owa strzałka biegnie do nazwy klasy, znaczy to, iż wszystkie jej metody są wywoływane, a~jeśli do konkretnej nazwy funkcji -- tylko ona.
	
	Większość singletonów, ale też i~ważniejszych klas programu, została napisana zgodnie z~prostą architekturą \emph{Initialize -- Run -- Shutdown}. W przypadku sceny, encji, komponentów (klasy \texttt{Component}) i~ich pochodnych, funkcja \texttt{Run} została rozbita na oddzielne \texttt{Update} i~\texttt{Draw}. Wywoływane są one w~głównej pętli programu. Jak łatwo się domyślić, \texttt{Initialize} i~\texttt{Shutdown} uruchamia się odpowiednio przy starcie i~wyłączaniu aplikacji. Takie podejście zapewnia dużą przejrzystość w~strukturze wywołań funkcji silnika.
	
	Podstawowym elementem spajającym działanie całego silnika jest klasa \texttt{System}. Odpowiada ona za inicjalizację wszystkich singletonów -- menedżerów, ich aktualizację w~głównej pętli programu, zwalnianie pamięci przy wyłączaniu aplikacji oraz za obsługę zdarzeń przychodzących z systemu Android. W~jej gestii leży uśpienie i~wznowienie programu, gdy takie żądanie zostanie wywołane. Przechowuje także aktualnie wczytaną scenę (klasa \texttt{Scene}) i~w każdej klatce wywołuje jej metodę \texttt{Update}, odświeżając stan wszystkich encji. Także tutaj znajduje się referencja do struktury \texttt{Engine}. Ta struktura jest utrzymywana głównie na potrzeby komunikacji z~Androidem i~dzięki niej istnieje dostęp do wszystkich danych, jakie aplikacja dostaje od systemu. Wśród nich są m.in. wskaźnik do androidowej struktury \texttt{android\_app}, gdzie przechowywane są te informacje, rozmiary ekranu oraz identyfikatory kontekstu graficznego, powierzchni rysowania i~wyświetlacza, niezbędne bibliotece EGL w~inicjalizacji OpenGL.
	
	Drugim najważniejszym singletonem systemu jest klasa \texttt{Renderer}. Razem z~klasami pochodnymi \texttt{Mesh} skupia w~sobie wszystkie funkcje dotyczące renderingu grafiki 3D. Do jego odpowiedzialności należy inicjalizacja bibliotek EGL oraz OpenGL, odpowiedni wybór parametrów okna, utworzenie powierzchni rysowania oraz kontekstu. Następnym krokiem jest załadowanie wszystkich potrzebnych shaderów (plików z rozszerzeniem \emph{.glsl}) i~ustawienie wybranych parametrów OpenGL. Do tych ostatnich zalicza się m.in. wybór koloru, jakim czyszczony jest bufor ramki, włączenie testu głębokości, odcinania tylnych ścianek wielokątów, uruchomienie i~ustawienie funkcji mieszania przezroczystości. Oczywiście przy wywołaniu metody \texttt{Shutdown} usuwane są wszelkie dane związane z~renderingiem oraz niszczone wymienione wyżej elementy. Jego funkcja \texttt{Run} zawiera przede wszystkim obsługę zmiany rozmiaru ekranu, a~co za tym idzie, parametrów okna i~powierzchni rysowania, obsługę przełączania trybu wyświetlania obiektów, a~wreszcie -- renderingu elementów sceny oraz interfejsu użytkownika poprzez wywołanie metod \texttt{Draw} i~\texttt{DrawGUI} obiektu typu \texttt{Scene}. Założeniem projektowym dla tej klasy była enkapsulacja większości wywołań funkcji OpenGL, tak, by kod renderingu dało się łatwo wymienić na inny. W~związku z~tym, klasa \texttt{Renderer} posiada także specjalistyczne funkcje do wczytywania shaderów, kerneli i~tekstur oraz zwalniania pamięci po tych zasobach graficznych. Warto wspomnieć o~tym, że z~projektowego punktu widzenia, w~aplikacji rozróżnia się pomiędzy shaderem (używanym do rysowania obiektów) a~kernelem (używanym do obliczeń GPGPU, w~transformacyjnym sprzężeniu zwrotnym), chociaż z punktu widzenia ich wczytywania, są \emph{de facto} tym samym -- kodem GLSL, który przekształcony zostaje w~tzw. \emph{program} OpenGL. 
	
	Kluczowym dla działania symulacji komponentem jest klasa \texttt{Timer}. Jak sama nazwa wskazuje, zajmuje się ona wszystkimi czynnościami dotyczącymi zliczania czasu. Do pobrania aktualnego \(t\) użyto funkcji \texttt{clock\_gettime}, zawartej w~bibliotece \emph{time.h}. Oferuje ona dokładność co do nanosekund, jednak na potrzeby systemu symulacyjnego wszelkie wielkości czasowe są przechowywane w formacie milisekund -- jest to wystarczająca precyzja. \texttt{Timer} udostępnia następujące dane: czas całkowity, który upłynął od startu programu, czas, jaki mija pomiędzy kolejnymi krokami głównej pętli -- \(\delta t \), czyli tzw. \emph{delta time}, liczba klatek na sekundę (FPS -- \emph{Frames Per Second}), będąca odwrotnością \(\delta t \), liczba kroków głównej pętli programu od początku jego działania oraz tzw. \emph{fixed delta time}, czyli stała, uśredniona wartość czasu pomiędzy krokami pętli, obliczona na podstawie ich pierwszych dziesięciu. Klasa \texttt{Timer} posiada też funkcję zapisywania stempli czasowych, umożliwiając łatwe odmierzanie czasu pomiędzy pewnymi wydarzeniami.
	
	\texttt{ResourceManager} jest odpowiedzialny za zarządzanie zasobami symulacji. W~tym przypadku ich rolę pełnią tylko tekstury, shadery i~kernele. Kluczową kwestią tutaj jest działanie funkcji rodziny \texttt{Load}: \texttt{LoadTexture}, \texttt{LoadShader}, \texttt{LoadKernel}. Zasoby są trzymane w~przeznaczonych do tego kolekcjach. Podczas wywołania tej funkcji sprawdzona zostaje najpierw odpowiednia kolekcja na obecność żądanego zasobu. Jeżeli takowy istnieje, jest od razu zwracany. Jeśli go nie ma, dopiero wtedy rozpoczyna się proces załadowania go z pliku. Dzięki takiemu podejściu w~żadnym miejscu kodu nie trzeba martwić się o~to, czy wczytano zasób, czy jeszcze nie.
	
	Do obowiązków klasy \texttt{PhysicsManager} należy tak naprawdę tylko rozwiązywanie kolizji oraz przechowywanie wszelkich danych z~tym związanych, tj. klas kolizyjnych, tzw. \emph{colliderów}, pochodnych klasy \texttt{Collider}. Omawiany singleton zawiera także aktualny wektor grawitacji. W~celach optymalizacyjnych, kolizje nie są sprawdzane na zasadzie ,,każdy obiekt z~każdym'', ale tylko dla tych encji, które w~danej klatce zmieniły swoje położenie. Wyjątkiem jest sama tkanina -- dla niej kolizje rozwiązywane są w każdym kroku symulacji i~poza klasą \texttt{PhysicsManager}. Z~racji konieczności wykonywania tych obliczeń na GPU było to najwygodniejszym podejściem.
	
	Ostatnimi omawianymi singletonami są \texttt{InputManager} oraz \texttt{InputHandler}, który go opakowuje. W~ich gestii leży obsługa zdarzeń pochodzących z~urządzeń wejściowych, takich jak ekran dotykowy na platformie mobilnej. W~przypadku ,,pecetowej'' wersji programu, nie ma tu mowy o~jakichkolwiek zdarzeniach, a~zapewniony jest po prostu interfejs do odpytywania systemu operacyjnego o~wciśnięcie konkretnych klawiszy na konkretnych urządzeniach. W~kodzie samej logiki wirtualnego świata korzysta się jednak z~funkcji klasy \texttt{InputHandler}, zamieniającej ,,surowe'' dane o~stanie przycisków bądź ekranu na informacje o~możliwości wykonania akcji przez program.
	
	Dla jasności poniżej umieszczone zostały algorytmy \ref{alg_4_1}, \ref{alg_4_2} i \ref{alg_4_3} dotyczące uproszczonego działania całego programu:
	
	\begin{algorithm}
		\label{alg_4_1}
		\caption{Inicjalizacja silnika symulacji.}
			
				Inicjalizuj połączenie z systemem Android.
				
				\Indp
				
					Pobierz struktury danych z Androida.
					
					Ustaw funkcje obsługujących zdarzenia systemu.
					
					Uruchom kolejkę zdarzeń.
					
					Czekaj na sygnał od systemu, mówiący, że można inicjalizować resztę programu.
				
				\Indm
									
				Inicjalizuj Renderer.
				
				\Indp
				
					Inicjalizuj EGL i OpenGL.
					
					Wczytaj shadery.
					
					Ustaw zmienne OpenGL.
					
				\Indm	
				
				Inicjalizuj Menedżer zasobów.
				
				\Indp
					
					Wczytaj początkowo wymagane zasoby.
				
				\Indm	
				
				Inicjalizuj Menedżer interfejsu.
				
				Inicjalizuj Menedżer fizyki.
				
				Inicjalizuj Timer.
				
				\Indp
				
					Pobierz od systemu operacyjnego czas startu aplikacji.
				
				\Indm
				
				Inicjalizuj scenę.
				
				\Indp
				
					Utwórz obiekty sceny i ich komponenty.
					
					Utwórz obiekt tkaniny i komponent symulatora tkaniny.
					
					Utwórz kamerę.
					
					Utwórz światła.
					
					Utwórz interfejs użytkownika.
					
					Utwórz obiekt z komponentem zarządzającym interfejsem.
				
				\Indm			
	\end{algorithm}
	\newpage
	
	\begin{algorithm}
		\label{alg_4_2}
		\caption{Praca silnika symulacji.}	
		
		\While{true}
		{
			Wyciągnij i obsłuż wszystkie zdarzenia z kolejki.
			
			\If{m\_running}
			{
				Aktualizuj timer.
				
				Aktualizuj dane z urządzeń wejściowych.
				
				Aktualizuj encje sceny.
				
				\Indp
				
				Rozwiąż ewentualne kolizje między obiektami.
				
				Oblicz jeden krok symulacji tkaniny.
				
				\Indm
				
				Narysuj jedną klatkę wizualizacji symulacji.
			}
		}	
	\end{algorithm}
	
	\begin{algorithm}
		\label{alg_4_3}
		\caption{Uśpienie i wyłączenie silnika symulacji.}	
		
		Uśpienie:
		
		\Indp
		
			Zamknij encję sceny i ją samą.
			
			Zamknij wszystkich menedżerów.
			
			Ustaw zmienną logiczną Systemu \emph{m\_running} na \emph{false}.
		
		\Indm
		
		Zamknięcie:
		
		\Indp
		
			Zamknij encję sceny i ją samą.
			
			Zamknij wszystkie menedżery.
			
			Wyślij do systemu informację o tym, że następuje ostateczne zamknięcie.
		
		\Indm
		
	\end{algorithm}
	\newpage
	
	\section{Budowa i działanie silnika dla wizualizacji i zarządzania symulacją}
	\label{t:praktyka:silnik}
	
		\subsection{Encje systemu}
		\label{t:praktyka:silnik:komponent}
		
		% model komponentowy dla encji systemu
		% Scene, SimObject, Component, Transform, Collider
		% symulator tkaniny jako komponent
		
		\myownfigure{Architektura wirtualnej sceny.}{figures/pic_4_2.png}{0.5}{pic_4_2}
		
		Diagram \ref{pic_4_2} przedstawia ogólną architekturę bardzo ważnego elementu silnika, jakim jest wirtualna scena, realizowanego przez klasę \texttt{Scene}. Zgodnie z przyjętym założeniem projektowym, zajmuje się ona przechowywaniem i~obsługą wszystkich obiektów, z~których stworzono świat symulacji. 
		
		Aby można było cokolwiek zobaczyć, niezbędna jest wirtualna kamera, której funkcjonalność enkapsuluje klasa \texttt{Camera}. Określające ją dane to macierze widoku i~projekcji oraz wszelkie informacje potrzebne do ich wygenerowania. Zalicza się do nich wektory pozycji kamery, miejsca, w~które patrzy się kamera, oraz kierunku ,,do góry''. Ostatni wektor ma kluczowe znaczenie, jeżeli wystąpiłaby konieczność obrotu w~osi równoległej do kierunku kamery (tzn. różnica wektorów pozycji i celu). Oprócz tego są tu dostępne informacje dotyczące własności macierzy projekcji, czyli pionowy kąt widzenia (tzw. FOV -- \emph{Field of View}), format obrazu oraz położenie płaszczyzn odcinania. Iloczyn macierzy widoku i~projekcji jest niezbędny w~procesie rysowania obiektów.
		
		Scena przechowuje w~kolekcjach referencje do znajdujących się w~niej kamer, obiektów, elementów interfejsu oraz świateł kierunkowych, dzięki czemu możliwe jest stworzenie ich w~dowolnej liczbie. Da się łatwo ustawić, która kamera jest bieżąca, z~pozycji której wyrenderowany zostanie obraz. Światła to po prostu proste kontenery danych, przechowujące kolor, kolor rozbłysku czy kierunek padania. Dane te są przekazywane do shadera podczas rysowania sceny. Elementy interfejsu zostaną dogłębniej omówione w~podrozdziale \ref{t:praktyka:silnik:gui}.
		
		\myownfigure{Architektura przykładowej encji systemu.}{figures/pic_4_3.png}{0.34}{pic_4_3}
		
		Diagram \ref{pic_4_3} opisuje budowę klasy \texttt{SimObject}, której obiekty to główne elementy wirtualnego świata symulacji. Została ona zaprojektowana zgodnie z~architekturą komponentową -- każdy \texttt{SimObject} zawiera w~sobie kolekcję obiektów dziedziczących po klasie abstrakcyjnej \texttt{Component}. Wszystkie znajdujące się w~tej kolekcji komponenty są aktualizowane podczas aktualizacji \texttt{SimObjectu}. Wpływają one na zachowanie encji w~scenie i~określają jej rolę z~punktu widzenia całego systemu. Symulator tkaniny także jest komponentem (nosi nazwę \texttt{ClothSimulator}) i~może zostać przypisany do dowolnego \texttt{SimObjectu}. Prostszy przykład to napisany do celów testowych komponent \texttt{RotateMe}, który sprawia, iż obiekt posiadający go obraca się w~osi Y ze stałą prędkością. Klasa \texttt{Component} ma wśród swoich składników referencję do obiektu klasy \texttt{SimObject}, w którego skład komponentów wchodzi. Umożliwia to łatwy i szybki dostęp oraz modyfikację wszystkich ważnych elementów i parametrów tego \texttt{SimObjectu}.
		
		Inne elementy: \texttt{Transform}, \texttt{Mesh} i~\texttt{Collider} dziedziczą po \texttt{Component}, dzięki czemu realizują funkcje komponentu, lecz dla przejrzystości traktowane są przez \texttt{SimObject} jako osobne byty. Pierwszy z~komponentów jest kluczowy do określenia pozycji obiektu w scenie. Odpowiada on za przechowywanie i~generowanie tzw. macierzy świata, będącej złożeniem informacji o przemieszczeniu, obrocie i~skali. Komponent udostępnia także te parametry w formie trójelementowych wektorów, automatycznie aktualizując macierz świata w momencie ich zmiany z~zewnątrz. Przy próbie modyfikacji pozycji automatycznie weryfikowane jest, za sprawą \texttt{PhysicsManagera}, czy po takim przesunięciu obiekt nie będzie przenikał przez inne obiekty. Sprawdzone zostają kolizje ze wszystkimi pozostałymi encjami, w oparciu o~\texttt{Collidery} znajdujące się w osobnej kolekcji. \texttt{Collider} to klasa abstrakcyjna, a~każda z~jej implementacji stanowi inną strukturę okalającą, która musi umieć obsłużyć przecięcia z~pozostałymi dostępnymi. Na potrzeby symulacji zostały stworzone, omawiane w~Rozdziale \ref{t:teoria:analiza:kolizje:zewn} sfery okalające i~prostopadłościany AABB. Klasa \texttt{Mesh} i~jej podklasa \texttt{MeshGL} enkapsulują wszystkie właściwości dotyczące siatki geometrycznej obiektu, rysowanej na ekranie. Ta ostatnia także jest abstrakcyjna, a~jej oferowane implementacje pozwalają na wyrenderowanie: prostopadłościanu, sfery, najprostszej płaszczyzny składającej się z czterech wierzchołków, płaszczyzny o~dowolnej gęstości (używanej jako model tkaniny) oraz prostokąta w przestrzeni ekranu, na którym zostanie przedstawiony element interfejsu. Działanie będzie omówione szerzej w podrozdziale \ref{t:praktyka:silnik:render}.
		
		\subsection{Komunikacja z Androidem}
		\label{t:praktyka:silnik:andro}
		
		% incjalizacja
		% zdarzenia systemu (klasa System)
		% obrót ekranu
		% zdarzenia interfejsu (InputManager)
		
		Konieczność komunikacji z~systemem operacyjnym Android występuje zarówno podczas inicjalizacji programu, jak i~w trakcie jego działania. Spora część operacji jest zautomatyzowana przez kod z pakietu Cross-platform Development, jednak o~kluczowe czynności musi zadbać programista. Algorytm \ref{alg_4_1} przedstawia ogólnie ich ciąg. Na samym początku funkcja \texttt{main} otrzymuje jako argument wskaźnik do pewnego zestawu danych, opisanych strukturą \texttt{android\_app}. Jej zawartość składa się z~referencji do obiektów udostępnianych przez zautomatyzowany kod Javy oraz trzech ważnych wskaźników na funkcję. Do pierwszego z~nich przypisany zostaje adres metody \texttt{AHandleCmd} klasy \texttt{System}, pozwalając na wywoływanie jej przez system w~momencie, gdy zajdzie potrzeba obsługi zdarzenia związanego z~cyklem życia aplikacji. Do drugiego wskaźnika należy podpiąć metodę odpowiedzialną za obsługę zdarzeń dla urządzeń wejściowych i~jest to funkcja \texttt{AHandleInput} klasy \texttt{InputManager}. Ostatni wskaźnik ustawiono na adres funkcji \texttt{AHandleResize} klasy \texttt{Renderer}, wykonującej zmianę rozmiaru obszaru rysowania zależnie od zmiany wymiarów wyświetlacza, pojawiającej się przy obróceniu ekranu. Następnie uruchomiona zostaje kolejka zdarzeń, która później w~każdym kroku pętli głównej zostaje sprawdzona. Na koniec należy poczekać, aż Android przekaże aplikacji wymagane zasoby, m.in. kontekst renderingu, niezbędny do poprawnej inicjalizacji OpenGL. To kolejna niewygodna właściwość tej platformy.
		
		Charakterystyczna dla systemu Android jest konieczność szczegółowego dbania o~cykl życia aplikacji, czyli określanie co aplikacja ma zrobić, gdy zostanie przełączony kontekst, nastąpi uśpienie, wznowienie czy całkowite wyłączenie programu. O~tych wszystkich akcjach system powiadamia aplikację, korzystając ze zdarzeń, które w~każdym kroku działania są wyciągane z~kolejki i~obsługiwane przez wspomnianą wyżej funkcję \texttt{AHandleCmd}. Po uśpieniu program utrzymuje działającą tylko pętlę sprawdzającą kolejkę, żaden z~komponentów nie jest aktualizowany. Podczas wznawiania wszystkie inicjalizowane są od nowa. Uniemożliwia to zapis stanu aplikacji przy jakiejkolwiek wymuszonej przez użytkownika przerwie w~jej działaniu, ale zabezpiecza przed powstaniem w~ten sposób jakichkolwiek błędów symulacji.
		
		Jedną z~największych różnic pomiędzy Androidem a~platformą PC, np. Windows, jest sposób obsługi urządzeń wejścia. W~drugim przypadku udostępniane są zazwyczaj funkcje pozwalające w~każdym momencie ,,zapytać'' konkretne urządzenie o~stan jednego bądź wszystkich przycisków. W~przypadku smartfona, nie ma klawiatury ani myszy, a ekran dotykowy. Sposób uzyskiwania informacji o~tym, co się z~nim aktualnie dzieje, także jest inny. Funkcja \texttt{AHandleInput} zostanie wywołana za każdym razem, gdy będzie mieć miejsce określone przez system operacyjny zdarzenie dotyczące tzw. sensorów, czyli właśnie ekranu dotykowego, akcelerometru, a~nawet podpiętej do urządzenia klawiatury. Na poziomie tej metody dokonuje się filtrowania informacji tak, by reagować tylko na to, co jest interesujące z punktu widzenia aplikacji. Wykrywane i~udostępniane są przesunięcia palca, wciśnięcia i~puszczenia ekranu. Jako osobny gest obsługiwane jest także przesunięcie dwoma palcami naraz i~tzw. \emph{uszczypnięcie}.
		
		Ostatnią ważną na platformie mobilnej kwestią jest obsługa obrotu ekranu, jako że użytkownik może chcieć oglądać aplikację, trzymając telefon pionowo (tryb \emph{portrait}) lub poziomo (tryb \emph{landscape}). Za każdym razem, gdy orientacja wyświetlacza zostanie zmieniona, system operacyjny wysyła odpowiednie zdarzenie. Obsługuje je metoda \texttt{AHandleResize}. Jej działanie jest bardzo proste -- odczytane zostają nowe wymiary ekranu, a~następnie wywołana zostaje funkcja \texttt{glViewport}, zmieniająca rozmiary wewnętrznego okna, do którego OpenGL rysuje. Następnie zostaje przeliczona na nowo, z~aktualnymi parametrami, macierz projekcji kamery, aby uniknąć rozciągnięcia obrazu. Odpowiednio przeskalowane są także elementy interfejsu.
		
		\subsection{Rendering}
		\label{t:praktyka:silnik:render}
		
		% inicjalizacja renderera
		% pętla renderingu
		% klasa MeshGL i podklasy - inicjalizacja i metoda rysowania
		% obsługiwane shadery, tryby rysowania
		
		Aby przedstawić daną encję w~świecie symulacji, trzeba ją narysować na ekranie. Niezbędny w~tym celu jest model 3D i wszystkie związane z~nim dane potrzebne do wyrenderowania go. Klasa \texttt{SimObject} posiada kolekcję obiektów typu \texttt{Mesh}, będącego typem abstrakcyjnym. Jedyne informacje, jakie posiada, to bardzo podstawowe parametry wyglądu, takie jak kolor, współczynnik rozbłysku, czy identyfikator tekstury. Dopiero dziedzicząca po nim klasa \texttt{MeshGL} zapewnia niemalże całą OpenGL-ową implementację. Zastosowano takie rozwiązanie, aby umożliwić łatwą podmianę kodu korzystającego z~aktualnie używanej biblioteki graficznej, a~co za tym idzie -- łatwe przełączanie między tymi bibliotekami. Poniżej podano dokładny przebieg inicjalizacji (algorytm \ref{alg_4_4}) i~rysowania (algorytm \ref{alg_4_5}).
		\newline
		
		\begin{algorithm}[H]
			\label{alg_4_4}
			\caption{Inicjalizacja modelu}	
			
			Utwórz struktury przechowujące dane siatki.
			
			\emph{Wypełnij te struktury danymi.}
			
			Wygeneruj VAO (\emph{Vertex Array Object}) i wszystkie niezbędne bufory na GPU oraz wypełnij je danymi.
		\end{algorithm}
		
		\begin{algorithm}[H]
			\label{alg_4_5}
			\caption{Rysowanie modelu}	
			
			\While{m\_visible}
			{
			Pobierz i włącz aktualnie używany przez \texttt{Renderer} shader.
			
			Ustaw wszystkie parametry jednorodne (\emph{uniforms}) shadera, w tym macierze, pozycję oka, kierunki i~kolory świateł oraz parametry materiału obiektu.
			
			Przypisz teksturę do shadera.
			
			Włącz tablice atrybutów wierzchołków (pozycja, koordynat UV, wektor normalny, kolor, koordynat barycentryczny i~indeks).
			
			Narysuj siatkę przy pomocy funkcji \texttt{glDrawElements}.
			
			Wyłącz tablice atrybutów wierzchołków.
			}
			
		\end{algorithm}
		
		Etap \emph{Wypełnij te struktury danymi.} nie bez powodu został napisany kursywą. Klasa \texttt{MeshGL} jest bowiem w~dalszym ciągu klasą abstrakcyjną, a~funkcja odpowiedzialna właśnie za tę czynność to jej jedyna abstrakcyjna funkcja. Każda klasa dziedzicząca po \texttt{MeshGL} może we własny sposób wypełnić tablice wierzchołków oraz~ich parametrów, za każdym razem w~efekcie tworząc inną siatkę geometryczną. Takie podejście pozwala na proste i~wygodne oddzielenie kodu samego rysowania od kodu generującego konkretny model 3D.
		
		Sam rendering przebiega w~najprostszym trybie \emph{forward}, oznacza to, że wszystkie obiekty program rysuje prosto do bufora ramki, bądź tylnego bufora, w~kolejności zgodnej z~ich kolejnością występowania w~kolekcji sceny i~kolekcjach poszczególnych \texttt{SimObjectów}, używając testu bufora głębi do poprawnego umiejscowienia względem odległości od kamery. Rozwiązanie to ogranicza możliwość wprowadzenia dodatkowych efektów graficznych i~większej liczby świateł, jednak w~niniejszej symulacji nie są one potrzebne.
		
		Wśród zasobów programu znajdują się w~sumie trzy główne shadery, przy pomocy których aplikacja rysuje scenę. Pierwszy, podstawowy, renderuje obiekty korzystając z modelu oświetlenia Phonga - Blinna, z~obsługą rozbłysku specular. Obliczany jest on w~shaderze fragmentów, zapewniając dużo lepszą jakość obrazu, niż w~przypadku użycia do tego shadera wierzchołków, kosztem wydajności. Można sobie na to pozwolić, z~racji małej liczby elementów sceny. Cel działania drugiego shadera to pokazanie struktury siatki geometrycznej obiektów -- rysuje on tylko krawędzie pomiędzy wierzchołkami. W~szczególnych przypadkach zwiększa to znacznie czytelność obrazu i~można wykorzystać tę funkcję kiedy np. tkanina w~niepoprawny sposób się zawinie, a~użytkownik będzie chciał obejrzeć dokładnie, gdzie znajdują się wierzchołki w~tym miejscu i~jak przebiegają połączenia między nimi. Nie zostały tu użyte żadne modele oświetlenia, a~po prostu narysowany zostaje jednolity kolor, przeciwny do koloru wierzchołka, tzn. \(1 - k\), gdzie \(k\) -- kolor wierzchołka w zakresie \([0, 1]\). Trzeci shader łączy w~sobie efekty działania dwóch poprzednich.
		
		\subsection{Interfejs użytkownika}
		\label{t:praktyka:silnik:gui}
		
		% niezbędność GUI -- podejmowanie akcji, informowanie użytkownika
		% InputManager i InputHandler -- co udostępniają, jak są powiązane
		% elementy GUI i ich role w systemie, hierarchiczność, dopasowanie do ekranu
		% komponent GUIController -- kontrola ekranu dotykiem
		
		Niezbędnym do satysfakcjonującej wizualizacji symulacji tkanin i~możliwości interakcji z~nią użytkownika jest stworzenie odpowiedniego interfejsu graficznego, zwanego w~skrócie GUI (\emph{Graphical User Interface}). Dzięki niemu zaistnieje możliwość zarówno podejmowania pewnych akcji w~symulowanym świecie, bądź poza nim, jaki i~przedstawienie użytkownikowi pewnych informacji zwrotnych dotyczących głównie statystyk działania programu. GUI umożliwiło zarówno łatwe i~szybkie zebranie wyników testów w Rozdziale \ref{t:wyniki}, jak i~miłe dla oka zaprezentowanie symulacji.
		
		Urządzenia wejściowe obsługują klasy \texttt{InputManager} oraz \texttt{InputHandler}. Są one ze sobą nierozerwalnie związane. Pierwsza zapewnia niskopoziomową obsługę wszelkich zdarzeń pochodzących z~urządzeń wskazujących oraz wyciągnięcie z~nich informacji o~np. wciśniętych klawiszach, o~ile to możliwe. Dane, które zawarte są w~owych zdarzeniach zostają sformułowane i~udostępnione w postaci wygodnych elementów, tj. zmiennych logicznych, umożliwiających przykładowo sprawdzenie, czy aktualnie ekran dotykowy jest wciśnięty, oraz wektorów dwuwymiarowych odzwierciedlających pozycję wciśnięcia oraz kierunek przesunięcia palca, bądź palców po ekranie. Przeciągnięcie dwoma palcami \texttt{InputManager} także obsługuje, tak samo jak gest ,,uszczypnięcia'', czyli tzw. \emph{pinch}. Z~kolei \texttt{InputHandler} przekuwa informacje o~stanie urządzeń wejściowych na dane o~możliwości wykonania konkretnych akcji systemu. Przykładowo, jego funkcja \texttt{GetCameraMovementVector} odnosi się do metody \texttt{GetDoubleTouchDirection} \texttt{InputManagera}. Takie podejście pozwala nam na szybką zmianę sterowania systemem bez potrzeby przerabiania wszystkich zaangażowanych w~to komponentów, a~jedynie zmieniając implementację funkcji klasy \texttt{InputHandler}. 
		
		Program potrafi rysować dwuwymiarowe elementy GUI w~przestrzeni ekranu, takie jak: tekst, obrazki oraz dwustanowe przyciski.
		Tekst może być dynamicznie zmieniany w~każdym kroku pętli głównej. Klasą bazową przedstawiającą abstrakcyjny element interfejsu jest klasa \texttt{GUIElement}, skupiająca w~sobie funkcje wspólne dla każdego z wymienionych wyżej rodzajów. Dziedziczą po niej m.in. \texttt{GUIText}, \texttt{Picture} czy \texttt{GUIButton}. Składniki GUI mogą być łączone w~hierarchię, co pozwala na łatwe wyłączenie lub włączenie pewnej części interfejsu, np. tekstu oraz na optymalizację wykrywania kliknięcia -- pozycja palca przy naciśnięciu ekranu nie musi być sprawdzana dla wszystkich elementów. Za większość głównych akcji, jakie użytkownik może wykonać odpowiadają przyciski, dla których można zdefiniować osobne zbiory operacji zarówno przy zwykłym krótkim wciśnięciu, jak i~przytrzymaniu.
		
		Komponent \texttt{GUIController} jest ostatnim, acz bardzo ważnym ogniwem systemu interfejsu. Zamienia on sygnały wejściowe, udostępniane przez klasę \texttt{InputHandler} na konkretne działanie w systemie. Zawiera kod obsługujący ruchy kamerą, logikę przycisków i~pozostałych elementów GUI. Aktualizuje także informacje tekstowe na ekranie. 
		
		\myownfigure{Interfejs użytkownika aplikacji.}{figures/pic_4_4.png}{0.38}{pic_4_4}
		
		% opis jeszcze co może robić użytkownik przy pomocy interfejsu
		
		Projekt aplikacji zakłada, że użytkownik musi mieć możliwość zmiany położenia, obrotu i~przybliżenia kamery, resetowania symulacji i~modyfikacji jej parametrów, zmiany trybu wyświetlania obiektów oraz interakcji z~tkaniną na dwa sposoby -- przemieszczając obiekt wchodzący z~nią w kolizje lub przesuwając ją samą przy pomocy ruchów palca. Wymagane jest także informowanie użytkownika o~szybkości działania symulacji oraz o~tym, jakie ma ona aktualnie parametry i~jakiego jest typu.
		
		Rysunek \ref{pic_4_4} pokazuje, w~jaki sposób zostały zrealizowane te założenia. Tekst wyświetlany u~góry ekranu zawiera wszelkie dane, pozwalające użytkownikowi dowiedzieć się o wydajności symulacji i~wybranym jej modelu. Przycisk z~rysunkiem siatki trójkąta przełącza tryby wyświetlania, czego efekt widać na lewym dolnym obrazku -- rysowane są tylko krawędzie geometrii tkaniny. Przycisk w~kształcie rączki bądź kulki ze strzałkami zmienia sposób, w~jaki użytkownik dokonuje interakcji. Może on przemieszczać wybrany obiekt kolidujący przy pomocy umieszczonych w lewym dolnym rogu ekranu strzałek, bądź w~drugim trybie przesuwać tkaninę palcem -- strzałki wtedy znikają. Przycisk z~zębatką otwiera menu wyboru parametrów aplikacji, dokładniej opisanych w Rozdziale \ref{t:symulacja:dzialanie:parametry}, gdzie korzystając z ikonek plus i minus da się je zmieniać. Użytkownik ma możliwość opuszczenia aplikacji, używając przycisku ,,X'' w~prawym górnym rogu ekranu.

\chapter{Budowa i działanie symulatora tkaniny}
\label{t:symulacja}

	\section{Założenia projektowe}
	\label{t:symulacja:zalozenia}
	
	%cel działania, możliwości, użycie pamięci, szybciej a więcej pamięci, przystosowanie do GPU, tryby, uproszczony algorytm -- podział na etapy, różnice między implementacjami
	
	Całość funkcjonalności dotyczących symulacji tkaniny skupiono w klasie \texttt{ClothSimulator}. Dziedziczy ona po typie abstrakcyjnym \texttt{Component}. Za sprawą tego bardzo dobrze komponuje się z architekturą silnika, bez problemu możemy ją dodać do dowolnego \texttt{SimObjectu} oraz wielokrotnie powielić, a także zarządzać jej działaniem poprzez zmienną włączającą bądź wyłączającą.
	
	W prezentowanej aplikacji został umieszczony jednak tylko jeden \texttt{ClothSimulator}, aby uprościć działanie i ułatwić pomiary testowe. Za cel części praktycznej pracy postawiono sobie możliwość zasymulowania zachowania pojedynczej tkaniny o prostokątnym kształcie, zawieszonej sztywno w powietrzu za dwa sąsiednie narożniki, poddającej się działaniom sił grawitacji oporu powietrza i wchodzącej w interakcje z obiektami sceny oraz sygnałami od użytkownika, za pośrednictwem ekranu dotykowego smartfona. Symulacja miała udostępniać opcję bycia obliczaną dwoma metodami -- masy na sprężynie i bazującą na pozycji oraz trzema sposobami -- przy użyciu GPU, sekwencyjnie na CPU oraz współbieżnie na CPU, z wykorzystaniem czterech wątków roboczych. Użytkownikowi pozwolono na zmianę istotnych parametrów symulacji poprzez ich wybór z ustalonego zakresu. Wszystkie wymienione wyżej cele zostały zrealizowane. Przeznaczeniem symulatora jest jednak nie tylko wizualizacja, ale też i udostępnienie możliwości oceny modeli tkanin oraz ich implementacji pod kątem wydajności, dlatego program wyświetla informacje o czasie trwania pojedynczego kroku symulacji, a także o całościowym okresie jednego przebiegu pętli głównej programu. 
	
	\section{Wydajność a użycie pamięci}
	\label{t:symulacja:wydajnoscpamiec}
	
	Jako, że flagowym celem niniejszej pracy było zaimplementowanie symulacji z użyciem GPU, nadrzędne założenie przy projektowaniu to jak największe przyspieszenie przetwarzania kosztem większego użycia pamięci. Wszystkie możliwe dane i czynniki zostają obliczone podczas inicjalizacji symulatora, a wyniki są po prostu przesyłane do odpowiednich funkcji w trakcie działania programu. Wpasowuje się to doskonale w metodykę programowania GPU, do której się dostosowujemy, za sprawą chociażby minimalizacji ilości instrukcji warunkowych oraz uniknięcia obliczeń, które niepotrzebnie byłyby wykonywane dla każdego wierzchołka tkaniny, a mogą przecież zostać przetworzone tylko raz. 
	
	Przykładowo, sprawdzając sąsiadów wierzchołka, musimy obliczyć ich identyfikatory (tzn. poznać, które to konkretnie są wierzchołki) oraz zawsze mieć pewność, iż takowy istnieje -- nie wszystkie mają czterech sąsiadów, a dokładnie -- nie posiadają tylu te znajdujące się na zewnętrznych krawędziach siatki. Problem zostaje rozwiązany, gdy każdemu wierzchołkowi przypiszemy ustalone przy starcie komponentu zarówno listę identyfikatorów oraz mnożników, które wynoszą 1 gdy sąsiad istnieje, bądź 0 gdy go nie ma, i w tym przypadku obliczona siła bądź przesunięcie nie biorą udziału w dalszym przetwarzaniu. Eliminujemy także konieczność użycia instrukcji warunkowej, co dodatkowo poprawia nam wydajność. 
	
	Niestety, takie podejście zwiększa zużycie pamięci. Jak można się domyślić, jest ono proporcjonalne do ustalonej przez użytkownika gęstości siatki tkaniny. Ponadto, wszystkie wierzchołki wraz z ich parametrami są przechowywane dwukrotnie, z racji konieczności posiadania informacji o pozycjach poprzednich, dla całkowania Verleta (wzór (2.9), Rozdział \ref{t:teoria:analiza:masa}). W przypadku, gdy wybranym trybem symulacji jest tryb GPU, wszystkie te dane zostają dodatkowo skopiowane do pamięci karty graficznej. Dochodzą jeszcze do tego parametry pomocnicze, takie jak wymienione wyżej listy identyfikatorów sąsiadów. Rozważmy przykład, gdzie ustaliliśmy siatkę o \(m \times n\) dodatkowych krawędzi bocznych. Ilość wierzchołków będzie wynosić \( (m + 2)(n + 2) \). Z każdym z nich wiążą się następujące atrybuty:
	
	\begin{itemize}
		\item Pozycja (16 B),
		\item Koordynat teksturowania (8 B),
		\item Wektor normalny (16 B),
		\item Kolor (16 B),
		\item Koordynat barycentryczny (16 B),
		\item Indeks (4 B).
	\end{itemize}
	
	Wielkość niektórych elementów została sztucznie zwiększona tak, aby była wielokrotnością 4 B. Dzięki temu dane zostały poprawnie ułożone w buforach jednorodnych, omawianych w Rozdziale \ref{t:technologie:narzedzia:bufory}. W sumie jeden wierzchołek zajmuje 152 B pamięci -- bierzemy pod uwagę podwójne występowanie. Należy jednak pamiętać jeszcze o właściwych dla symulacji parametrach, opisanych w podrozdziale \ref{t:symulacja:dzialanie:parametry}, z których część jest przypisywana każdemu wierzchołkowi oddzielnie. Dodają one do naszych obliczeń kolejne 128 B. W sumie cała tkanina wraz z parametrami waży:
	
	\begin{equation}
	s = 280(m + 2)(n + 2) + 32 \ .
	\end{equation} 
	
	Ostatnia wartość wynika z konieczności przechowywania pojedynczego wektora początkowych odległości między danym a sąsiednimi wierzchołkami, takich samych dla każdego, gdyż rozważamy jednorodną prostokątną siatkę, oraz wektora przesunięcia palca użytkownika po ekranie dotykowym, kluczowego dla realizacji opisanej w podrozdziale \ref{t:symulacja:dzialanie:interakcja} interakcji. Przykładowo, dla gęstej, jak na warunki urządzenia mobilnego, siatki \( 98 \times 98 \) krawędzi, zajętość pamięci wynosi 2800032 B, czyli ok. 2,7 MiB. Jest to spora jak na jeden obiekt logiczny, lecz w żadnym wypadku nie krytyczna ilość dla testowego smartfona, wyposażonego w 2 GB RAM. Oczywiście w naszych obliczeniach nie bierzemy pod uwagę rozmaitych zmiennych pomocniczych dla symulatora tkanin, identyfikatorów buforów i struktur GPU oraz zapisanych wartości parametrów, jednak ich rozmiar jest tu pomijalnie mały.
	
	\section{Zasada działania}
	\label{t:symulacja:dzialanie}
	
		\subsection{Ogólny algorytm}
		\label{t:symulacja:dzialanie:algorytm}
		
		
	
		\subsection{Parametry symulacji}
		\label{t:symulacja:dzialanie:parametry}
			
		% lista parametrów z objaśnieniami
			
		\subsection{Obliczenia ruchu tkaniny}
		\label{t:symulacja:dzialanie:ruch}
			
		% różnice między MS i PB z punktu widzenia implementacji, algorytm dla MS, algorytm dla PB, ograniczenie buforów jednorodnych
			
		\subsection{Rozwiązywanie kolizji}
		\label{t:symulacja:dzialanie:kolizje}
			
		% algorytm, komunikacja z PhysicsManagerem -- struktury i pakowanie tego na GPU
			
		\subsection{Interakcja z użytkownikiem}
		\label{t:symulacja:dzialanie:interakcja}
			
		% po co, co możemy robić, jak to działa, wzory, algorytm
			
		\subsection{Przeliczenie wektorów normalnych}
		\label{t:symulacja:dzialanie:normalne}
			
		% po co, wzór, algorytm
	
	%\section{Budowa i działanie symulatora tkaniny na platformie Windows}	% - ni ma CUDY, będzie to samo	%\label{t:praktyka:symulacjapc}
	

\chapter{Wyniki testów symulatora}
\label{t:wyniki}

	\section{Czas wykonania}
	\label{t:wyniki:czas_wykonania}

		Czas wykonania jest rozumiany jako czas potrzebny na przetworzenie jednego pełnego kroku symulacji tkaniny. Wyrażony został w~milisekundach. To najważniejsze kryterium porównawcze, gdyż mówi nam, jak bardzo nasze obliczenia obciążają sprzęt, jak duży procent całości pracy silnika stanowią i~w efekcie -- czy działanie symulatora cechuje płynność. 
		
		Wpływ na czas wykonania ma ilość przetwarzanych danych, czyli gęstość siatki tkaniny, oraz oczywiście wybrana implementacja. Pierwszą zależność przedstawiono w~formie tabel oraz wykresów, osobno dla każdej metody i~implementacji. Liczbę wierzchołków można w~aplikacji łatwo modyfikować, zmieniając liczbę krawędzi poziomych i~pionowych. Przyjęto zakres od siatki posiadającej \(10 \times 10 \) wszystkich krawędzi (100 wierzchołków) do \( 120 \times 120 \) (14400 wierzchołków), z~krokiem co 10 krawędzi poziomych i~pionowych.
		
		Warto wspomnieć, że do zachowania pełnej płynności obrazu na ekranie należy rysować jedną jego klatkę przynajmniej 30 razy na sekundę. Oznacza to, iż czas wykonania symulacji nie może być większy niż ok. 33 ms. Najbardziej satysfakcjonującym wynikiem byłoby osiągnięcie go niższego niż ok. 16 ms, co równe jest 60 klatkom na sekundę -- to maksymalna szybkość renderingu przy włączonej synchronizacji pionowej obrazu. Założono oczywiście, że pozostałe obliczenia związane z~pracą silnika symulacji są pomijalnie krótkie.
		
		Przyjęto oznaczenia:
		
		\begin{enumerate}
			\item C -- liczba wszystkich wierzchołków,
			\item MS-GPU-A -- Model masy na sprężynie, implementacja GPU, platforma Android,
			\item PB-GPU-A -- Model oparty na pozycji, implementacja GPU, platforma Android,
			\item MS-GPU-W -- Model masy na sprężynie, implementacja GPU, platforma Windows,
			\item PB-GPU-W -- Model oparty na pozycji, implementacja GPU, platforma Windows,
			\item MS-CPU-A -- Model masy na sprężynie, implementacja CPU, platforma Android,
			\item PB-CPU-A -- Model oparty na pozycji, implementacja CPU, platforma Android,
			\item MS-CPUx4-A -- Model masy na sprężynie, implementacja CPU (4 wątki robocze), platforma Android,
			\item PB-CPUx4-A -- Model oparty na pozycji, implementacja CPU (4 wątki robocze), platforma Android.
			\newline
		\end{enumerate}
		
		\pgfplotstabletypeset{chart_6_1_a.dat}
		
		\pgfplotstabletypeset{chart_6_1_b.dat}
		
		\begin{tikzpicture}
			\begin{axis}[
			xlabel=C,
			ylabel=$t_{x}$,
			y SI prefix=milli,
			y unit=s,
			width=15cm,
			grid=major,
			legend style={at={(0.025, 0.8)}, anchor=west}
			]
			\addplot[orange, very thick] table [y=$t_{MS-GPU-A}$, x=C]{chart_6_1_a.dat};
			\addlegendentry{$t_{MS-GPU-A}$}
			\addplot[red, very thick] table [y=$t_{PB-GPU-A}$, x=C]{chart_6_1_a.dat};
			\addlegendentry{$t_{PB-GPU-A}$}
			\addplot[yellow, very thick] table [y=$t_{MS-GPU-W}$, x=C]{chart_6_1_a.dat};
			\addlegendentry{$t_{MS-GPU-W}$}
			\addplot[purple, very thick] table [y=$t_{PB-GPU-W}$, x=C]{chart_6_1_a.dat};
			\addlegendentry{$t_{PB-GPU-W}$}
			\addplot[blue, very thick] table [y=$t_{MS-CPU-A}$, x=C]{chart_6_1_b.dat};
			\addlegendentry{$t_{MS-CPU-A}$}
			\addplot[cyan, very thick] table [y=$t_{PB-CPU-A}$, x=C]{chart_6_1_b.dat};
			\addlegendentry{$t_{PB-CPU-A}$}
			\addplot[green, very thick] table [y=$t_{MS-CPUx4-A}$, x=C]{chart_6_1_b.dat};
			\addlegendentry{$t_{MS-CPUx4-A}$}
			\addplot[olive, very thick] table [y=$t_{PB-CPUx4-A}$, x=C]{chart_6_1_b.dat};
			\addlegendentry{$t_{PB-CPUx4-A}$}
			\end{axis}
		\end{tikzpicture}
		
		Wykres pokazuje dużą przewagę wydajnościową metod implementowanych na GPU. W~przypadku Androida, czas obliczeń jest niemalże stały, niezależnie od liczby wierzchołków tkaniny. Drobne wahania wynikają głównie z~błędu pomiaru (rzędu kilku ms). Niewielki wzrost czasu przetwarzania w~końcowej fazie testów może wynikać nie tyle z~samego narzutu obliczeniowego, ile z~rosnącej temperatury urządzenia i~związanego z tym stopniowego obniżania wydajności przez system operacyjny. 
		
		Niemożność uzyskania czasu obliczeń niższego niż ok. 12--15 ms wynika prawdopodobnie z~faktu wymuszenia synchronizacji pionowej przez implementację transformacyjnego sprzężenia zwrotnego na karcie graficznej Adreno. Jak można się było spodziewać, wersję GPU na platformie PC cechuje dużo większa wydajność. W~omawianym przypadku jest ona niemal 300-krotnie większa. Co ciekawe, problem z~synchronizacją pionową tu nie występuje, choć czas przetwarzania także utrzymuje się na stałym poziomie. 
		
		Osobną kwestią są implementacje na CPU. Można zauważyć, iż czas przetwarzania rośnie liniowo wraz z~liczbą wierzchołków i~bardzo szybko osiąga wartości, które uniemożliwiają generowanie płynnego obrazu. Jedynie dla niskiej gęstości siatki uzyskano przewagę nad GPU, z~racji wspomnianego wcześniej problemu. Widać także, że spadek wydajności dla implementacji z~użyciem 4 wątków roboczych jest ok. dwukrotnie mniejszy niż w~przypadku podejścia sekwencyjnego.
		
		W przypadku GPU nie zarejestrowano znaczących różnic czasu wykonania pomiędzy metodami symulacji, aczkolwiek na CPU model oparty na pozycji osiągał dla dużych liczby wierzchołków minimalnie lepsze wyniki niż jego rywal.
		
		%\subsection{Model masy na sprężynie -- GPU -- Android}
		%\label{t:wyniki:czas_wykonania:ms_gpu_andro}
		
		
		%\subsection{Model oparty na pozycji -- GPU -- Android}
		%\label{t:wyniki:czas_wykonania:pb_gpu_andro}
		
		
	%	\subsection{Model masy na sprężynie -- GPU -- Windows}
	%	\label{t:wyniki:czas_wykonania:ms_gpu_pc}
		
		
	%	\subsection{Model oparty na pozycji -- GPU -- Windows}
	%	\label{t:wyniki:czas_wykonania:pb_gpu_pc}
		
		
	%	\subsection{Model masy na sprężynie -- CPU -- Android}
	%	\label{t:wyniki:czas_wykonania:ms_cpu_andro}
		
		
	%	\subsection{Model oparty na pozycji -- CPU -- Android}
	%	\label{t:wyniki:czas_wykonania:pb_cpu_andro}
		
		
	%	\subsection{Model masy na sprężynie -- CPU (4 wątki) -- Android}
	%	\label{t:wyniki:czas_wykonania:ms_cpux4_andro}
		
		
	%	\subsection{Model oparty na pozycji -- CPU (4 wątki) -- Android}
	%	\label{t:wyniki:czas_wykonania:pb_cpux4_andro}
		
	
	\section{Stabilność}
	\label{t:wyniki:stabilnosc}
	
		Drugim najważniejszym problemem symulacji jest jej niestabilność, rozumiana jako skłonność do wpadania siatki tkaniny w~niekontrolowane drgania, co w~efekcie może prowadzić do ,,eksplozji''. Nawet jeśli się tak nie stanie, ciągłe ruchy układu skutkują nierealistycznym efektem wizualnym. Zjawisko to jest więc bardzo niepożądane i~często zmusza do uruchomienia symulatora od początku.
		
		Trudno określić, które dokładnie parametry mają wpływ na stabilność tkaniny. Z~pewnością najważniejszym z~nich jest sztywność -- większe siły sprężystości bądź większy udział ograniczników mogą prowadzić do powstawania anomalii w~procesie symulacji. Dla modelu masy na sprężynie znaczenie w~redukcji drgań ma także współczynnik ich tłumienia. Nie bez wpływu pozostają też takie zmienne jak gęstość siatki, masa czy siła grawitacji.
		
		Na potrzeby testów wybrano jeden z~położonych w środku tkaniny wierzchołków oraz zbadano jego drgania w stanie spoczynku, tj. średnią różnicę pomiędzy położeniem obecnym a~poprzednim, w~każdym kroku symulacji. Pomiarów dokonano dla różnych współczynników sztywności, a następnie przedstawiono tę zależność w~postaci tabel i~wykresów. Przy każdej metodzie zostały zbadane dwa przypadki, uwzględniające inne masy, siły grawitacji, współczynniki tłumienia oraz gęstości siatki. Stan spoczynku określono jako stan, w którym tkanina opadnie swobodnie z~pozycji poziomej do pionowej, zawieszonej w~dwóch punktach, i~przestanie się poruszać. Warto przypomnieć, że dla modelu opartego na pozycji parametr sztywności (\(s\)) został odpowiednio przeskalowany tak, by mieścił się w wymaganym zakresie [0, 1] i niósł ze sobą podobny efekt, co jego odpowiednik w~modelu masy na sprężynie. Platformą testową jest mobilna wersja aplikacji, z~implementacją na GPU.
		%\newline
		
		Pomiar pierwszy -- sztywność: [50, 600], krok 50; masa: 0.2 \(kg\); grawitacja: 1 \(\frac{m}{s^2}\); współczynnik tłumienia: -0.5; gęstość siatki: 625 wierzchołków.
		
		Pomiar drugi -- sztywność: [50, 600], krok 50; masa: 0.7 \(kg\); grawitacja: 2 \(\frac{m}{s^2}\); współczynnik tłumienia: -10; gęstość siatki: 6400 wierzchołków.
		%\newline
		
		Drgania (\(d_{x}\)) podano w mikrometrach. Ponadto przyjęto oznaczenia: \(A-n\), gdzie \(A\) -- rodzaj zastosowanego modelu symulacji (MS -- masy na sprężynie, PB -- oparty na pozycji), \(n\) -- numer pomiaru.
		\newline
		
		\pgfplotstabletypeset{chart_6_2.dat}
		
		\begin{tikzpicture}
			\begin{axis}[
			xlabel=s,
			ylabel=$d_{x}$,
			y SI prefix=micro,
			y unit=s,
			width=14cm,
			grid=major,
			legend style={at={(0.025, 0.9)}, anchor=west},
			every axis y label/.style={
				at={(-0.12, 0.5)}, rotate=90,
				anchor=east}
			]
			\addplot[orange, very thick] table [y=$d_{MS-01}$, x=s]{chart_6_2.dat};
			\addlegendentry{$d_{MS-01}$}
			\addplot[green, very thick] table [y=$d_{PB-01}$, x=s]{chart_6_2.dat};
			\addlegendentry{$d_{PB-01}$}
			\addplot[red, very thick] table [y=$d_{MS-02}$, x=s]{chart_6_2.dat};
			\addlegendentry{$d_{MS-02}$}
			\addplot[olive, very thick] table [y=$d_{PB-02}$, x=s]{chart_6_2.dat};
			\addlegendentry{$d_{PB-02}$}
			\end{axis}
		\end{tikzpicture}
		\pagebreak
		
		\myownfigure{Niestabilności występujące w obu modelach symulacji.}{figures/pic_6_1.png}{0.38}{pic_6_1}
		
		Główna różnica pomiędzy modelami masy na sprężynie i~opartym na pozycji ukazuje się właśnie tutaj. Widać, że w~pierwszym przypadku, dla pierwszej próby, z~początku zarejestrowano najniższą ze wszystkich oscylację drgań, jednak rośnie ona szybko wraz ze wzrostem parametru sztywności, dla najwyższej jego wartości doprowadzając nawet do ,,wybuchu'' symulacji. Jeśli chodzi o~drugie podejście, można zaobserwować duże oscylacje praktycznie niezależnie od elastyczności tkaniny, co pozwala wnioskować, iż zagęszczanie siatki także ma niebagatelny wpływ na drgania. Były one obecne praktycznie przez cały czas symulacji, widać je na rysunku \ref{pic_6_1} (po lewej). Ciągle poruszające się drobne zniekształcenia bardzo negatywnie wpływają na odbiór wizualny i~w~jakichkolwiek zastosowaniach praktycznych byłyby nie do zaakceptowania.
		
		Testy udowodniły, iż model oparty na pozycji cechuje wyjątkowa stabilność -- oscylacje są czasem nieznacznie większe niż u~rywala, jednakże w~obu próbach utrzymywały się na stałym poziomie, niezależnie od zwiększania parametru sztywności czy liczby wierzchołków. Drugi test ukazał jednak, że dla małej elastyczności i~gęstej siatki, tkanina zaczyna wchodzić w~niekontrolowane kolizje z samą sobą. Jest ona na tyle sztywna, by przy odpowiednim ułożeniu cząstek masy doprowadzić do ,,zawiśnięcia samej na sobie'' i~unieruchomieniu się w~powietrzu, de facto ignorując siłę grawitacji. Efekt ten można zaobserwować w~prawej części rysunku \ref{pic_6_1}. Takie zachowanie tkaniny także jest nie do zaakceptowania w warunkach praktycznych, jednak należy zaznaczyć, iż nie dochodzi tu do ,,wybuchu'' a~niestabilności nie mają charakteru drobnych, szybkich drgań, a raczej niekontrolowanego falowania. Spadek oscylacji w~przypadku współczynnika sztywności większego niż 450, w drugiej próbie, da się prawdopodobnie wytłumaczyć sytuacją widoczną na rysunku \ref{pic_6_1} (po prawej). Silne falowanie miało miejsce głównie w~części siatki oznaczonej czerwonym prostokątem, a~obszar środkowy, z~którego pobrana została próbka, pozostawał we względnym spoczynku.
		
	%	\subsection{Model masy na sprężynie}
	%	\label{t:wyniki:stabilnosc:ms}
		
		
	%	\subsection{Model oparty na pozycji}
	%	\label{t:wyniki:stabilnosc:pb}
		
		
	\section{Efekt wizualny}
	\label{t:wyniki:efektwiz}
		
		Ostatnim kryterium oceny to tzw. ,,efekt wizualny''. Przyjęta nazwa oznacza po prostu stopień, w jakim zachowanie i~wygląd symulowanej tkaniny odzwierciedla rzeczywistość. Wyznacznik ten jest całkowicie subiektywny, jednak na pewno można zauważyć wprost proporcjonalną zależność pomiędzy jakością a~gęstością siatki. Mała liczba wierzchołków fizycznie nie pozwala na wygenerowanie realistycznych zmarszczek ani zagięć, tak charakterystycznych elementów animacji tkanin. Dla każdego modelu symulacji zaprezentowane zostaną zrzuty ekranu, prezentujące ,,efekt wizualny'' z~różnymi liczbami krawędzi pionowych oraz poziomych, a~także innymi współczynnikami. Platformą testową jest mobilna wersja aplikacji.
		
		Na zrzutach ekranu wchodzących w~skład rysunku \ref{pic_6_2} przedstawiono wygląd tkaniny symulowanej modelem masy na sprężynie. Zgodnie z~ruchem wskazówek zegara, począwszy od lewego górnego obrazka przyjęto następujące parametry:
		
		\begin{enumerate}
			\item Siatka \(10 \times 10\) krawędzi, współczynnik sztywności 200, współczynnik tłumienia -0.5, przyspieszenie grawitacyjne 5 \( \frac{m}{s^2} \), masa 0.8 \(kg\);
			\item Siatka \(40 \times 40\) krawędzi, współczynnik sztywności 500, współczynnik tłumienia -3.3, przyspieszenie grawitacyjne 1 \( \frac{m}{s^2} \), masa 0.9 \(kg\);
			\item Siatka \(80 \times 80\) krawędzi, współczynnik sztywności 500, współczynnik tłumienia -3.3, przyspieszenie grawitacyjne 0.5 \( \frac{m}{s^2} \), masa 0.5 \(kg\);
			\item Siatka \(120 \times 120\) krawędzi, współczynnik sztywności 700, współczynnik tłumienia -10, przyspieszenie grawitacyjne 0.5 \( \frac{m}{s^2} \), masa 0.5 \(kg\).
		\end{enumerate}
		
		Natomiast na rysunku \ref{pic_6_3} pokazano efekt wizualny dla modelu opartego na pozycji:
		
		\begin{enumerate}
			\item Siatka \(10 \times 10\) krawędzi, współczynnik sztywności 50, przyspieszenie grawitacyjne 5 \( \frac{m}{s^2} \), masa 2 \(kg\);
			\item Siatka \(40 \times 40\) krawędzi, współczynnik sztywności 100, przyspieszenie grawitacyjne 5 \( \frac{m}{s^2} \), masa 0.1 \(kg\);
			\item Siatka \(80 \times 80\) krawędzi, współczynnik sztywności 200, przyspieszenie grawitacyjne 5 \( \frac{m}{s^2} \), masa 0.1 \(kg\);
			\item Siatka \(120 \times 120\) krawędzi, współczynnik sztywności 300, przyspieszenie grawitacyjne 0.5 \( \frac{m}{s^2} \), masa 0.1 \(kg\).
		\end{enumerate}
		
		% kolizje z boksem
			
		\myownfigure{Wygląd tkaniny dla różnych parametrów (model masy na sprężynie).}{figures/pic_6_2.png}{0.38}{pic_6_2}
		
		\myownfigure{Wygląd tkaniny dla różnych parametrów (model oparty na pozycji).}{figures/pic_6_3.png}{0.38}{pic_6_3}
		\pagebreak
		
		% podobieństwa między modelami, różnice między modelami, co się dzieje dla małej ilości krawędzi, co się dzieje dla dużej ilości krawędzi, kolizje z boksem, kolizje z samym sobą
		
		Jak może się wydawać z~pobieżnych oględzin zrzutów ekranu, rozbieżności w~wyglądzie tkaniny symulowanej różnymi modelami nie są duże. Faktem jest, iż dla małej liczby krawędzi tkanina wygląda i~zachowuje się niemal tak samo. Zmarszczki i~kształt ułożenia na obiekcie faktycznie wydają się wyglądać tym lepiej i~bardziej realistycznie, im gęstsza jest siatka, zarówno w pierwszej, jak i~w~drugiej metodzie. 
		
		Podobieństwa kończą się jednak w momencie wzięcia pod uwagę parametrów, jakich użyto do uzyskania pokrewnych efektów -- są zupełnie inne. Ogólnie rzecz biorąc, model oparty na pozycji generuje sztywniejszą tkaninę (czasem nawet -- jak pokazano wcześniej -- za bardzo) niż jego rywal, lecz nie występują tu mikrodrgania widoczne na prawym dolnym zrzucie ekranu rysunku \ref{pic_6_2}. Duże znaczenie ma także szybkość samej animacji tkaniny, powinna ona opadać i~reagować na interakcje z~poruszającymi się obiektami mniej więcej tak szybko, jak w rzeczywistości. Mimo swoich anomalii oraz trudności uzyskania sztywnego modelu, dla gęstszych siatek metoda masy na sprężynie daje w tej kwestii lepsze rezultaty. Z~drugiej strony metodę opartą na pozycji cechuje dużo łatwiejsza regulacja elastyczności i~większa stabilność, jednak mogą się pojawić tu problemy z~ustaleniem odpowiedniej szybkości animacji. Założono oczywiście stałość parametru \(\delta t \) przesyłanego do symulatora. W~obu metodach dobranie parametrów dla uzyskania pożądanej prędkości jest tym łatwiejsze, im mniej wierzchołków posiada siatka.
		
		W~przypadku małej liczby krawędzi można zaobserwować niedokładne wykrywanie kolizji pomiędzy tkaniną a~prostopadłościanem (rysunek \ref{pic_6_2}, oba górne zrzuty ekranu). Nie jest to jednak regułą, jako że problem pojawia się też dla gęstszych siatek (rysunek \ref{pic_6_3}, lewy dolny obrazek). Tutaj jednak winę prawdopodobnie ponosi także brak implementacji siły tarcia, co sprawia, że wierzchołki prześlizgują się po prostych ściankach obiektu, rozciągając tkaninę i~tworząc większe otwory w miejscu przebicia. Dla sfery okalającej, ze względu na jej obły kształt, problemy przebicia nie występują. Wyjątkiem są szybko poruszające się obiekty, które mogą zwyczajnie przeskoczyć przez tkaninę, w~jednym kroku obliczeń znajdując się przed nią, a~w~następnym -- już za. Należałoby tu zastosować ciągłą metodę wykrywania kolizji, dużo bardziej skomplikowaną matematycznie, lecz usuwającą takie zjawiska.
		
		\myownfigure{Wygląd tkaniny po przeniknięciu przez prostopadłościan (model oparty na pozycji).}{figures/pic_6_4.png}{0.38}{pic_6_4}
		
		Co gorsza, w~przypadku prostopadłościanu ześlizgiwanie się wierzchołków może stopniowo prowadzić także do kompletnego zsunięcia się tkaniny z obiektu kolizyjnego, poprzez powolne przenikanie przez niego kolejnych punktów masy. Efekt tego widać na 4ysunku \ref{pic_6_4}. Zdecydowanie sytuację poprawiłoby wprowadzenie siły tarcia bądź dokładniejszej metody detekcji kolizji, gdzie brane pod uwagę byłyby trójkąty siatki, a~nie tylko same wierzchołki.
		
		Zieloną ramką oznaczono fragment tkaniny, gdzie wystąpił błąd rozwiązywania kolizji wewnętrznych. Widać, że przeniknął on przez inną część modelu i~zawinął się w drugą stronę. Stało się tak z~powodu bardzo uproszczonej techniki sprawdzania tych kolizji, biorącej pod uwagę tylko cztery sąsiednie wierzchołki. Najprostszym rozwiązaniem problemu byłoby zwiększenie ich liczby, jednak wiąże się to z~coraz większym kosztem obliczeniowym. Wyjściem jest także zastosowanie innej metody detekcji, o~której mowa w~poprzednim akapicie.
			
	%	\subsection{Model masy na sprężynie}
	%	\label{t:wyniki:efektwiz:ms}
			
			
	%	\subsection{Model oparty na pozycji}
	%	\label{t:wyniki:efektwiz:pb}
\chapter{Podsumowanie i wnioski}
\label{t:wnioski}

	\section{Porównanie obu modeli symulacji}
	\label{t:wnioski:porownanie}
	
	% trudność pojęciowa -- ms łatwe, pb trudne
	% trudność implementacji -- bardzo podobna w obecnej formie ale PB elastyczniejsze, udostępnia więcej możl.
	% wydajność -- podobna, dużą część zajmuje wykrywanie kolizji
	% stabilność - PB wins, spada wraz ze wzrostem ilości krawędzi i elastyczności
	% ładność - PB lekko wins, można było bawić się różnymi paramterami dla różnych sprężyn, kolizje suck, szybkość i zabawa delta t
	% zastosowanie w praktyce -- MS dla prostych, PB dla każdych, konieczność lepszych kolizji
	
	Porównania omówionych w niniejszej pracy modeli symulacji tkanin należy dokonać z~uwzględnieniem wielu różnych czynników, tak aby na koniec móc jasno określić, który nadaje się lepiej do zastosowań praktycznych. 
	
	Aby móc zaimplementować którąkolwiek z~metod, programista musi wpierw dobrze zrozumieć jej działanie. Podstawy teoretyczne modelu masy na sprężynie są dużo prostsze, jako że oparto go o~łatwy do wyobrażenia system punktów masy połączonych sprężynami, których parametry, takie jak np. współczynnik sprężystości, wpływają na zachowanie tkaniny. Całość działa, wykorzystując znane z~podstaw fizyki prawo Hooke'a. Sprawia to, że nawet początkującemu programiście, nieobeznanemu z~meandrami matematyki, łatwo przyjdzie pojęcie i~przedstawienie sobie wizualnie tej metody. Z~kolei model oparty na pozycji wykorzystuje do swoich obliczeń dużo bardziej skomplikowane pojęcia i~może nastręczyć takiej osobie niemałych trudności.
	
	Sytuacja wygląda inaczej, jeśli weźmie się pod uwagę poziom trudności implementacji. Tworzenie omawianej aplikacji wykazało, że jest on niemal identyczny dla obu modeli symulacji. Każdy z~nich da się zaprogramować tak, aby korzystał z~tych samych danych oraz funkcji, różniąc się jedynie obliczeniami wykonywanymi przy konkretnych sprężynach bądź ogranicznikach. Obie techniki podczas swojego przetwarzania wykorzystują różnicę pomiędzy odległością spoczynkową a~aktualną między wierzchołkami. Niebotyczny wpływ na taki stan rzeczy miała na pewno decyzja o~nieużywaniu ograniczników zginania w metodzie opartej na pozycji. Zaimplementowanie wszystkich rozwiązań opisanych w~\cite{posbased} na pewno zmieniłoby stan rzeczy na korzyść modelu masy na sprężynie. Należy jednak zaznaczyć, że system ograniczników zapewnia dużo większy wachlarz zastosowań, umożliwiając wykorzystanie ich nie tylko do obliczeń ruchu, ale też i~np. rozwiązywania kolizji. Dołożono starań, by wyodrębnić jak największą część wspólną obliczeń dla obu modeli, z~chęci minimalizacji redundancji kodu i~dzięki temu do detekcji kolizji, reakcji na sygnał od użytkownika oraz przeliczenia normalnych używany są te same instrukcje.
	
	W~najważniejszej kwestii, czyli wydajności, okazuje się, iż obie metody także wykazują się podobnymi rezultatami, z~bardzo niewielkim zwycięstwem modelu opartego na pozycji w~implementacjach CPU. Nie można było dokładnie zbadać różnic w przypadku GPU, jako że objętość buforów jednorodnych nie pozwoliła na wygenerowanie tkaniny o~tak dużej liczbie wierzchołków, by czas wykonania wzbił się wyżej niż 20 ms. Biorąc pod uwagę duże podobieństwo samego kodu, należy przypuszczać, że także byłaby ona niewielka. Podczas testów zauważono, iż znaczną część czasu obliczeń zajmuje część algorytmu odpowiedzialna za rozwiązywanie kolizji. Może to wynikać z~faktu użycia instrukcji warunkowych w~kodzie wykonywanym na GPU. Większa liczba obiektów w~scenie wiązałaby się na pewno z~koniecznością głębszej optymalizacji tego zagadnienia, na przykład ograniczając liczbę potencjalnych encji mogących wejść w kolizje z~tkaniną jeszcze na poziomie CPU.
	
	Oba wykorzystane modele symulacji charakteryzuje pewna, zależna od parametrów, niestabilność, jednak jest ona dużo większa w~przypadku modelu masy na sprężynie. Niewątpliwe zaletą okazuje się fakt, iż w~skład wzoru, na podstawie którego obliczane są siły działające na wierzchołek wchodzi komponent odpowiedzialny za tłumienie drgań, a~użytkownik może go dowolnie regulować, w~ten sposób mając na nie wpływ. Tę metodę charakteryzuje wzrost oscylacji siatki wraz ze wzrostem współczynnika elastyczności. Mają one postać małych, acz szybkich wibracji na całej powierzchni tkaniny. Z~kolei dla dużych liczb krawędzi potrzeba dużej sztywności, aby zachować odpowiedni kształt, co jeszcze bardziej powiększa problem. Duże drgania z~pozoru sprawiają wrażenie, że utrzymują się na stałym poziomie, jednak przy jakiejkolwiek nagłej zmianie położenia wierzchołków, np. przy kolizji, mogą doprowadzić do nagłego ,,wybuchu'' symulacji, co jest w~praktyce niedopuszczalne. W~przypadku modelu masy na sprężynie także zaobserwowano zależność pomiędzy wzrostem gęstości siatki, elastyczności a~utratą stabilności. Jednakże ta metoda w każdym momencie i~tak jest stabilniejsza niż jej rywalka, co pokazały wyniki zaprezentowane w~rozdziale \ref{t:wyniki:stabilnosc}. Drgania są tutaj dużo wolniejsze i~mają postać delikatnego, niekontrolowanego falowania, co dużo mniej zwraca uwagę użytkownika. Dużym plusem jest brak występowania efektu ,,eksplozji'', niezależnie od ustawionych parametrów. Efekt ten udało się uzyskać poprzez pewnego rodzaju ,,trik'' implementacyjny -- pozycja wierzchołka przesyłana do funkcji obliczającej ogranicznik jest aktualizowana na bieżąco tylko w~ramach sąsiadów położonych w jednej linii. Minusem modelu opartego na pozycji w~niniejszej implementacji jest tendencja do blokowania się tkaniny samej na sobie, przy dużych współczynnikach elastyczności.
	
	Obie metody symulacji tkaniny generują pożądany efekt graficzny, czyli realistyczne zagięcia i~zmarszczki tkaniny oraz jej charakterystyczne ułożenie na obiekcie. Ich jakość jest minimalnie lepsza dla modelu opartego na pozycji, m.~in. z~racji nie występowania tam drobnych drgań oraz większej responsywności na zmiany współczynnika sztywności. Należy zauważyć, że np. na potrzeby gier w~wielu przypadkach nie potrzeba szczegółowo odwzorowanych detali tkanin, te można uzyskać przy pomocy map normalnych, a~po prostu przybliżonej symulacji ich ruchu. Dwie omawiane metody cechuje wierne odwzorowanie tego aspektu nawet dla małej liczby krawędzi. Z~kolei gdy wziąć pod uwagę gęste siatki, nasuwa się problem związany z~szybkością animacji. Duża liczba wierzchołków wymaga odpowiedniego współczynnika elastyczności, a~ten spowalnia przesuwanie tkaniny. Widać to szczególnie w~modelu opartym na pozycji. Rozwiązaniem mogłoby być bardziej dokładne dopasowanie współczynników bądź zwiększenie parametru \( \delta t \), niezmiennego w~omawianej symulacji. Poprawę sytuacji prawdopodobnie można też uzyskać poprzez ustawianie innych parametrów sztywności dla każdej z~grup sprężyn, bądź ograniczników (tj. równoległych do krawędzi tkaniny, leżących po przekątnej oraz takich jak pierwsze, lecz położonych o jedną pozycję dalej). Dużym minusem w~kwestii efektu wizualnego okazała się zastosowana metoda detekcji kolizji. Nie rozwiązuje ona w~satysfakcjonujący sposób kolizji wewnętrznych i~często występują błędy kolizji zewnętrznych, np. w~przypadku opadnięcia tkaniny na prostopadłościan. Aby naprawić problem, należałoby zaimplementować inną technikę. Jednak na pewno wiązałby się z~tym ubytek w wydajności i~tak najcięższego obliczeniowo komponentu symulacji.
	
	Biorąc pod uwagę wszystkie wymienione powyżej czynniki, można uznać iż do zastosowań praktycznych model oparty na pozycji nadaje się bardziej. Lepiej radzi on sobie z~dużymi liczbami wierzchołków, cechuje go względna stabilność i~jest minimalnie wydajniejszy. Niezależnie od pożądanej jakości wizualizacji, w~większości przypadków parametry da się ustalić tak, by uzyskać dobry efekt graficzny. Nie znaczy to jednak, że w~obecnej implementacji jest on wolny od wad, a różnice między metodami zanikają wraz ze spadkiem gęstości siatki -- wtedy i~model oparty na pozycji, i~masy na sprężynie cechują bardzo podobny wygląd oraz zachowanie.
	
	\section{Porównanie implementacji CPU i GPU smartfona}
	\label{t:wnioski:cpu_vs_gpu}
	
	% nieporównywalne zwycięstwo na GPU, idealne do tego problemu, stabilność wydajności, wada blokady fps, CPU tylko do bardzo prostych tkanin bądź słabych GPU na mobilnych, bądź gdy dysponuje się starym api, albo w 2d
	% implementacja CPU łatwiejsza, GPU narzuca ograniczenia i sposób wykonywania obliczeń
	
	Kolejnym aspektem niniejszej pracy było wykazanie wyższości zastosowania kart graficznych urządzeń mobilnych nad zastosowaniem procesorów tychże w~dziedzinie symulacji fizycznej tkanin. Testy zaprezentowane w rozdziale \ref{t:wyniki} jasno wskazują zwycięzcę tego porównania pod względem wydajności. GPU są wielokrotnie szybsze od CPU podczas obliczeń zagadnień mogących być przetwarzanymi równolegle, a~takim właśnie jest omawiany problem. Rozłożenie pracy nad wierzchołkami siatki tkaniny na poszczególne jednostki przetwarzające GPU to wręcz intuicyjne, a~zarazem skuteczne i~wydajne rozwiązanie pomimo występującej tu redundancji. Zarejestrowana szybkość działania okazuje się być taka sama, niezależnie od ilości danych, czego nie można powiedzieć o~implementacjach CPU, gdzie maleje ona liniowo. Podział na wątki robocze zwiększa ją dwukrotnie, co trochę poprawia sytuację, jednak w~przypadku szczegółowych tkanin i~tak wydajność jest za niska. Z~kolei na GPU zaobserwowano blokowanie liczby renderowanych klatek na sekundę do wartości zgodnej z odświeżaniem ekranu przez transformacyjne sprzężenie zwrotne. Jest to wada, nie pozwalająca w pełni ocenić wydajności i~blokująca prędkość działania aplikacji, jeśli nie dba się o~występowanie efektu ,,tearingu'', a~chce uzyskać jak najszybsze działanie. Problem być może rozwiązałaby zmiana testowego sprzętu na inny, bądź wykorzystanie innego API do obliczeń GPGPU, co niestety też oznacza konieczność wymiany.
	
	Wszystko to nie zmienia jednak faktu, iż implementacja CPU także ma swoje zastosowania i~zalety. Użycie jej jest konieczne w sytuacji, gdy urządzenie nie obsługuje wersji OpenGL ES 3.0, ani żadnej z specjalistycznych API, jak OpenCL. Możliwe, że uzyskałaby ona przewagę wydajnościową w~sytuacji, gdy platforma testowa dysponowałaby GPU z~bardzo niskiej półki. Da się ją także z~pewnością zastosować, gdy zbiorem danych jest jedynie bardzo niewielka liczba wierzchołków, bądź jeśli wybrano animację tylko w~przestrzeni 2D. Należy także zaznaczyć, że symulację tkanin dużo łatwiej zaimplementować na CPU, gdyż nie wymaga to głębszej znajomości API graficznego ani żadnych innych oraz tworzenia dość złożonego sterowania buforami, zmiennymi jednorodnymi, programami i~transformacyjnym sprzężeniem zwrotnym.
	
	\section{Porównanie implementacji GPU smartfona i GPU PC}
	\label{t:wnioski:andro_vs_pc}
	
	% olbrzymia różnica w wydajności i brak blokady FPS ale do tkanin występujących w małych ilościach bądź do prostych tkanin wielokrotnie powielanych GPU mobajla nadaje się bez problemu. Problemem jest też mniejsza ilość pamięci wideo i przegrzewanie się, oraz krótkość baterii.
	% implementacja PC łatwiejsza -- nowsze technologie, przystosowane do GPGPU, mobajle są kilka lat wstecz
	
	W rozdziale \ref{t:technologie} przewidywano, że wydajność urządzeń mobilnych w~omawianym zagadnieniu będzie wielokrotnie niższa, niż komputerów klasy PC. Stworzenie dwóch wersji aplikacji, jednej na platformę Android, drugiej -- na platformę Windows pozwoliło potwierdzić to przypuszczenie. Różnica szybkości działania jest ok. 300-krotna, w~dodatku na PC znika problem dotyczący blokady liczby renderowanych klatek na sekundę. Pozostaje więc na tej podstawie odpowiedzieć na fundamentalne pytanie: czy implementacja symulacji tkanin na urządzeniach mobilnych ma w~ogóle sens, jeśli do dyspozycji są dużo szybsze komputery stacjonarne? Okazuje się, że tak, aczkolwiek na trochę mniejszą skalę. O~ile smartfon z użyciem GPU może bezproblemowo animować tkaninę o naprawdę gęstej siatce, wystarczającej do odwzorowania większości szczegółów, to tę platformę dręczy kilka istotnych problemów. 
	
	Pierwszym jest wspominany już wielokrotnie brak buforów teksturowych na części urządzeń, co ogranicza maksymalną możliwą do uzyskania jakość. Pamiętać należy także o~ogólnie niższej ilości pamięci karty graficznej, wprowadzającej kolejne limity w~tej kwestii. Symulacja tkanin mimo wszystko bardzo mocno wykorzystuje tu możliwości sprzętowe, doprowadzając do przegrzewania się urządzenia. Prowadzi to do redukcji wydajności przez system operacyjny, co z~kolei skutkuje znacznym wydłużeniem przetwarzania i~miało wpływ na wyniki testów w~rozdziale \ref{t:wyniki:czas_wykonania}. Tworząc aplikacje, które cechuje znaczne zapotrzebowanie na moc obliczeniową należy o~tym fakcie pamiętać. To, że program początkowo działa bez zarzutu nie znaczy, iż za parę minut nie może zwolnić, gdy temperatura osiągnie wysokie wartości. Intensywna eksploatacja zasobów sprzętowych skutkuje także szybkim zużyciem baterii, co nie pozostaje bez znaczenia dla użytkownika końcowego i~powinno być istotne podczas projektowania aplikacji.
	
	Trzeba także wspomnieć o~różnicach w~API dostępnych na obu platformach. W przypadku PC, poczyniono ostatnimi czasy duże kroki w~celu udostępnienia programistom kart graficznych jako urządzeń do obliczeń ogólnego przeznaczenia. Jest tu dostępne wiodące w~tej dziedzinie API, czyli CUDA, cechujące się bardzo prostą obsługą i~dużą elastycznością. Ponadto także twórcy oprogramowania graficznego starali się udostępnić te funkcje, wprowadzając Compute Shadery. Implementacja symulacji tkanin z~ich użyciem byłaby zdecydowanie dużo łatwiejsza. Należy pamiętać, że wsparcie dla GPGPU na urządzeniach mobilnych nie jest tak zaawansowane, jak na PC i~w~efekcie programista musi się poruszać w oprogramowaniu starszej o~kilka lat generacji. Mimo to symulację można tu stworzyć, choć nastręcza to więcej trudności.
	
	%\section{Wpływ symulacji tkaniny na całość działania silnika}
	%\label{t:wnioski:wplyw}
	
	% Lepiej pisać aplikacje zorientowane na szczegółowe tkaniny, jako elementy tła mogą być niepotrzebnym obciążeniem, GPU - niewielki udział w procesie obliczeniowym lecz wraz ilością tkanin może on wzrastać, CPU - nie ma sensu
	
	
	
	%\section{Podsumowanie}
	%\label{t:wnioski:podsumowanie}
	
	% implementacja na urządzeniach mobilnych ma sens i da się wykorzystać do unikalnych zastosowań
	% wydajnościowo można generować szczegółowe tkaniny, ale tylko na GPU
	% implementacja GPU wymaga nowych api opengl bądź opencl / renderscript ale niekoniecznie telefonów z najwyższej półki
	% sensowną do użycia w praktyce wydaje się być metoda oparta na pozycji



\begin{thebibliography}{999}

\bibitem{cloth-dobre-wzory} Hanwen Li, Yi Wan, Guanghui Ma, \emph{A CPU-GPU Hybrid Computing Framework for Real-time Clothing Animation}, School of Information Science and Engineering, Lanzhou University, Lanzhou, China, 2011

\bibitem{posbased} Matthias Müller, Bruno Heidelberger, Marcus Hennix, John Ratcliff, \emph{Position Based Dynamics}, 3rd Workshop in Virtual Reality Interactions and Physical Simulation "VRIPHYS", 

\bibitem{cuda} Shane Cook, \emph{CUDA Programming: A Developer's Guide to Parallel Computing with GPUs}, Elsevier Inc., 2013

\bibitem{deformable} Louis Li-Fang Chang, Damon Shing-Min Liu, \emph{Deformable Object Simulation in Virtual Environment}, Association for Computing Machinery, Inc., 2006

\bibitem{tryon} Bart Kevelham, Nadia Magnenat-Thalmann, \emph{Virtual Try On: An application in need of GPU optimization}, ATIP/A\*CRC HPC Workshop, 2012

\bibitem{oglspec} Mark Segal, Kurt Akeley, \emph{The OpenGL Graphics System: A Specification (Version 4.5 (Core Profile) -- May 28, 2015)}, The Kronos Group Inc., 2015

\bibitem{receptury} Muhammad Mobeen Movania, \emph{OpenGL Receptury dla programisty}, Helion, 2015

\bibitem{wzory_sfera} \emph{Circle-Circle Collision Tutorial} \href{http://ericleong.me/research/circle-circle/}{http://ericleong.me/research/circle-circle/}, stan na dzień 04.01.2016

\bibitem{wzory_sfera_box} \emph{Static Sphere vs AABB} \href{http://blog.nuclex-games.com/tutorials/collision-detection/static-sphere-vs-aabb/}{http://blog.nuclex-games.com/tutorials/collision-detection/static\allowbreak-sphere-vs-aabb/}, stan na dzień 04.01.2016

\bibitem{gpu_wiki} \emph{Graphics processing unit}, \href{https://en.wikipedia.org/wiki/Graphics\_processing\_unit}{https://en.wikipedia.org/wiki/Graphics\_processing\_unit}, stan na dzień 04.01.2016

\bibitem{branch_prediction_wiki} \emph{Branch prediction}, \href{https://en.wikipedia.org/wiki/Branch\_predication}{https://en.wikipedia.org/wiki/Branch\_predication}, stan na dzień 05.04.2016

\bibitem{simd} \emph{SIMD} \href{https://en.wikipedia.org/wiki/SIMD}{https://en.wikipedia.org/wiki/SIMD}, stan na dzień 23.01.2016

\bibitem{flops} \emph{FLOPS} \href{https://en.wikipedia.org/wiki/FLOPS}{https://en.wikipedia.org/wiki/FLOPS}, stan na dzień 23.01.2016

\bibitem{geekbench} \emph{Geekbench 3 Results -- Geekbench Browser}, \href{http://browser.primatelabs.com/geekbench3/}{http://browser.primatelabs.com/geekbench3/}, stan na dzień 08.01.2016

\bibitem{versus} \emph{Qualcomm Adreno 320 vs Nvidia GeForce GTX 750}, \href{http://versus.com/en/qualcomm-adreno-320-vs-nvidia-geforce-gtx-750}{http://versus.com/en/\allowbreak{}qualcomm-adreno-320-vs-nvidia-geforce-gtx-750}, stan na dzień 08.01.2016

\bibitem{specs} \emph{LG Nexus 4 Technical Specifications}, \href{http://www.lg.com/uk/mobile-phones/lg-E960-nexus-4-by-lg}{http://www.lg.com/uk/mobile-phones/lg-E960-\allowbreak{}nexus-4-by-lg}, stan na dzień 08.01.2016

\bibitem{specs_adreno} \emph{Adreno}, \href{https://en.wikipedia.org/wiki/Adreno}{https://en.wikipedia.org/wiki/Adreno}, stan na dzień 08.01.2016

\bibitem{specs_gtx750} \emph{GeForce 700 Series}, \href{https://en.wikipedia.org/wiki/GeForce_700_series}{https://en.wikipedia.org/wiki/GeForce\_700\_series}, stan na dzień 08.01.2016

\bibitem{specs_gtxtitan} \emph{GeForce 900 Series}, \href{https://en.wikipedia.org/wiki/GeForce_900_series}{https://en.wikipedia.org/wiki/GeForce\_900\_series}, stan na dzień 08.01.2016

\bibitem{opengl_wiki} \emph{OpenGL Wiki}, \href{https://www.opengl.org/wiki/}{https://www.opengl.org/wiki/}, stan na dzień 10.01.2016

\bibitem{buffers} \emph{Uniform Buffers VS Texture Buffers}, \href{http://rastergrid.com/blog/2010/01/uniform-buffers-vs-texture-buffers/}{http://rastergrid.com/blog/2010/01/uniform-buffers\allowbreak{}-vs-texture-buffers/}, stan na dzień 15.01.2016

\end{thebibliography}


\chapter*{Abstract}

% what exactly is cloth simulation
% why was it created and where is it used
% usage of cloth simulation on mobile devices
% main purposes of this paper
% what is to be done
% every chapter's summary

The realistic simulation of cloths is nowadays a~key to produce good-quality, authentic graphical visualizations of various fabrics, such as characters' garment elements, flags or curtains. This can be computationally expensive, more and more as number of particles, which fabric is divided into, increases. The solution to this matter was to use GPU -- \emph{Graphic Processing Unit} and perform all calculations on this device. On PC platform, this technique proved to be much faster than the standard CPU approach.

The main purpose of this work is to check whether this solution could also be introduced on the mobile devices. Most of them nowadays also have their own specialized GPU chips, but will they prove to be computationally faster than mobile CPUs? Is it possible and worth one's while to create visually appealing and efficient cloth simulation here? And how big is the difference between PC and mobile platform in GPGPU performance? This paper answers these questions.

A~test application was created, one of its main purposes being the visualisation of two selected simulation methods -- mass--spring and position--based model. It was equally important to show cloth's collisions between other objects in scene and itself. The user is allowed to set various parameters that influence the simulation, such as the aforementioned method type, mesh density and dimensions or elasticity coefficient. He can also impact the movement of the cloth, swiping his finger along the device's touch screen, which is something unique to the mobile platform. To fully measure every important factor of the simulation, its three implementations were created -- one using GPU for computing and the other two using GPU, in sequential and multi-threaded approach. To have a~comparison between mobile and PC platform, a~PC version of the application was created, both similar and sharing as much code with each other as possible.

In chapter 2 of this paper there are described theoretical basis of incorporated simulation methods and collision resolving. General Purpose GPU Computing is also mentioned, along with GPU framework and a~comparison between it and a~CPU is made, in the matter of architecture and performance. Chapter 3 analyses abilities of mobile devices, also mentioning their unique UI capabilities after comparing test device -- LG Nexus 4 -- to an~example PC. All used APIs, libraries and most important functions are also described. Among them are OpenGL ES 3.0, OpenGL 3.3 and Android NDK, as test application is completely written in C++. Chapter 4 shows architecture of~its engine, magnifying characteristics of~the most important components. Algorithms of~the engine's operation are also given, including such actions as~updating scene's entities, communication with Android OS, rendering or UI. In Chapter 5 attention is turned to the cloth simulator only, describing its general work flow, divided into three stages -- particle movement computation, collision solving and recalculation of normal vectors. Every stage is then thoroughly described, with their most important fragments's code given. Finally, in chapters 6 and 7, results of test application's work are shown, in the matter of both simulation models' computation time, stability and visual appeal, considering all three implementations and two tested platforms. In the end it is proved that the cloth simulation can be implemented on mobile devices and the average one's GPU can perform very well, producing smooth animation of fabric' dense mesh, but not without a few important limitations.

\listoffigures

\listoftables

\newpage
\thispagestyle{empty}
\begin{textblock}{1}(-2.65,-1.65)
\includegraphics{figures/oswiadczenie_o_samodzielnosci.pdf}
\end{textblock}

\end{document}
