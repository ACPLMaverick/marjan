\chapter{Teoria symulacji tkanin i obliczeń na GPU}
\label{t:teoria}


	\section{Analiza istniejących modeli symulacji tkanin}
	\label{t:teoria:analiza}
	
		\subsection{Model masy na sprężynie}
		\label{t:teoria:analiza:masa}		
			
			Pierwszym z rozważanych w niniejszej pracy modeli symulacji tkanin jest model masy na sprężynie. Wiadomo, że rysowana przez API graficzne tkanina jest w postaci siatki wielokątowej. Siatka taka składa się z punktów w przestrzeni 3D -- wierzchołków. Na potrzeby symulacji przyjmujemy, że każdy z tych wierzchołków ma określoną masę i poddajemy go działaniu sił, w wyniku których następuje przemieszczenie. Aby zachować kształt i odpowiednie dla tkaniny zachowanie się siatki, wierzchołki połączone są sprężynami o określonych współczynnikach sprężystości oraz tłumienia drgań. 
			
			
			\myfigure{Rysunek 2.1}{figures/pic_2_1.png}{0.4}{Schemat modelu masy na sprężynie}
			
			
			Rysunek 2.1 przedstawia przykładowy fragment tkaniny opartej o model masy na sprężynie. Zakładamy dla uproszczenia, iż poszczególne wierzchołki tworzą kształt prostokątów, aczkolwiek w praktyce mogą przyjmować dowolne ustawienia. Ważne jednak dla zachowania poprawnej symulacji jest to, by punkty masy były równomiernie rozłożone w całej powierzchni tkaniny. W omawianym przypadku tak właśnie jest. 
			
			Możemy zaobserwować trzy rodzaje sprężyn, jakie występują w modelu na Rysunku 2.1. Kolorem czerwonym zostały oznaczone sprężyny strukturalne, które służą do utrzymania ogólnego kształtu tkaniny. Jednakże one same nie są w stanie zasymulować zachowania tkaniny w poprawny i miły dla oka sposób. Kolorem zielonym narysowane zostały sprężyny odpowiedzialne za wierne oddanie zgięć tkaniny, położone są one wzdłuż diagonalnych krawędzi siatki. Kolorem niebieskim zaznaczono sprężyny, których obecność zapewnia tkaninie odpowiednią elastyczność i chroni przed nadmiernym jej rozciąganiem. Łączą one nie sąsiednie wierzchołki, lecz następne, za sąsiadem, w tym samym kierunku. Każdy z rodzajów sprężyn może być opisany innymi współczynnikami sprężystości i tłumienia drgań, co pozwala na uzyskanie innych zachowań tkaniny w symulacji.
			
			
			\myfigure{Rysunek 2.2}{figures/pic_2_2.png}{0.3}{}
			
			
			Na Rysunku 2.2 widzimy, jakie siły oddziałują na każdy punkt masy -- wierzchołek siatki tkaniny. Wszystkie równania oprócz (2.4) zostały zaczerpnięte z \cite{cloth-dobre-wzory}. Siły możemy zaklasyfikować jako wewnętrzne i zewnętrzne. Z zewnętrznych wyróżniamy siłę grawitacji, opisaną wzorem:
			
			\begin{equation}
			F_{g} = m_{i} \cdot g \ .
			\end{equation}
			
			Kolejna siła zewnętrzna to siła oporu powietrza. Zgodnie z Prawem Stokesa jest ona proporcjonalna do prędkości punktu masy oraz pewnego współczynnika oporu \emph{k}:
			
			\begin{equation}
			F_{a} = -k_{a} \cdot v_{i} \ .
			\end{equation}
			
			Siłami wewnętrznymi działającymi na wierzchołki tkaniny są oczywiście siły sprężystości, wynikające z istnienia omówionych wyżej sprężyn. Do wyznaczenia jej wartości wykorzystujemy Prawo Hooke'a, mówiące, że siła sprężystości oraz jej kierunek i zwrot są proporcjonalne do wychylenia sprężyny, tj. różnicy odległości między jej aktualną długością, a długością w stanie spoczynku:
			
			\begin{equation}
			F_{se} = - \sum_{n = 0}^{n < 12} k_{s} (|x_{i} - x_{j}| - l_{(i, j)}) \cdot \frac{x_{i} - x_{j}}{|x_{i} - x_{j}|} \ ,
			\end{equation}
			
			gdzie \(k_{s}\) -- współczynnik sprężystości, \(x_{i}\) oraz \(x_{j}\) -- położenia wierzchołków połączonych daną sprężyną, \(l_{(i, j)}\) -- odległość między tymi punktami w stanie spoczynku.
			
			Zgodnie z \cite{receptury}, wprowadzamy także siłę tłumienia drgań sprężystych, aby zminimalizować niepotrzebne, nierealistyczne drgania oraz ryzyko wymknięcia się symulacji spod kontroli ("wybuchnięcia" - obliczane siły są takie, że pozycje wierzchołków dążą do nieskończoności). Przedstawia się ona następującym wzorem:
			
			\begin{equation}
			F_{sd} = \sum_{n = 0}^{n < 12} k_{d} (\frac{|x_{i} - x_{j}| \cdot |v_{i} - v_{j}|}{l_{(i, j)}}) \ ,
			\end{equation}
			
			gdzie oznaczenia są takie same, jak wyżej z tym, że \(k_{d}\) jest w tym przypadku współczynnikiem tłumienia drgań, \(|v_{i} - v_{j}|\) oznacza różnicę prędkości obu wierzchołków, a działanie \(|x_{i} - x_{j}| \cdot |v_{i} - v_{j}|\) to iloczyn skalarny wektora różnicy położenia i wektora różnicy prędkości. Końcowa siła sprężystości jest sumą dwóch powyższych wzorów i możemy zapisać ją w postaci:
			
			\begin{equation}
			F_{s} = F_{se} + F_{sd} \ ,
			\end{equation}
			\begin{equation}
			F_{s} = \sum_{n = 0}^{n < 12} - k_{s} (|x_{i} - x_{j}| - l_{(i, j)}) \cdot \frac{x_{i} - x_{j}}{|x_{i} - x_{j}|} + k_{d}(\frac{|x_{i} - x_{j}| \cdot |v_{i} - v_{j}|}{l_{(i, j)}}) \ .
			\end{equation}
			
			W sumie dla każdego wierzchołka rozważamy 12 sił sprężystości dla wszystkich przyłączonych do niego sprężyn, siłę grawitacji oraz oporu powietrza. Należy zauważyć, że nie uwzględniamy tutaj żadnych sił związanych z reakcją na kolizje -- będą one rozwiązane w późniejszej sekcji algorytmu. Wzór na siłę wypadkową działającą na pojedynczy wierzchołek tkaniny oraz wypadkowe przyspieszenie można zapisać w postaci:
			
			\begin{equation}
			F = F{s} + F{g} + F{d}		
			\end{equation}
			
			\begin{equation}
			a(t) = \frac{F}{m_{i}}	
			\end{equation}
			
			W celu wyznaczenia zmiany położenia punktu masy w danym kroku symulacji, posługujemy się, zgodnie z \cite{cloth-dobre-wzory} i \cite{receptury}, całkowaniem Verleta, będącym jedną z technik całkowania numerycznego nie wprost. Opisane jest ono wzorem:
			
			\begin{equation}
			x(t + \delta t) = 2x(t) - x(t - \delta t) + a(t) \delta t^{2} \ ,		
			\end{equation}
			
			gdzie \(a(t)\) jest przyspieszeniem, a \(x(t + \delta t)\), \(x(t)\) i \(x(t - \delta t)\) oznaczają położenia wierzchołka w następnym, obecnym oraz poprzednim kroku symulacji. 
			
			Całkowanie to jest równie proste obliczeniowo i implementacjyjnie, jak najzwyklejsza metoda Eulera, a zapewnia dużo stabilniejsze zachowanie się symulacji i minimalizację tendencji do "wybuchnięcia". Niejawnie obliczona zostaje tutaj aktualna prędkość wierzchołka, co sprawia, że do symulatora musimy dostarczyć nie tylko obecną pozycję każdego punktu masy, ale także położenie poprzednie. Zwiększa to koszt pamięciowy symulacji względem innych technik całkowania, lecz zapewnia bardzo szybkie obliczenia i stabilne ich wyniki.
		
		\subsection{Model oparty na pozycji}
		\label{t:teoria:analiza:poz}
		
			Model oparty na pozycji i model masy na sprężynie mają pewną część wspólną - jest to obliczanie przesunięć wynikających z sił grawitacji oraz oporu powietrza za pomocą całkowania Verleta. Podejście do symulacji ruchów wynikających ze struktury samej tkaniny jest jednak kompletnie odmienne. Oczywiście tkanina ma w tym modelu taką samą postać, jak i w poprzednim - zbiór wierzchołków, połączonych krawędziami, tworzących prostokątne kształty. Ponadto zaznaczyć należy, że w niniejszej pracy została wykorzystana tylko część aktualnie omawianego modelu, bardziej szczegółowo opisanego w \cite{posbased}. Prezentowane tu rozwiązanie nie bierze pod uwagę ograniczników zginania oraz całego systemu detekcji kolizji, zaproponowanego w rozdziale 4 \cite{posbased}.
			
			Przesunięcia wynikające z sił zewnętrznych działających na układ nazywamy tutaj przesunięciami przewidywanymi. Każdy wierzchołek tkaniny opisywany jest, oprócz, masy, pozycji i prędkości, także przez zbiór tzw. ograniczników. Każdy z nich definiowany jest poprzez pewną funkcję \(C_{j} : R^{3n_{j}} \rightarrow R\), zestaw indeksów \(\{ i_{1}, \dots, i_{n_{j}}  \}, i_{k} \in [1, \dots, N] \) i parametr sztywności \(k \in [0\dots1] \). Ogranicznik może być typu równości, co oznacza, że jego ograniczenie jest spełnione, kiedy \( C_{j}(x_{i_{1}}, \dots, x_{i_{n_{j}}} ) = 0 \). Może być także typu nierówności, i spełnia je warunek \( C_{j}(x_{i_{1}}, \dots, x_{i_{n_{j}}} ) \geq 0 \). W naszym przypadku będziemy rozważać tylko ograniczniki pierwszego typu. Kluczowym elementem jest oczywiście funkcja \(C_{j}\), która określa sposób, w jaki przewidywana pozycja zostanie poprawiona, czyli w efekcie zachowanie się tkaniny. Funkcja ta może być zupełnie inna, gdy rozważać będziemy ograniczenia przesuwania, a inna przy ograniczeniach zginania bądź kolizjach. Widać, że dzięki temu model ten charakteryzuje się pewną elastycznością, a ograniczniki są narzędziem, przy pomocy którego możemy realizować wiele rozmaitych założeń dotyczących ruchu tkaniny. Przesunięcia wynikające z nałożonych ograniczników są obliczane po kolei, następnie pozycja przewidywana jest poprawiana do takiej, która spełnia wszystkie warunki, równości bądź nierówności, dla funkcji \(C_{j}\) ograniczników. Proces ten autorzy \cite{posbased} nazywają projekcją. Pod uwagę brana jest także sztywność ogranicznika, czyli procent, w jakim się go stosuje.
			
			\myfigure{Rysunek 2.2}{figures/pic_2_3.png}{0.3}{}
			
			Podstawowym typem ogranicznika jest ogranicznik rozciągania. To on definiuje ogólny kształt i odpowiednie zachowanie tkaniny. Jego funkcja ma postać:
			
			\begin{equation}
			C(p_{1}, p_{2}) = |p_{1} - p_{2}| - d \ .		
			\end{equation}
			
			Gdzie \( p_{1} \) i \( p_{2} \) są pozycjami rozpatrywanych wierzchołków, a \(d\) - początkową odległością między nimi. Na Rysunku 2.3 widzimy efekt projekcji. W [3] wyprowadzone zostają wzory na jej obliczenie, na podstawie funkcji \( C_{j}(x_{i_{1}}, \dots, x_{i_{n_{j}}} ) \):
			
			\begin{equation}
			s = \frac{C_{j}(p_{i_{1}}, \dots, p_{i_{n_{j}}} )}{ \sum _{j} w_{j} | \nabla _{p_{j}} C_{j}(p_{i_{1}}, \dots, p_{i_{n_{j}}} ) | ^{2} } \ ,		
			\end{equation}
			
			\begin{equation}
			\delta p_{i} = -sw_{i} \nabla _{p_{i}} C_{j}(p_{i_{1}}, \dots, p_{i_{n_{j}}} ) \ .
			\end{equation}
			
			Gdzie \(w_{i}\) jest odwrotnością masy wierzchołka tkaniny. Biorąc pod uwagę, iż przemieszczenie jest do niej wprost proporcjonalne, możemy łatwo wyobrazić sobie dowód na poprawność tego podejścia - w przypadku, gdy masa cząstki jest nieskończona, przesunięcie będzie równe zeru. Kiedy na miejsce funkcji \(C_{j}(x_{i_{1}}, \dots, x_{i_{n_{j}}} ) \) wstawimy \(C(p_{1}, p_{2}) = |p_{1} - p_{2}| - d\), otrzymamy po przekształceniach: 
			
			\begin{equation}
			\delta p_{1} = - \frac{w_{1}}{w_{1} + w{2}} (|p_{1} - p_{2}| - d) \frac{p_{1} - p_{2}}{|p_{1} - p_{2}|} \ ,
			\end{equation}
			
			\begin{equation}
			\delta p_{2} = \frac{w_{2}}{w_{1} + w{2}} (|p_{1} - p_{2}| - d) \frac{p_{1} - p_{2}}{|p_{1} - p_{2}|} \ .
			\end{equation}
			
			Możemy zauważyć, że wzory (2.13) i (2.14) są bardzo podobne do wzoru (2.3). Tak samo, jak w modelu masy na sprężynie, "siła" ogranicznika zależy od różnicy aktualnej odległości pomiędzy punktami masy i odległości spoczynkowej. Rolę współczynnika sprężystości pełni tutaj parametr sztywności, przez który mnożymy na koniec przesunięcie będące wynikiem projekcji. Dla \(k\) równego 0 ogranicznik nie będzie w ogóle brany pod uwagę, a dla równego 1 -- punkt nigdy nie zmieni swojej początkowej pozycji.
			
			W rozważanym przypadku nie stosujemy ograniczników zginania, używamy także innej metody detekcji kolizji. Autorzy \cite{posbased} używają ograniczników rozciągania biorąc pod uwagę tylko wierzchołki leżące w sąsiedztwie danego punktu. Okazuje się, że podobny do użycia ograniczników zginania efekt osiągamy zwiększając zbiór rozpatrywanych wierzchołków o te leżące jedną pozycję w siatce dalej - tak jak sprężyny oznaczone kolorem niebieskim na Rysunku 2.1. Przykładowo, ograniczając elastyczność przemieszczania wierzchołka \(A\) względem \(C\) ograniczamy przecież tak naprawdę możliwość zginania się trójkątów \(ABD\) i \(BCD\) na wspólnej krawędzi \(BD\). Nie jest to tak dokładna metoda jak ograniczniki zginania, gdzie regulujemy kąt pomiędzy trójkątami, lecz mimo to daje ona poprawny wizualny efekt, jak pokazane zostanie w Rozdziale 5. Jest też szybsza do obliczenia przez procesor, jako że sam wzór jest prostszy.
		
		\subsection{Porównanie powyższych metod}
		\label{t:teoria:analiza:porownanie}
		
		\subsection{Detekcja kolizji}
		\label{t:teoria:analiza:kolizje}
		
			\subsubsection{Kolizje zewnętrzne}
			\label{t:teoria:analiza:kolizje:zewn}
			
			\subsubsection{Kolizje wewnętrzne}
			\label{t:teoria:analiza:kolizje:wewn}
			
			
	\section{Zastosowanie GPU w symulacji tkanin}
	\label{t:teoria:gpu}
	
		\subsection{Architektura GPU}
		\label{t:teoria:gpu:architektura}
		
		\subsection{Porównanie GPU i CPU pod kątem architektury i zastosowań}
		\label{t:teoria:gpu:porownanie}
		
		\subsection{Zalety zastosowania GPU w symulacji tkanin}
		\label{t:teoria:gpu:zalety}
