\chapter{Opis aplikacji}
\label{t:praktyka}

	\section{Ogólna architektura aplikacji}
	\label{t:praktyka:ogolne}
	
	
	\section{Budowa i działanie silnika dla wizualizacji i zarządzania symulacją}
	\label{t:praktyka:silnik}
	
		\subsection{Trzon silnika}
		\label{t:praktyka:silnik:trzon}
		
		\subsection{Komunikacja z Androidem}
		\label{t:praktyka:silnik:andro}
		
		\subsection{Rendering}
		\label{t:praktyka:silnik:render}
		
		\subsection{Interfejs użytkownika}
		\label{t:praktyka:silnik:gui}
	
	
	\section{Budowa i działanie symulatora tkaniny}
	\label{t:praktyka:symulacja}
	
		\subsection{Obliczenia ruchu tkaniny}
		\label{t:praktyka:symulacja:ruch}
		
		\subsection{Rozwiązywanie kolizji}
		\label{t:praktyka:symulacja:kolizje}
		
		\subsection{Przeliczenie wektorów normalnych}
		\label{t:praktyka:symulacja:normalne}
	
	
	%\section{Budowa i działanie symulatora tkaniny na platformie Windows}	% - ni ma CUDY, będzie to samo
	%\label{t:praktyka:symulacjapc}
	
